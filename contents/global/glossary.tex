% =============================================================================
% When using latexmk, add this to .~/latexmkrc (otherwise glossary is empty)
% =============================================================================
%
% add_cus_dep('glo', 'gls', 0, 'run_makeglossaries');
% add_cus_dep('acn', 'acr', 0, 'run_makeglossaries');
%
% sub run_makeglossaries {
%   if ( $silent ) {
%     system "makeglossaries -q '$_[0]'";
%   }
%   else {
%     system "makeglossaries '$_[0]'";
%   };
% }
%
% push @generated_exts, 'glo', 'gls', 'glg';
% push @generated_exts, 'acn', 'acr', 'alg';
% $clean_ext .= ' %R.ist %R.xdy';
%
% =============================================================================
% When compiling manually (e.g. pdflatex), use 'makeglossaries'
% =============================================================================
%
% $> pdflatex main.tex
% $> makeglossaries main
% $> bibtex main.aux
% $> pdflatex main.tex
% $> pdflatex main.tex
%
% =============================================================================
% Create glossary entries
% =============================================================================
%
% \newglossaryentry{<key>}
% {
%     name=<name>,
%     plural=<plural>,
%     description={A name is an identifier.},
%     see=<another-key>
% }
%
% =============================================================================
% Use glossary entries
% =============================================================================
%
% \gls{}    - lower-case singular
% \glspl{}  - lower-case plural
%
% \Gls{}    - upper-case singular
% \Glspl{}  - upper-case plural
%
% =============================================================================

% TODO Remove whitespace between last word of desc. and implicit period if last word is an acronym (appending \xspace seems to work)

\newglossaryentry{tx}
{
    name=transaction,
    description={
        A sequence of operations that appears as a single atomic operation. It
        either succeeds or fails without any side effects
    }
}

\newglossaryentry{mmdb}
{
    name=main-memory database,
    description={
        A database that keeps all of its data in main memory. This enables data
        access with very low latency. Still, with volatile memory such as
        \ac{DRAM}, data is vulnerable to crashes and power failures which
        violates ACID. For this reason, most \acp{MMDB} must retain an
        up-to-date recovery partition in slower non-volatile memory which is
        copied back into main memory in case of a restart. Ensuring this kind of
        recovery is a prominent bottleneck of \acp{MMDB}
    }
}
