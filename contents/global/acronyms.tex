% Copyright (C)  2013 Jana Traue (jana.traue[at]tu-cottbus.de)
%
% Permission is granted to copy, distribute and/or modify this document
% under the terms of the GNU Free Documentation License, Version 1.3
% or any later version published by the Free Software Foundation;
% with no Invariant Sections, no Front-Cover Texts, and no Back-Cover Texts.
% A copy of the license is included in the file entitled "LICENSE".
%
% =============================================================================
% LATEX acronyms
% =============================================================================
%
%usage:
%	\ac{acronym} first occurence: complete text. thereafter, short form.
%	in footnotes: always short form (first time with foot note)
%	\acf{acronym} long form
%	\acs{acronym} short form
%	\acl{acronym} long form without kink (to acronym section)
%	\acfi{acronym} long form in italics, followed by short form (normal font)
%	\acp, \acfp, \acsp und \aclp create plural versions
%
\begin{en}
\chapter{Acronyms}
\end{en}
\begin{de}
\chapter{Abk\"urzungsverzeichnis}
\end{de}
\label{sec:acronyms}
%
\begin{multicols}{2}
\begin{acronym}[NVRAM]		% enter the longest acronym here in the braces

	\acro{RAM} {random access memory}
	\acro{DRAM} {dynamic RAM}
	\acro{NVRAM} {non-volatile RAM}
	\acro{PCM} {phase-change memory}
	\acro{MRAM} {magnetoresistive RAM}
	\acro{NVM} {non-volatile memory}

    \acro{MMDB} {main-memory database}
    \acro{KVS} {key-value store}

    \acro{MVCC} {multiversion concurrency control}
    \acro{SI} {snapshot isolation}
    \acro{2PL} {two-phase locking}

	\acro{UML} {Unified Modelling Language}
	\acro{BTU} {Brandenburg Technical University}

\end{acronym}
\end{multicols}
%
% EOF
%
