\chapter{A Key-Value Store with Serializable Transactions for NVRAM}
\label{ch:concept}

The aim of this thesis is to design a KVS with affordable serializable
transactions by exploiting the benefits of NVRAM. In order to provide the
necessary groundwork, the previous chapters give an overview on recent research
in NVRAM, KVS, and concurrency control in databases. It is further pointed out
that, as of this writing, there appears to be no previous work on leveraging
NVRAM for concurrency control in KVS or MMDB, respectively. While recent works
primarily see NVRAM as a means to reduce recovery overhead, this thesis explores
a different approach.

As mentioned in Chapter \ref{ch:kvs}, many transaction processing systems do not
support or encourage serializable transactions due to severe performance
degradations. Therefore, the idea is to use the benefits of NVRAM to make
serializable transactions affordable. While this may not provide the highest
possible transaction throughput, the aim is to achieve performance on a par with
non-serializing solutions for traditional storage. In other words, instead of
increasing maximum performance, this work attempts to increase performance with
maximum consistency.

NVRAM is especially significant for memory-resident databases as data no longer
need to copied to a much slower storage device for recoverability. Also,
restarts can be near-instantaneous as all data are already in memory and need
not be fetched. Therefore, the proposed concept is exclusively targeted at MMDB.
Given the vast complexity of fully-featured in-memory DBMS, it seems
appropriate, for an initial study, to resort to much more manageable KVS. Based
on whether the approach turns out to work well for KVS, it may still be applied
on MMDB in future work.

This chapter presents the concept for a NVRAM-aware KVS with serializable
transactions. After a a brief overview in the next section, follow-up sections
outline the architecture, concurrency control, and consistency measures.

\section{Overview}
\label{sec:concept-overview}

\section{Architecture}
\label{sec:concept-arch}

\section{Consistency}
\label{sec:concept-concistency}

\section{Concurrency Control}
\label{sec:concept-cc}
