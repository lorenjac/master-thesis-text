\chapter{A Key-Value Store with Serializable Transactions for NVRAM}
\label{ch:concept}

The aim of this thesis is to design a KVS with affordable serializable
transactions by exploiting the benefits of NVRAM. In order to provide the
necessary groundwork, the previous chapters give an overview on recent research
in NVRAM, KVS, and concurrency control in databases. It is further pointed out
that, as of this writing, there appears to be no previous work on leveraging
NVRAM for concurrency control in KVS or MMDB, respectively. While recent works
primarily see NVRAM as a means to reduce recovery overhead, this thesis explores
a different approach.

As mentioned in Chapter \ref{ch:kvs}, many transaction processing systems do not
support or encourage serializable transactions due to severe performance
degradations. Therefore, the idea is to use the benefits of NVRAM to make
serializable transactions affordable. While this may not provide the highest
possible transaction throughput, the aim is to achieve performance on a par with
non-serializing solutions for traditional storage. In other words, instead of
increasing maximum performance, this work attempts to increase performance with
maximum consistency.

NVRAM is especially significant for memory-resident databases as data no longer
need to copied to a much slower storage device for recoverability. Also,
restarts can be near-instantaneous as all data are already in memory and need
not be fetched. Therefore, the proposed concept is exclusively targeted at MMDB.
Given the vast complexity of fully-featured in-memory DBMS, it seems
appropriate, for an initial study, to resort to much more manageable KVS. Based
on whether the approach turns out to work well for KVS, it may still be applied
on MMDB in future work.

This chapter presents the concept for an NVRAM-aware KVS with serializable
transactions. After a a brief overview in the next section, follow-up sections
outline the architecture, concurrency control, and consistency measures.

\section{Overview}
\label{ch:concept-overview}
This section provides an overview on the concept of this work. For this purpose,
goals, assumptions, and design constraints are outlined. The section concludes
with an outline of the desired \ac{API} and practical examples.

\subsubsection{Goal}

The intent of this thesis is to determine whether \ac{MMDB} could exploit
\ac{NVRAM} to make transactions with strong consistency affordable. Given the
overwhelming complexity of full-scale \ac{DBMS}, this work resorts to in-memory
\ac{KVS}. \ac{NVRAM} significantly reduces the required recovery overhead. While
others have used this circumstance to increase transaction throughput alone,
this work chooses to leverage the headroom to compensate for the cost of
serializability. The goal is to design a serializing in-memory \ac{KVS} for
\ac{NVRAM} which performs on par with non-serializing \ac{KVS} based on volatile
\ac{RAM}.

\subsubsection{Assumptions}

The concept is based on several assumptions concerning both technical aspects
and workloads characteristics.

\paragraph{Hardware}

In order to take advantage of concurrency, the concept is designed for
multi-core architectures. While this increases the number of threads that can be
run in parallel, it also introduces synchronization issues for access to shared
memory which must be handled with care. However, to keep complexity manageable,
the concept refrains from distributed computing and targets single-node
databases. In accordance with recent research, it is assumed that volatile
\ac{RAM} will continue to be present and share the same memory interface with
\ac{NVRAM}. This reduces individual bandwidths but enables uniform access
methods. It is further assumed that target systems provide sufficient hardware
and software facilities to manage \ac{NVRAM}. Details concerning crash
consistency are provided in Chapter \ref{ch:concept-consistency}

\paragraph{Workloads}

When designing systems and transaction processors, in particular, it is helpful
to know in advance which kind of workloads are expected or should be given
priority. Given that many \ac{MMDB} are dominated by read operations
\cite{andrei2017sap}, this work is intended for read-mostly workloads. While
read-only transactions are supported, there seems to be no hard evidence on the
importance or quantity of such transactions. Likewise, long-running transactions
are not handled separately, as their share could not be determined.

\subsubsection{Design Constraints}

\paragraph{In-Memory Operation}

Since the intent of this work is to evaluate opportunities of \ac{NVRAM} for
\ac{MMDB}, the target \ac{KVS} must hold all its data in volatile or
non-volatile main memory with no disk storage involved. This way, access
latencies are limited to main memory rather than slower disk storage or
\ac{SSD}.

\paragraph{Transactions}

Contrary to full-grown \ac{MMDB}, a number of \ac{KVS} does not support
transactions that span multiple primitive operations. However, in order to allow
conceptual conclusions to \ac{MMDB}, it is important to maintain sufficient
generality. Therefore, the concept requires full-featured transactions as in
\ac{MMDB}. As a result, multiple operations, such as reading or writing an item,
may be enclosed within a transaction. Likewise, full ACID support is required to
guarantee sound transactional semantics. In order to guarantee strong
consistency and isolation in the presence of concurrent transactions,
serializability is a central requirement. Note that, in contrast to some
\ac{KVS} which are designed as caches the target \ac{KVS} in this work supports
durability. Concerning the nature of key-value pairs, this work imposes no
restrictions on their datatypes. Still, implementations are free to limit the
length of keys, for instance, if the underlying data structure requires it.

\subsubsection{API}

Given the simplistic nature of \ac{KVS}, this concept anticipates a narrow
\ac{API} that features the very basic building blocks of transactional
semantics. The \ac{API} can be described as a tuple of three instruction sets.
First, there are routines to create or manage instances of the \ac{KVS}. The
second set consists of routines to start and end transactions. Transactions are
managed through handles which are retrieved when creating them. Such transaction
handles are required for the third set of instructions, namely inserting or
deleting pairs and retrieving values. Table \ref{tab:api} gives an outline of
the intended \ac{API}.

\begin{figure}[!h]
    \centering
    \begin{tabular}{|l|l|}
        \hline
        \textbf{Function}          & \textbf{Description} \\
        \hline
        kvstore()                  & Create a key-value store instance \\
        begin() : tx               & Start a transaction \\
        begin\_ro() : tx            & Start a read-only transaction \\
        commit(tx) : bool          & Commit a transaction \\
        abort(tx) : void           & Abort a transaction \\
        get(key, tx) : value       & Retrieve value for a given key \\
        put(key, value, tx) : void & Insert a key-value pair \\
        remove(key, tx) : void     & Remove a key-value pair \\
        \hline
    \end{tabular}
    \caption{API of the intended key-value store.}
    \label{tab:api}
\end{figure}

This \ac{API} is sufficient to power a basic database with transactional
semantics. Listing \ref{lst:api_ex} shows an example where a transaction checks
whether the balance is negative and, if so, applies a penalty.

\begin{lstlisting}
kvstore kvs;

/* ... lots of transactions ... */

auto tx = kvs.begin();
auto deb = kvs.get("debit", tx);
auto sav = kvs.get("saving", tx);
if (getBalance(sav, deb) < 0.0) {
	kvs.put("debit", applyInterest(deb), tx);
}
tx->commit();
\end{lstlisting}\label{lst:api_ex}

For an example with concurrent transactions, see Listing \ref{lst:api_ex2}.
Here, two transactions are executed concurrently which leads to a data race.

\todo[inline]{Insert listing/control flow for concurrent transactions}


\section{Architecture}
\label{ch:concept-arch}
After outlining goals and preconditions in the previous section, this section
presents the architecture of the KVS. It covers all aspects, except crash
consistency and concurrency control which are dealt with in more detail in
subsequent sections.

The KVS is designed for multi-core architectures and relies on both volatile and
non-volatile memory attached to the system memory interface. In order to take
advantage of both types of memory, the KVS is designed as a two-level store
which only updates NVRAM when a transaction commits. Consistency across crashes
is ensured with existing hardware primitives and upcoming platform features.
Concurrent transactions are controlled by a serializable variant of SI.
Background on these design decisions is given below.

\subsection{System Architecture}

Designing a runtime-critical software such as databases not only involves
knowledge about expected workloads but also about the underlying computing
device. While workloads have been discussed earlier, this section describes the
system architecture of the intended KVS.

\subsubsection{Single-Node}

First and foremost, the KVS is designed for a single-nodes. Even though
distributed databases are fairly common, there seems to be no apparent reason
for them to reveal any more insight on leveraging NVRAM for concurrency. Also
distributed systems involve much more complex mechanisms such as consensus among
distributed transactions, all of which are beyond the scope of this work.
However, future work should investigate whether the conclusions of this work
also hold for distributed databases.

\subsubsection{Multi-Core Processors}

In order to achieve scalable transaction throughput through concurrency, the
target system is a multi-core architecture. That means, the system features one
or more processors with multiple cores, where each core may support multiple
hardware threads. On such a system, each transaction is executed in the context
of a thread which is scheduled and assigned to a processor core by the operating
system. [Should future work address many-cores? Could memory bandwidth (bus
contention) become a problem there?] In order to coordinate their work,
processors usually communicate via some form of chip interconnect. This work
makes no assumptions concerning the nature of such networks or their topologies.

\subsubsection{Hybrid Memory Architecture}

Recent research shows that on traditional hardware it is advisable to continue
integrating volatile RAM together with NVRAM. The reason is that not all data is
meant to be durable which is especially true for NVRAM where crash consistency
is linked with considerable overhead. Manufacturing NVRAM is still challenging,
especially in terms of access latency and endurance, but it is expected that
these issues will be resolved in the near future. Therefore, in an effort to
combine the benefits of both technologies, the memory subsystem is required to
feature both volatile RAM and NVRAM. [<-- Move to top of segment?] In accordance
with recent research it is assumed that both kinds of memory can be access
through the same memory interface. This work assumes a shared memory
architecture. That is, processors may one or more private cache levels but main
memory is accessible to all processors. Conceptually, cache coherence is not
required but has the advantage that less effort is spent on coordinating
concurrent access to shared data.

The KVS is designed to exclusively reside in main memory. All data that is not
required across restarts is stored in volatile RAM, whereas all other data are
stored in NVRAM. Multiple recent works have demonstrated that NVRAM can be used
to build MMDB without conventional non-volatile storage such as hard drives. As
a result, ensuring recovery, which has always been an inherent bottleneck of
MMDB, can be eliminated. In addition, near-instantaneous restarts become
feasible. As a consequence, conventional storage is not part of the concept for
this KVS. While such components may very well be present in a system, they are
never used to store any data of the KVS other than its binaries. This way, data
access incurs no I/O and restarts do not have to fetch data from slower storage
devices. In return, candidate systems must provide sufficient NVRAM capacity to
hold the entire database.

A disadvantage of this approach is that the size of the database is bounded by
the amount of available NVRAM. In contrast, MMDB usually allow for larger data
sets by keeping frequently used data in memory, while others are moved to slower
mass storage media. However, main memory capacities have been steadily growing
and NVRAM capacities are projected to have at least twice the capacity of DRAM.
Another drawback is recoverability in case of device failures. Mass storage not
only scales better in terms of capacity, it can also be used to employ
redundancy through RAID, for instance. With NVRAM, both capacity and scalability
are lower, so employing information redundancy may be prohibitively expensive.
Without such measures of fault tolerance, however, a single failed NVRAM module
may lead to permanent data loss. This issue is not tackled in this thesis and is
therefore left for future work.

\subsection{Key-Value Store Design}

This section describes the software architecture of the KVS. That includes the
operation principle in terms of storage and concurrency as well as the general
structure.

\subsubsection{Two-Level Store}

As mentioned above, the KVS resides entirely in main memory. This has the
disadvantage that the size of the databases is bounded by the total NVRAM
capacity but allows for fast access to all parts of the database.

With both DRAM and NVRAM in a system, the KVS has access to memory that is fast
but volatile or slightly slower but durable. As pointed out in Chapter
\ref{ch:nvram}, NVRAM latencies mainly affect writes and are attributed to both
technology parameters and crash consistency measures. Details on how the KVS
preserves consistency across crashes are given in Chapter
\ref{sec:concept-consistency}. In order to benefit from both memory
technologies, the KVS employs a two-level store architecture.

\subsubsection{Multiversioning}

\subsubsection{Structures}


% \section{Concurrency Control}
% \label{ch:concept-cc}

\section{Crash Consistency}
\label{ch:concept-consistency}
Posting updates to NVRAM is different from traditional non-volatile storage
media. With NVRAM, write-back is not directly observable by software and the
underlying memory subsystem is unaware of transactional semantics. As a result,
updates may be reordered or get stuck in store buffers and caches, thus
threatening consistency across crashes.

As pointed out in Chapter \ref{ch:nvram-consistency}, programmers need to
manually enforce write-back, in order to preserve consistency. At first, it is
important to enforce strict program order for transactionally related memory
operations. This can be done with memory barriers or fences. While such measures
usually include flushing store buffers, in-flight updates may still get stuck in
caches. To prevent this, cache line flushes or non-temporal stores could be
used. Even when reaching a memory controller, stores are once again buffered to
speed up subsequent reads to that item. While earlier works anticipated
designated flush instructions for controller buffers, both researchers and
hardware vendors have agreed on the platform feature ADR. When power fails, it
receives a signal and utilizes the remaining electrical energy to flush all
memory controller buffers.

With the exception of ADR, all of these methods can significantly increase
execution latencies as they work against many aspects of modern microprocessors.
Deferred write-backs for example are useful to decrease access times for
recently written data. Flushing store buffers, however, also affects data that
are not involved in transactions. In addition, enforcing strict program order
usually results in pipeline stalls. Since there are no other options at the
moment, programmers have to meticulously manage and optimize updates to NVRAM.

Summing up, preserving consistency across crashes is necessary but also
introduces adverse effects on performance. In accordance with recent research,
this work requires the following features

\begin{itemize}
    \item means to enforce program order for memory accesses
    % \item means to flush potential memory order buffers
    \item means to flush cashes or individual cache lines
    \item ADR
\end{itemize}

Both memory order enforcement and fine-grained cache flushes are provided by
many instruction sets including \code{x86\_64}, SPARC, and IBM POWER and zSeries
\cite{mckenney2007memory}. ADR, on the other hand, is a new platform feature
that is already supported on some systems. Therefore, no changes to existing
hardware are needed.

