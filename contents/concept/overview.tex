This section provides an overview on the concept of this work. For this purpose,
goals, assumptions, and design constraints are outlined. The section concludes
with an outline of the desired API and practical examples.

\subsubsection{Goal}

The intent of this thesis to determine whether MMDB could exploit NVRAM to make
transactions with strong consistency affordable. Given the overwhelming
complexity of full-scale DBMS, this work resorts to in-memory KVS. NVRAM
significantly reduces the required recovery overhead. While others have used
this circumstance to increase transaction throughput alone, this work chooses to
leverage the headroom to compensate for the cost of serializability. The goal is
to design a serializing in-memory KVS for NVRAM which performs on par with
non-serializing KVS based on volatile RAM.

\subsubsection{Assumptions}

The concept is based on several assumptions concerning both technical aspects
and workloads characteristics.

\paragraph{Hardware}

In order to take advantage of concurrency, the concept is designed for
multi-core architectures. While this increases the number of threads that can be
run in parallel, it also introduces synchronization issues for access to shared
memory which must be handled wit care. However, to keep complexity manageable,
the concept refrains from distributed computing and targets single-node
databases. In accordance with recent research, it is assumed that volatile RAM
will continue to be present and share the same memory interface with NVRAM. This
reduces individual bandwidths but enables uniform access methods. It is further
assumed that target systems provide sufficient hardware and software facilities
to manage NVRAM.

\paragraph{Workloads}

When designing systems and transaction processors, in particular, it is helpful
know in advance which kind of workloads are expected or should be given
priority. Given that many MMDB are dominated by read operations, this work is
intended for read-mostly workloads \cite{andrei2017sap}. While read-only
transactions are supported, there seems to be no hard evidence on the importance
or quantity of such transactions. Likewise, long-running transactions are not
handled separately, as their share could not be determined.

\subsubsection{Design Constraints}

\paragraph{In-Memory Operation}

Since the intent of this work is to evaluate opportunities of NVRAM for MMDB,
the target KVS must hold all its data in volatile or non-volatile main memory
with no disk storage involved. This way, access latencies are limited to main
memory rather than slower disk storage or SSD.

\paragraph{Transactions}

Contrary to full-grown MMDB, a number of KVS does not support transactions that
span multiple primitive operations. However, in order to allow conceptual
conclusions to MMDB, it is important to maintain sufficient generality.
Therefore, the concept requires full-featured transactions as in MMDB. As a
result, multiple operations, such as reading or writing an item, may be enclosed
within a transaction. Likewise, full ACID support is required to guarantee sound
transactional semantics. In order to guarantee strong consistency and isolation
in the presence of concurrent transactions, serializability is a central
requirement. Note that, in contrast to some KVS which are designed as caches the
target KVS in this work supports durability. Concerning nature of key-value
pairs, this work imposes no requirements regarding their datatypes. Still,
implementations are free to, for instance, limit the length of keys if the
underlying data structures requires it.

\subsubsection{API \& Examples}

Given the simplistic nature of KVS, this concept anticipates a narrow API that
features the very basic building blocks of transactional semantics. The API can
be described as a tuple of three instruction sets. First, there are routines to
create or manage instances of the KVS. The second set consists of routines to
start and end transactions. Transactions are managed through handles which are
retrieved when creating them. Such transaction handles are required for the
third set of instructions, namely inserting or deleting pairs and retrieving
values. Table \ref{tab:api} gives an outline of the intended API.

\begin{figure}[!h]
    \centering
    \begin{tabular}{|l|l|}
        \hline
        \textbf{Function}          & \textbf{Description} \\
        \hline
        kvstore()                  & Create a key-value store instance \\
        begin() : tx               & Start a transaction \\
        begin\_ro() : tx            & Start a read-only transaction \\
        commit(tx) : bool          & Commit a transaction \\
        abort(tx) : void           & Abort a transaction \\
        get(key, tx) : value       & Retrieve value for a given key \\
        put(key, value, tx) : void & Insert a key-value pair \\
        remove(key, tx) : void     & Remove a key-value pair \\
        \hline
    \end{tabular}
    \caption{API of the intended key-value store.}
    \label{tab:api}
\end{figure}

This API is sufficient to power a basic database with transactional semantics.
Listing \ref{lst:api_ex} shows an example where a transaction checks whether the
balance is negative and, if so, applies a penalty.

\begin{lstlisting}
// create key-value store
kvstore kvs;

// ... lots of transactions ...

// begin transaction
auto tx = kvs.begin();

// retrieve current balance
auto balance = kvs.get("balance");

// test if balance is negative
if (isNegative(balance)) {

	// apply penalty
	kvs.put("balance", decrement(balance), tx);
}

// commit transaction
tx->commit();
\end{lstlisting}\label{lst:api_ex}

For an example with concurrent transactions, see Listing \ref{lst:api_ex2}.
Here, two transactions are executed concurrently which leads to a data race.

\todo[inline]{Example with concurrency}
