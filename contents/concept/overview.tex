This section provides an overview of the concept. For this purpose, goals,
assumptions, and design constraints are outlined. The section concludes with an
outline of the desired \ac{API} and practical examples.

\subsubsection{Goal}

The intent of this thesis is to determine whether \ac{MMDB} could exploit
\ac{NVRAM} to make transactions with strong consistency affordable. Given the
overwhelming complexity of full-scale \ac{DBMS}, this work resorts to in-memory
\ac{KVS}. \ac{NVRAM} significantly reduces the required recovery overhead. While
others have used this condition to increase transaction throughput alone,
this work chooses to leverage the headroom to compensate for the cost of
serializability. The goal is to design a serializing in-memory \ac{KVS} for
\ac{NVRAM} which performs on par with non-serializing \ac{KVS} based on volatile
\ac{RAM}.

\subsubsection{Assumptions}

The concept is based on several assumptions concerning both technical aspects
and workloads characteristics.

\paragraph{Hardware}

In order to take advantage of concurrency, the concept is designed for
multi-core architectures. While this increases the number of threads that can be
run in parallel, it also introduces synchronization issues for access to shared
memory which must be handled with care. However, to keep complexity manageable,
the concept refrains from distributed computing and targets single-node
databases. In accordance with recent research, it is assumed that volatile
\ac{RAM} will continue to be present and share the same memory interface with
\ac{NVRAM}. This reduces individual bandwidths but enables uniform access
methods. It is further assumed that target systems provide sufficient hardware
and software facilities to manage \ac{NVRAM}. Details concerning crash
consistency are provided in Chapter \ref{ch:concept-consistency}

\vfill

\paragraph{Workloads}

When designing systems and transaction processors, in particular, it is helpful
to know in advance which kind of workloads are expected or should be given
priority. Given that many \acp{MMDB} are dominated by read operations
\cite{andrei2017sap}, this work is intended for read-mostly workloads. While
read-only transactions are supported, there seems to be no hard evidence on the
importance or quantity of such transactions. Likewise, long-running transactions
are not handled separately, as their share could not be determined.

\subsubsection{Design Constraints}

\paragraph{In-Memory Operation}

Since the intent of this work is to evaluate opportunities of \ac{NVRAM} for
\ac{MMDB}, the target \ac{KVS} must hold all its data in volatile or
non-volatile main memory with no disk storage involved. This way, access
latencies are limited to main memory rather than slower disk storage or
\ac{SSD}.

\paragraph{Transactions}

Contrary to full-grown \ac{MMDB}, a number of \acp{KVS} do not support
transactions that span multiple primitive operations. However, in order to allow
conceptual conclusions to \ac{MMDB}, it is important to maintain sufficient
generality. Therefore, the concept requires full-featured transactions as in
\ac{MMDB}. As a result, multiple operations, such as reading or writing an item,
may be enclosed within a transaction. Likewise, full ACID support is required to
guarantee sound transactional semantics. In order to guarantee strong
consistency and isolation in the presence of concurrent transactions,
serializability is a central requirement. Note that, in contrast to some
\acp{KVS}, which are designed as caches, the target \ac{KVS} in this work
supports durability. Concerning the nature of key-value pairs, this work imposes
no restrictions on their data types. Still, implementations are free to limit
the length of keys, for instance, if the underlying data structure requires it.

\subsubsection{API}

Given the simplistic nature of \ac{KVS}, this concept anticipates a narrow
\ac{API} that features the very basic building blocks of transactional
semantics. The \ac{API} can be described as a tuple of three instruction sets.
First, there are routines to create or manage instances of the \ac{KVS}. The
second set consists of routines to start and end transactions. Transactions are
managed through handles which are retrieved when creating them. Such transaction
handles are required for the third set of instructions, namely inserting or
deleting pairs and retrieving values. Table \ref{tab:api} gives an outline of
the intended \ac{API}.

\begin{figure}[!h]
    \centering
    \begin{tabular}{|ll|l|}
        \hline
        \textbf{Function}          &  & \textbf{Description} \\
        \hline
        kvstore()           & : void  & Create a key-value store instance \\
        begin()             & : tx    & Start a transaction \\
        commit(tx)          & : bool  & Commit a transaction \\
        abort(tx)           & : void  & Abort a transaction \\
        get(key, tx)        & : value & Retrieve value for a given key \\
        put(key, value, tx) & : bool  & Insert a key-value pair \\
        remove(key, tx)     & : bool  & Remove a key-value pair \\
        \hline
    \end{tabular}
    \caption{API of the intended key-value store.}
    \label{tab:api}
\end{figure}

This \ac{API} is sufficient to power a basic database with transactional
semantics. Listing \ref{lst:api_ex} shows an example where a transaction checks
whether the balance is negative and, if so, applies a penalty.

\begin{lstlisting}[caption={An example program showcasing the intended API of the KVS.}, captionpos=b, label=lst:api_ex]
kvstore kvs;

/* ... lots of transactions ... */

auto tx = kvs.begin();
auto deb = kvs.get("debit", tx);
auto sav = kvs.get("saving", tx);
if (getBalance(sav, deb) < 0.0) {
	kvs.put("debit", applyInterest(deb), tx);
}
tx->commit();
\end{lstlisting}

% For an example with concurrent transactions, see Listing \ref{lst:api_ex2}.
% Here, two transactions are executed concurrently which leads to a data race.

% \todo[inline]{Insert listing/control flow for concurrent transactions}
