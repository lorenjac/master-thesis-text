As pointed out in Chapter \ref{sec:nvram-consistency}, posting updates to NVRAM
is different from traditional non-volatile storage media. With NVRAM, write-back
is not directly observable by software and the underlying memory subsystem is
unaware of transactional semantics. As a result, updates may be reordered or get
stuck in store buffers and caches, thus threathing consistency across crashes.

The current state of affairs requires programmers to manually enforce
write-back, in order to preserve consistency. At first, it is important to
enforce strict program order for transactionally related memory operations. This
can be done with memory barriers or fences. While such measures usually include
flushing store buffers, in-flight updates may still get stuck in caches. To
prevent this, cache line flushes or non-temporal stores could be used. When
reaching a memory controller, stores are usually enqueued in another buffer to
speed up subsequent reads to that item. While earlier works anticipated
designated flush instructions for controller buffers, both researchers and
hardware vendors have agreed on the platform feature ADR. When power fails, it
receives a signal and utilizes the residual electric energy to flush memory
controller buffers. With the exception of ADR, all of these measures can
significantly increase execution latencies as they work against many aspects of
modern microprocessors. Deferred write-backs for example are useful to decrease
access times for recently written data. Flushing store buffers, however, also
affects data that are not involved in transactions. In addition, strict program
order usually result in pipeline stalls. Since there are no other options at the
moment, programmers have to meticulously manage and optimize updates to NVRAM.

Summing up, preserving consistency across crashes is necessary but also
introduces adverse effects on performance. In particular, this concept
anticipates to provide the following features

\begin{itemize}
    \item means to enforce program order for memory accesses
    \item means to flush potential memory order buffers
    \item means to flush individual cache lines
    \item ADR
\end{itemize}
