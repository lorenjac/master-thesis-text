Posting updates to NVRAM is different from traditional non-volatile storage
media. With NVRAM, write-back is not directly observable by software and the
underlying memory subsystem is unaware of transactional semantics. As a result,
updates may be reordered or get stuck in store buffers and caches, thus
threatening consistency across crashes.

As pointed out in Chapter \ref{sec:nvram-consistency}, programmers need to
manually enforce write-back, in order to preserve consistency. At first, it is
important to enforce strict program order for transactionally related memory
operations. This can be done with memory barriers or fences. While such measures
usually include flushing store buffers, in-flight updates may still get stuck in
caches. To prevent this, cache line flushes or non-temporal stores could be
used. Even when reaching a memory controller, stores are once again buffered to
speed up subsequent reads to that item. While earlier works anticipated
designated flush instructions for controller buffers, both researchers and
hardware vendors have agreed on the platform feature ADR. When power fails, it
receives a signal and utilizes the remaining electrical energy to flush all
memory controller buffers.

With the exception of ADR, all of these methods can significantly increase
execution latencies as they work against many aspects of modern microprocessors.
Deferred write-backs for example are useful to decrease access times for
recently written data. Flushing store buffers, however, also affects data that
are not involved in transactions. In addition, enforcing strict program order
usually results in pipeline stalls. Since there are no other options at the
moment, programmers have to meticulously manage and optimize updates to NVRAM.

Summing up, preserving consistency across crashes is necessary but also
introduces adverse effects on performance. In accordance with recent research,
this concept anticipates to provide the following features

\begin{itemize}
    \item means to enforce program order for memory accesses
    \item means to flush potential memory order buffers
    \item means to flush individual cache lines
    \item ADR
\end{itemize}

Apart from ADR, all of these features are provided by many instruction sets
including \code{x86\_64}, SPARC, and IBM POWER and zSeries
\cite{mckenney2007memory}. ADR is platform feature that is required for future
systems featuring NVRAM. Therefore, no changes to existing hardware are
required.
