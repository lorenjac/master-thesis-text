\chapter{Conclusions and Future Work}
\label{ch:summary}

Recent advances in non-volatile memory research indicate that fast high-density
NVRAM is available in the near future. Later research found that novel NVRAM can
significantly increase transaction throughput in main-memory databases. Still,
for performance reasons, many databases resort to relaxed isolation of
concurrent transactions. Therefore, this thesis investigates if NVRAM can
provide sufficient leverage to make the isolation level of serializability
affordable.

Due to the complexity of full-featured MMDB this work is focused entirely on
key-value stores. Based on recent research, a concept for an NVRAM-resident
key-value store with serializable transactions was developed. Subsequently,
\midas, a prototype of the concept was implemented. \midas relies on MVCC and
implements a serializable variant of Snapshot Isolation. Access to potential
NVRAM and durable data structures is managed through PMDK, a state of the art
library for programming against NVRAM.

In order to evaluate the concept, a synthetic benchmark over \ttp was developed.
In the experiment, \midas is compared to \echo, another NVRAM-resident KVS but
with non-serializable transactions. According to the results, \midas mostly
performs much worse than \echo. Except at low contention, \midas has lower
throughput, higher abort rates, and fails to utilize additional processors.
However, it was found that the results can be largely attributed to an
implementation flaw in \midas. A modified variant of \midas achieved higher
\tput and better scalability than \echo at comparable abort rates. These are
promising results and indicate that the concept of \midas may be worth pursuing.

\paragraph{Future Work}

There are some aspects of this work that should see further attention.

\begin{itemize}
    \item further experiments
    \item distributed transactions
    \item concurrent data structures for NVRAM
    \item better garbage collection
\end{itemize}

First and foremost, additional experiments are needed to better evaluate the
concept of powering serializability with NVRAM. For example, it should be
investigated how \midas compares to other solutions, such as BerkeleyDB
\cite{olson1999berkeley}, that support serializability but use disk storage for
recovery. Ultimately, the concept of \midas could be evaluated for full-featured
MMDB. Since many modern KVS and MMDB, such as DynamoDB \cite{decandia2007dynamo}
or HANA \cite{lee2013sap}, are distributed systems, it would be interesting to
see if the concept of \midas can be beneficial to distributed transactions.

On the implementation side, there is also room for improvement. First, the
durable data structures in \midas are not concurrent and require locking which
introduces undesirable bottlenecks. Therefore, future work should consider the
design or adaptation of highly-concurrent data structures for NVRAM. A
possibility would be to port existing implementations, such as the concurrent
hash table \emph{libcuckoo} \cite{li2014algorithmic}, to NVRAM.

\midas is based on MVCC which requires garbage collection to release versions
that are no longer reachable by any transaction. The current design of \midas
only performs garbage collection on startup to reduce resource contention during
runtime. Therefore, it could be investigated how online garbage collection
can be integrated in serializable MVCC with reasonable overhead.
