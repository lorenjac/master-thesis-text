There are numerous examples for applications of \ac{NVRAM}. While earlier works
often considered \ac{NVRAM} as a way to improve fault tolerance, recent research
suggests a broader range of applications. This development is especially driven
by recent advances in the manufacture of standalone \ac{NVRAM}.

\paragraph{Fault-Tolerance}

As pointed out earlier, a well-known use case of \ac{NVRAM} is to increase fault
tolerance towards crashes. The goal is to retain main memory content even in
case of a crash, for instance by an abrupt power loss \cite{molina1992main,
eich1986main}. This way, critical data such as logs of file systems or databases
remain durable and can be used to recover and even complete unfinished
operations such as making a transaction durable \cite{liskov1991replication,
chen1996rio}. In the past, such solutions relied on battery-backed \ac{RAM}.
This was subject to criticism as batteries only have a limited charge to ensure
durability. Also, batteries degrade over time and need to be maintained to
prevent unexpected failure \cite{molina1992main}. Therefore, modern \ac{NVRAM}
solutions which no longer require peripherals are a welcome improvement in this
area.

\paragraph{Disk Caches}

A significant amount of research on \ac{NVRAM} is dedicated to mitigating the
\ac{IO} bottleneck imposed by traditional disk storage. One way to do so is to
defer disk \ac{IO} via durable disk caches \cite{chen1996rio, wu1994envy}. When
an object on disk is requested, it is moved to the disk cache. Once an object is
cached, it may be read or modified without accessing the underlying disk.
Write-back is only required when there is not enough space for an incoming cache
item. In doing so, the number of data accesses involving disk \ac{IO} can be
greatly reduced. Many operating systems including Linux or FreeBSD mimic this
concept through the use of page caches in volatile memory. The difference is
that volatile caches need to be flushed at some point which requires careful
resource management. \ac{NVRAM} caches on the other hand, only need to evict
items when there is no slot for an incoming item.

\paragraph{Disk Replacement}

Another approach is to treat \ac{NVRAM} as an equivalent to traditional disk
storage. Early works, which were strongly influenced by the lack of
high-capacity \ac{NVRAM}, proposed hybrid storage systems where disk storage was
used in conjunction with \ac{NVRAM} \cite{wang2002conquest, miller2001hermes}.
These works have very similar assignment policies in that they only store small
files such as metadata or libraries in \ac{NVRAM} whereas larger files remain on
disk. While this does not remove disk access as a common bottleneck, it
certainly alleviates latency for some frequently accessed files. In that regard,
\ac{NVRAM}-complemented disk storage systems are similar to those with
\ac{NVRAM} disk caches.

\paragraph{Modern Use Cases}

For the time being, many applications will have to deal with a scarcity of
\ac{NVRAM}. But as improving technologies achieve higher capacities with better
parameters, new system architectures become feasible. In some cases, traditional
storage may be eliminated altogether, making \ac{NVRAM} the primary storage
medium. A prominent use case for this architecture are \acp{MMDB}. These
databases reside entirely in main memory which drastically reduces latencies
when accessing their data. However, they are also vulnerable to crashes, as main
memory is still mostly volatile. In order to prevent data loss, \acp{MMDB} have
to mirror the entire database to non-volatile memory. For that, they perform
logging or checkpointing to synchronize individual or groups of changes to
non-volatile memory. When the database is restarted, for example in response to
a crash, checkpoints and logging information are used to recover the most-recent
state of the database that was durable at the time of the crash. That is,
\acp{MMDB} require non-volatile memory for the sole purpose of recovery.
Recurring recovery measures such as logging have been a long-standing issue with
\acp{MMDB} as they incur expensive disk \ac{IO}, thus limiting transaction
throughput \cite{molina1992main, wust2012efficient,
malviya2014rethinking}. In addition, restarts reduce the availability of a
system as recovering large databases from slow storage can be time-consuming.
With \ac{NVRAM} on the other hand, any \ac{MMDB} would be implicitly durable,
hence making disk \ac{IO} obsolete. Moreover, it has been shown that with
\ac{NVRAM} logging is no longer necessary which paves the way for instantaneous
restarts \cite{oukid2015instant}. The concept of \ac{NVRAM}-based \ac{MMDB} is
especially promising as some upcoming variants of \ac{NVRAM} are projected to
feature larger capacities than conventional \ac{DRAM} \cite{lee2009architecting,
zilberberg2013phase, dulloor2014system}. For that reason, recent research has
investigated \ac{NVRAM}-aware designs for \ac{MMDB} ranging from \acp{KVS}
\cite{bailey2013exploring, zhou2016nvht, wu2016nvmcached} to full-fledged
database systems \cite{oukid2015instant, schwalb2016hyrise, andrei2017sap}.
Research results in these areas are central to this work and are reflected in
chapter \ref{ch:nvram} through \ref{ch:concept}. Given their importance for this
thesis, existing \acp{KVS} for \ac{NVRAM} are looked into in more detail at the
end of chapter \ref{ch:kvs}.

\paragraph{Experimental Use Cases}

While most works aim to improve existing architectures, some explore different
computation models. A recent example is a proposal to use \ac{NVRAM} to enable
on-chip machine learning. The idea is to move away from the well-known Von
Neumann architecture and implement \ac{ANN} by means of \ac{NVRAM}
\cite{fumarola2016accelerating}. \acp{ANN} perform a weighted sum over their
inputs before an activation function classifies the result. But before
satisfiable results can be obtained, \acp{ANN} need to be trained by properly
adjusting the scalar weights of their inputs. It has been suggested that with
\ac{NVRAM} weight adjustment could be performed on-chip where updated weights
would be durable without the need for write-back. However, neither artificial
intelligence nor alternative computation models are subject of this work.

Most proposals concerning \ac{NVRAM} assume the presence of volatile \ac{RAM}.
\cite{oukid2017data}. The reason behind this assumption is that not all parts of
memory may be intended for durability. Nonetheless, recent research suggests
that systems exclusively based on \ac{NVRAM} can be built
\cite{narayanan2012whole, courtland2016can}. Clearly, such an architecture would
have severe consequences for both operating systems and applications
\cite{bailey2011operating}. For example, operating system processes would remain
in memory even if terminated. On the one hand, it could significantly accelerate
the procedure of invoking a process. On the other hand, all data belonging to a
process' address space would be durable even if they were corrupted by a crash.
Other issues are concerned about memory management, device drivers, and vital
information. An early prototype of such a system is currently in development
\cite{courtland2016can}. This topic however, is beyond the scope of this work
and it is henceforth assumed that volatile \ac{RAM} will co-exist with
\ac{NVRAM}.

\newpage

As shown above, \ac{NVRAM} provides an opportunity to improve existing
architectures and even create new computation models. Given sufficient
parameters such as capacity and endurance, \ac{NVRAM} could resolve the \ac{IO}
bottleneck of non-volatile storage media. Especially systems such as \acp{MMDB}
have shown considerable gains in transaction throughput and recovery times.
Consequently, \acp{MMDB} and in particular \acp{KVS} are at the center of this
work.
