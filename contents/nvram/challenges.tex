Despite the conceivable advantages of \ac{NVRAM}, there are also challenges to
be addressed. Although most issues are of practical nature there also are
conceptual concerns.

\paragraph{Unintended Durability}

The key feature of \ac{NVRAM} is to retain its data across restarts. However,
not all data are necessarily intended to be durable. Notable examples include
transient, confidential, and corrupt data. The former comprises data which may
not be valid after a system restart, as is the case with data related to machine
or device state.

Other data such as passwords, encryption keys, or decrypted data may be
confidential and should not be durable. It has been shown that even volatile
\ac{RAM} holds its charge long enough so that a module can be moved to an
attacker's machine for read-out \cite{halderman2008lest}. Despite being
countered with hardware scramblers, researchers still managed to apply
variations of the technique and obtain vital information
\cite{yitbarek2017cold}. Such attacks would be trivial on \ac{NVRAM}
\cite{bailey2011operating}, but that is beyond the scope of this work.

% By cooling the module, capacitor
% discharge can be slowed so that it can be moved to another machine where its
% content may be read and parsed. Despite being countered with hardware
% scramblers, researchers still managed to apply variations of the technique and
% obtain vital information \cite{yitbarek2017cold}. Such attacks would be trivial
% on \ac{NVRAM}, as durability is its primary feature \cite{bailey2011operating}. Of
% course, confidential data could be overwritten with zeros after usage, but there
% is always a possibility that a crash might prevent such clean up tasks from
% completing. That said, sensitive data should at all times remain in volatile
% memory and be nulled after use. Although information security is an important
% matter, this thesis does not address such issues. Although it is important,
% information security in terms of attack resilience contributes little to the aim
% of this thesis and is therefore not addressed further. Also, since volatile \ac{RAM}
% continues to be available, there is no need to use \ac{NVRAM} for sensitive data.

When an operating system or application behaves in an erratic fashion or
crashes, it may produce corrupt data in memory. This is called a stray write. In
this case, systems incorporating \ac{NVRAM} could face durable memory
corruptions \cite{condit2009better, venkataraman2011consistent}. In contrast to
conventional non-volatile memory, \ac{NVRAM} is particularly vulnerable as it is
directly accessible through the \ac{CPU}. However, it has been shown that,
compared to disk storage, stray writes do not occur significantly more often in
\ac{NVRAM} \cite{chen1996rio}. Therefore, stray writes are not an issue in this
work.

% When an operating system or application behaves in an erratic fashion or crashes, it
% may produce corrupt data in memory. Unless memory is cleared or rewritten,
% systems incorporating \ac{NVRAM} could face durable memory corruptions. \ac{NVRAM} on the
% other hand is expected to be connected to the memory bus, enabling unbuffered
% access through virtual memory addresses. This makes \ac{NVRAM} vulnerable to stray
% writes \cite{condit2009better, venkataraman2011consistent}. However, it has been
% shown that, compared to disk storage, stray writes do not occur significantly
% more often in \ac{NVRAM} \cite{chen1996rio}. That said, stray writes are not
% considered an issue in this work.

\paragraph{Memory Management}

\ac{NVRAM} is a new type of memory that can also be used as durable mass
storage. In order to benefit from this new technology, both platforms and
operating systems need to find ways to efficiently manage it. There are several
issues to be addressed in this area.

% Memory Interface

An important aspect in managing \ac{NVRAM} is the memory interface. Recent
research suggests that \ac{NVRAM} will be attached to the system memory bus
using the \ac{DIMM} format known from \ac{DRAM} \cite{volos2017whisper,
oukid2017data, andrei2017sap, intel2017nvdimm}. A decisive advantage of this
approach is much lower latencies compared to the alternative \ac{IO} bus.
Another reason is that in a previous effort to produce \ac{NVRAM}, known as
\ac{NVDIMM}, modules have also been integrated this way \cite{dulloor2014system,
huang2014design}. Consequently, system designers can build on an existing
software stack. Still, there are drawbacks to be considered. Clearly, the number
of available \ac{DIMM} slots in a machine is limited, so \ac{NVRAM} may not
scale well for mass storage. That situation is especially relevant in hybrid
systems containing both \ac{RAM} and \ac{NVRAM}. Also, in hybrid systems both
kinds of memory are likely to be attached to the same memory interface thus
sharing its bandwidth.

With \ac{NVRAM} devices integrated into the system, programmers still need a way
to access it. Several approaches have been proposed to this end
\cite{volos2017whisper}. While it is always possible to operate on \ac{NVRAM} by
mapping individual device regions into virtual memory, there are considerable
weaknesses to this approach \cite{condit2009better, volos2011mnemosyne,
dulloor2014system, volos2017whisper}. A major challenge of working with
\ac{NVRAM} is to provide consistency guarantees across possible system failures.
Yet, systems are largely unaware of these circumstances. With raw device access
which is already error-prone, the complex task of preserving consistency is
handed to the programmer. Another challenge is that virtual memory mappings are
volatile and may no longer be valid after a restart.

Therefore, it has been proposed to rely on dedicated high-level programming
primitives as in Mnemosyne, NV-Heaps, and NVML \cite{volos2011mnemosyne,
coburn2011nv_heaps, intel2017nvml}. These systems provide interfaces for memory
allocation and consistent updates based on transactions. An important
distinction to the previous low-level approach is that memory is accessed
through an \ac{NVRAM}-aware \ac{API} instead of basic load and store statements.
The difference is that the latter have no knowledge of non-volatile memory and
its implications.

\begin{figure}[!ht]
    \centering
    \includegraphics[scale=0.75]{figures/nvml-arch.jpg}
    \caption{System hierarchy indicating the position of NVML in red \cite{intel2014nvml}}
    \label{fig:nvml}
\end{figure}

Another discussed approach to manage \ac{NVRAM} is through designated file
systems \cite{oukid2017data, andrei2017sap}. File systems provide a well-known
and suitable abstraction for non-volatile storage. In order to enable regular
memory access in a load-store manner, individual files can be mapped into
virtual memory. However, traditional file systems are not directly well-suited
for use with \ac{NVRAM}. One reason is that most operating systems provide a
page cache which is used by file systems to defer expensive disk \ac{IO}. In the
case of \ac{NVRAM}, page caches may be no longer needed, as updates to
\ac{NVRAM} incur far less latency compared to other non-volatile memories. In
this regard, page caches even add overhead instead of mitigating it. Apart from
that, they add a level of indirection which makes writes to \ac{NVRAM} more
likely to be torn by failures. Also, traditional file systems are usually
designed for block-oriented devices which may no longer be the best option.
Therefore, several \ac{NVRAM}-aware file systems have been proposed
\cite{condit2009better, wu2011scmfs, dulloor2014system, xu2016nova}. The key
feature of these file systems is a zero-copy mechanism by circumventing page
caches. This enables true store-load semantics for memory-mapped files. Other
aspects include attempts to leverage the byte-addressable nature of \ac{NVRAM}
and crash-related consistency issues.

Unfortunately, as of this writing there is no evident consensus regarding the
programming model to use for \ac{NVRAM} \cite{boehm2016persistence}. Still,
middlewares such as NVML appear to be gaining the upper hand
\cite{oukid2017data, volos2017whisper, malinowski2017using, andrei2017sap}.

\paragraph{Consistency}

A notorious problem with \ac{NVRAM} is consistency in case of crashes
\cite{condit2009better, dulloor2014system, oukid2017data}. Due to the complex
nature of this subject, further discussion is deferred to the next section.
