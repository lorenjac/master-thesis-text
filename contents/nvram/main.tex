\chapter{Non-Volatile RAM}
\label{ch:nvram}

Conventional non-volatile memories such as \acp{HDD} or \acp{SSD} provide large
capacities but are only block-oriented and incur substantial access latencies
compared to \ac{RAM}. Therefore, \ac{IO}-bound applications tend to keep as much
data in \ac{RAM} as possible but are eventually forced to access slower media
for durable storage. This led to the development of non-volatile variants of
\ac{RAM}. Recent research suggests that both fast and high-capacity \ac{NVRAM}
will become widely available in the near future.

This chapter gives an overview on the state-of-the-art in \ac{NVRAM} research.
Included is a discussion of opportunities and challenges, such as the notorious
consistency issues in the presence of failures.

\section{Architectures and Applications}
\label{ch:nvram-architectures}
There are numerous examples for applications of NVRAM. While earlier works often
considered NVRAM as a means to improve fault tolerance, recent research suggests
a broader range of applications. This development is especially driven by
recent advances in the manufacture of standalone NVRAM.

As pointed out earlier, a well-known use case of NVRAM is to increase fault
tolerance towards crashes. The goal is to retain main memory content even in
case of a crash, for instance by an abrupt power loss \cite{molina1992main,
eich1986main}. This way, critical data such as logs of file systems or databases
remain durable and can be used to recover and even complete unfinished
operations such as making a transaction durable \cite{liskov1991replication,
chen1996rio}. In the past, such solutions relied on battery-backed RAM. This was
subject to criticism as batteries only have a limited charge to ensure
durability. Also batteries degrade over time and need to be maintained to
prevent unexpected failure \cite{molina1992main}. Therefore, modern NVRAM
solutions which no longer require peripherals are a welcome improvement in this
area.

A significant amount of research on NVRAM is dedicated to mitigating the IO
bottleneck imposed by traditional disk storage. One way to do so is to defer
disk IO via durable disk caches \cite{chen1996rio, wu1994envy}. When an object
on disk is requested it is moved to the disk cache. Once an object is cached, it
may be read or modified without accessing the underlying disk. Write-back is
only required when there is not enough space for an incoming cache item. In
doing so, the number of data accesses involving disk IO can be greatly reduced.
Many operating systems including Linux or FreeBSD mimic this concept through the
use of page caches in volatile memory. The difference is that volatile caches
need to be flushed at some point which requires careful resource management.
NVRAM caches on the other hand, only need to evict items when there is no slot
for an incoming item.

Another approach is to treat NVRAM as an equivalent to traditional disk storage.
Early works, which were strongly influenced by the lack of high-capacity NVRAM,
proposed hybrid storage systems where disk storage was used in conjunction with
NVRAM \cite{wang2002conquest, miller2001hermes}. These works have very similar
assignment policies in that they only store small files such as metadata or
libraries in NVRAM whereas larger files remain on disk. While this does not
remove disk access as a common bottleneck, it certainly alleviates latency for
some frequently accessed files. In that regard, NVRAM-complemented disk storage
systems are similar to those with NVRAM disk caches.

For the time being, many applications will have to deal with a scarcity of
NVRAM. But as improving technologies achieve higher capacities with better
parameters, new system architectures become feasible. In some cases, traditional
storage may be eliminated altogether, making NVRAM the primary storage medium. A
prominent use case for this architecture are MMDB. These databases reside
entirely in main memory which drastically reduces latencies when accessing its
data. However, they are also vulnerable to crashes, as main memory is still
mostly volatile. In order to prevent data loss, MMDB have to mirror the entire
database to non-volatile memory. For that, they perform logging or checkpointing
to synchronize individual or groups of changes to non-volatile memory. When the
database is restarted, for example in response to a crash, checkpoints and
logging information are used to recover the most-recent state of the database
that was durable at the time of the crash. That is, MMDB require non-volatile
memory for the sole purpose of recovery. Recurring recovery measures such as
logging have been a long-standing issue with MMDB as they incur expensive disk
IO, thus limiting transaction throughput \cite{eich1986main, molina1992main,
wust2012efficient, malviya2014rethinking}. In addition, restarts reduce the
availability of a system as recovering large databases from slow storage can be
time-consuming. With NVRAM on the other hand, any MMDB would be implicitly
durable, hence making disk IO obsolete. Moreover, it has been shown that with
NVRAM logging is no longer necessary which paves the way for instantaneous
restarts \cite{oukid2015instant}. The concept of NVRAM-based MMDB is especially
promising as some upcoming variants of NVRAM are projected to feature larger
capacities than conventional DRAM \cite{lee2009architecting,
zilberberg2013phase, dulloor2014system}. For that reason, recent research has
investigated NVRAM-aware designs for MMDB ranging from KVS
\cite{bailey2013exploring, zhou2016nvht, wu2016nvmcached} to full-fledged
database systems \cite{oukid2015instant, schwalb2016hyrise, andrei2017sap}.
Research results in these areas are central to this work and are reflected in
chapter \ref{ch:nvram} through \ref{ch:concept}. Given their importance for this
thesis, existing KVS for NVRAM are looked into in more detail at the end of
chapter \ref{ch:kvs}.

While most works aim to improve existing architectures, some explore different
computation models. A recent example is a proposal to use NVRAM to enable
on-chip machine learning. The idea is to move away from the well-known Von
Neumann architecture and implement artificial neural networks by means of NVRAM
\cite{fumarola2016accelerating}. ANN perform a weighted sum over their inputs
before an activation function classifies the result. But before satisfiable
results can be obtained, ANN need to be trained by properly adjusting the scalar
weights of their inputs. It has been suggested that with NVRAM weight adjustment
could be performed on-chip where updated weights would be durable without the
need for write-back. However, neither artificial intelligence nor alternative
computation models are subject of this work.

Most proposals concerning NVRAM assume the presence of volatile RAM.
\cite{oukid2017data}. The reason behind this assumption is that not all parts of
memory may be intended for durability. Nonetheless, recent research suggests
that systems exclusively based on NVRAM can be built \cite{narayanan2012whole,
courtland2016can}. Clearly, such an architecture would have severe consequences
for both operating systems and applications \cite{bailey2011operating}. For
example, operating system processes would remain in memory even if terminated.
On the one hand, it could significantly accelerate the procedure of invoking a
process. On the other hand, all data belonging to a process' address space would
be durable even if they were corrupted by a crash. Other issues are concerned
about memory management, device drivers, and vital information. An early
prototype of such a system is currently in development \cite{courtland2016can}.
This topic however, is beyond the scope of this work and it is henceforth
assumed that volatile RAM will co-exist with NVRAM.

As shown above, NVRAM provides an opportunity to improve existing architectures
and even create new computation models. Given sufficient parameters such as
capacity and endurance, NVRAM could resolve the IO bottleneck of non-volatile
storage media. Especially systems such as MMDB have shown considerable gains in
transaction throughput and recovery times. Consequently, MMDB and in particular
KVS are at the center of this work.


\section{Technologies}
\label{ch:nvram-technologies}
The design and integration of fast \ac{NVRAM} is not a new research area. In the
past, there have been multiple attempts to produce non-volatile equivalents of
main memory. While earlier approaches were mainly designed to make systems more
tolerant to crashes \cite{molina1992main}, recent research suggests \ac{NVRAM}
to hold entire \acp{MMDB} and speed up recovery \cite{oukid2015instant,
schwalb2016hyrise, andrei2017sap}.

One way to achieve byte-addressable \ac{NVM} is to attach \ac{DRAM} or \ac{SRAM}
to backup power supplies as in \cite{liskov1991replication, wang2002conquest}.
In other cases, conventional non-volatile storage, such as flash memory, is
directly attached to the \ac{DRAM} module \cite{shi2010write, huang2014design,
oe2016feasibility}. However, these approaches rely on batteries, which must be
maintained, or block-oriented memory which is still much slower than DRAM. A
more promising approach is to develop alternative memory techniques that provide
the features of \ac{NVRAM}, natively.

Among a range of recent \ac{NVRAM} designs, the most promising are \ac{PCM} and
\ac{STT-RAM}\cite{zilberberg2013phase, mittal2016survey, jain2017computing}.
These technologies vary according to the following parameters:

\begin{itemize}
    \item density
    \item endurance
    \item latency
    \item power consumption
\end{itemize}

While \ac{PCM} is projected to have the highest density of all, including
\ac{DRAM}, it is also features a lower endurance. In terms of endurance,
\ac{STT-RAM} fares better than \ac{PCM} but is still surpassed by \ac{DRAM}.
Also, \ac{STT-RAM} has a very low density which prevents high capacity memory
modules. \ac{NVRAM} is known to incur higher access latency that \ac{DRAM}. This
is certainly true for \ac{PCM} which can have up to 10\% latency than \ac{DRAM}.
\ac{STT-RAM}, on the other hand, is projected to the on a par with \ac{DRAM}.
Concerning power consumption, \ac{STT-RAM} appears to have advantages over both
\ac{DRAM} and \ac{PCM}. In this case, \ac{PCM} registers an even higher
consumption than \ac{DRAM} \cite{mittal2016survey}.

In total, \ac{STT-RAM} poses a very promising technology but provides suboptimal
capacities and is also less enduring than \ac{DRAM}. \ac{PCM}, on the other
hand, provides a higher capacity but at the cost of higher latencies. Still,
recent research suggests \ac{PCM} to be the first fast high-capacity \ac{NVRAM}
to be ready to manufacture \cite{zilberberg2013phase, dulloor2014system,
mittal2016survey}. Nevertheless, this work is independent of the underlying
NVRAM technology.

%\subsection{Promising NVRAM Technologies}

% mittal2016survey

%\paragraph{PCM}

% lee2009architecting
% zilberberg2013phase

%\paragraph{STT-MRAM}


\section{Challenges}
\label{ch:nvram-challenges}
Despite the conceivable advantages of \ac{NVRAM}, there are also challenges to
be addressed. Although most issues are of practical nature there also are
conceptual concerns.

\paragraph{Unintended Durability}

The key feature of \ac{NVRAM} is to retain its data across restarts. However,
not all data are necessarily intended to be durable. Notable examples include
transient, confidential, and corrupt data. The former comprises data which may
not be valid after a system restart, as is the case with data related to machine
or device state.

Other data such as passwords, encryption keys, or decrypted data may be
confidential and should not be durable. It has been shown that even volatile
\ac{RAM} holds its charge long enough so that a module can be moved to an
attacker's machine for read-out \cite{halderman2008lest}. Despite being
countered with hardware scramblers, researchers still managed to apply
variations of the technique and obtain vital information
\cite{yitbarek2017cold}. Such attacks would be trivial on \ac{NVRAM}
\cite{bailey2011operating}, but that is beyond the scope of this work.

% By cooling the module, capacitor
% discharge can be slowed so that it can be moved to another machine where its
% content may be read and parsed. Despite being countered with hardware
% scramblers, researchers still managed to apply variations of the technique and
% obtain vital information \cite{yitbarek2017cold}. Such attacks would be trivial
% on \ac{NVRAM}, as durability is its primary feature \cite{bailey2011operating}. Of
% course, confidential data could be overwritten with zeros after usage, but there
% is always a possibility that a crash might prevent such clean up tasks from
% completing. That said, sensitive data should at all times remain in volatile
% memory and be nulled after use. Although information security is an important
% matter, this thesis does not address such issues. Although it is important,
% information security in terms of attack resilience contributes little to the aim
% of this thesis and is therefore not addressed further. Also, since volatile \ac{RAM}
% continues to be available, there is no need to use \ac{NVRAM} for sensitive data.

When an operating system or application behaves in an erratic fashion or
crashes, it may produce corrupt data in memory. This is called a stray write. In
this case, systems incorporating \ac{NVRAM} could face durable memory
corruptions \cite{condit2009better, venkataraman2011consistent}. In contrast to
conventional non-volatile memory, \ac{NVRAM} is particularly vulnerable as it is
directly accessible through the \ac{CPU}. However, it has been shown that,
compared to disk storage, stray writes do not occur significantly more often in
\ac{NVRAM} \cite{chen1996rio}. Therefore, stray writes are not an issue in this
work.

% When an operating system or application behaves in an erratic fashion or crashes, it
% may produce corrupt data in memory. Unless memory is cleared or rewritten,
% systems incorporating \ac{NVRAM} could face durable memory corruptions. \ac{NVRAM} on the
% other hand is expected to be connected to the memory bus, enabling unbuffered
% access through virtual memory addresses. This makes \ac{NVRAM} vulnerable to stray
% writes \cite{condit2009better, venkataraman2011consistent}. However, it has been
% shown that, compared to disk storage, stray writes do not occur significantly
% more often in \ac{NVRAM} \cite{chen1996rio}. That said, stray writes are not
% considered an issue in this work.

\paragraph{Memory Management}

\ac{NVRAM} is a new type of memory that can also be used as durable mass
storage. In order to benefit from this new technology, both platforms and
operating systems need to find ways to efficiently manage it. There are several
issues to be addressed in this area.

% Memory Interface

An important aspect in managing \ac{NVRAM} is the memory interface. Recent
research suggests that \ac{NVRAM} will be attached to the system memory bus
using the \ac{DIMM} format known from \ac{DRAM} \cite{volos2017whisper,
oukid2017data, andrei2017sap, intel2017nvdimm}. A decisive advantage of this
approach is much lower latencies compared to the alternative \ac{IO} bus.
Another reason is that in a previous effort to produce \ac{NVRAM}, known as
\ac{NVDIMM}, modules have also been integrated this way \cite{dulloor2014system,
huang2014design}. Consequently, system designers can build on an existing
software stack. Still, there are drawbacks to be considered. Clearly, the number
of available \ac{DIMM} slots in a machine is limited, so \ac{NVRAM} may not
scale well for mass storage. That situation is especially relevant in hybrid
systems containing both \ac{RAM} and \ac{NVRAM}. Also, in hybrid systems both
kinds of memory are likely to be attached to the same memory interface thus
sharing its bandwidth.

With \ac{NVRAM} devices integrated into the system, programmers still need a way
to access it. Several approaches have been proposed to this end
\cite{volos2017whisper}. While it is always possible to operate on \ac{NVRAM} by
mapping individual device regions into virtual memory, there are considerable
weaknesses to this approach \cite{condit2009better, volos2011mnemosyne,
dulloor2014system, volos2017whisper}. A major challenge of working with
\ac{NVRAM} is to provide consistency guarantees across possible system failures.
Yet, systems are largely unaware of these circumstances. With raw device access
which is already error-prone, the complex task of preserving consistency is
handed to the programmer. Another challenge is that virtual memory mappings are
volatile and may no longer be valid after a restart.

Therefore, it has been proposed to rely on dedicated high-level programming
primitives as in Mnemosyne, NV-Heaps, and NVML \cite{volos2011mnemosyne,
coburn2011nv_heaps, intel2017nvml}. These systems provide interfaces for memory
allocation and consistent updates based on transactions. An important
distinction to the previous low-level approach is that memory is accessed
through an \ac{NVRAM}-aware \ac{API} instead of basic load and store statements.
The difference is that the latter have no knowledge of non-volatile memory and
its implications.

\begin{figure}[!ht]
    \centering
    \includegraphics[scale=0.75]{figures/nvml-arch.jpg}
    \caption{System hierarchy indicating the position of NVML in red \cite{intel2014nvml}}
    \label{fig:nvml}
\end{figure}

Another discussed approach to manage \ac{NVRAM} is through designated file
systems \cite{oukid2017data, andrei2017sap}. File systems provide a well-known
and suitable abstraction for non-volatile storage. In order to enable regular
memory access in a load-store manner, individual files can be mapped into
virtual memory. However, traditional file systems are not directly well-suited
for use with \ac{NVRAM}. One reason is that most operating systems provide a
page cache which is used by file systems to defer expensive disk \ac{IO}. In the
case of \ac{NVRAM}, page caches may be no longer needed, as updates to
\ac{NVRAM} incur far less latency compared to other non-volatile memories. In
this regard, page caches even add overhead instead of mitigating it. Apart from
that, they add a level of indirection which makes writes to \ac{NVRAM} more
likely to be torn by failures. Also, traditional file systems are usually
designed for block-oriented devices which may no longer be the best option.
Therefore, several \ac{NVRAM}-aware file systems have been proposed
\cite{condit2009better, wu2011scmfs, dulloor2014system, xu2016nova}. The key
feature of these file systems is a zero-copy mechanism by circumventing page
caches. This enables true store-load semantics for memory-mapped files. Other
aspects include attempts to leverage the byte-addressable nature of \ac{NVRAM}
and crash-related consistency issues.

Unfortunately, as of this writing there is no evident consensus regarding the
programming model to use for \ac{NVRAM} \cite{boehm2016persistence}. Still,
middlewares such as NVML appear to be gaining the upper hand
\cite{oukid2017data, volos2017whisper, malinowski2017using, andrei2017sap}.

\paragraph{Consistency}

A notorious problem with \ac{NVRAM} is consistency in case of crashes
\cite{condit2009better, dulloor2014system, oukid2017data}. Due to the complex
nature of this subject, further discussion is deferred to the next section.


% \section{Preserving Consistency}
% \label{ch:nvram-consistency}
% As pointed out earlier, the consistency of data stored in \ac{NVRAM} is
vulnerable to crashes or power failures \cite{condit2009better,
dulloor2014system, oukid2017data}. Since \ac{NVRAM} is directly attached to the
processor memory interface, there is no need to use techniques such as \ac{DMA}
to transfer a modified page to external storage. This also means that a memory
operation solely relies on the \ac{CPU} which usually gives no confirmation when
that operation completes. In this context, there are two major issues that
threaten the consistency of data written to \ac{NVRAM}, namely out-of-order
execution and deferred write-back.

\subsubsection{Out-Of-Order Execution}

When executing a program, processors fetch instructions in a consecutive manner.
Some instructions may inflict minimal latency, while others such as load
operations may delay further execution for hundreds of cycles. In an attempt to
optimize instruction throughput, individual instructions may be reordered. While
compilers may statically define promising orders, processors are able to reorder
instructions at runtime. This enables processors to optimize resource
utilization and hide latencies of time-consuming instructions. However, only
reorderings that do not violate data dependencies between instructions are
possible. While processors do prevent such conflicts, there are dependencies
that cannot be observed. For example, in order to mark a chunk of data as
durable in \ac{NVRAM}, one might store a designated flag immediately after the
operation completed. With out-of-order execution it is possible that the flag is
written before the payload. This can lead to severe inconsistencies especially
when a crash prevents the chunk from being written.

A common method to counter this issue is to enforce memory order with memory
barriers (also fences) \cite{dulloor2014system, schwalb2016hyrise,
oukid2017data}. A memory barrier prevents the \ac{CPU} from proceeding until all
prior memory operations have completed. Although a barrier does not directly
order its preceding instructions, it can be used to impose an order on separate
sequences of instructions. An example for a memory barrier is \code{sfence} on
x86 architectures. While this approach solves the initial problem, it has a
notable drawback. Memory barriers defeat the purpose of out-of-order execution.
As a result, \ac{CPU} pipelines are likely to stall, hence reducing resource
utilization. Furthermore, store buffers are flushed leading to higher latencies
when accessing data of deferred store operations. Therefore, barriers can have a
significant impact on runtime performance, unless used judiciously. With
\emph{epoch barriers} a similar approach has been proposed to address both order
and durability issues \cite{condit2009better}.

\subsubsection{Deferred Write-Back}

In many modern processor architectures store operations may not immediately
lead to an update in main memory. This behavior can be caused by intermediary
buffers such as memory order buffers, caches, and memory controller buffers.
While their individual purpose may vary, they all defer memory write-back
operations. This is a known vulnerability for consistency in \ac{NVRAM} as the
mentioned buffers are volatile and deferred stores may be lost when power fails
\cite{condit2009better, oukid2017data}. In order to preserve consistency, it is
necessary to force write-back in all of these cases. Figure
\ref{fig:memory-interface} shows a typical memory hierarchy with several layers
by which stores can be deferred.

\begin{figure}[h!]
    \centering
    \includegraphics[scale=0.9]{figures/nvram/memory-subsystem.pdf}
    \caption{Architecture of memory subsystem \cite{bhandari2012implications}.}
    \label{fig:memory-interface}
\end{figure}

\paragraph{\acp{MOB}}

In conjunction with instruction scheduling and cache coherency protocols a
memory order buffer may be present. It holds all loads and stores, with the
exception of non-temporal operations. In order to prevent a deferred write-back
through \acp{MOB}, their store buffers must be flushed. On x86
architectures this can be achieved with a store fence operation such as
\code{sfence}. As mentioned above, memory barriers carry a considerable
overhead. However, if memory barriers are already used for enforcing program
order, then flushing store buffers is a desirable side effect and incurs no
overhead.

\paragraph{Caches}

Processor caches help avoid access latencies and reduce memory bus traffic for
frequently used data. A possible exception are non-temporal stores and data
chunks marked as uncacheable. Similar to \acp{MOB}, caches are
volatile so an abrupt power failure may lead to lost updates. The issue with
this is not that updates are lost but that it is unclear which updates are lost,
if any. The reason for this circumstance is the cache eviction policy trying to
compensate for typically narrow cache volume. Depending on policy, cache
content, and system load, a modified chunk may or may not be flushed to main
memory. An application scenario where such behavior is unacceptable is
transactions. In the example in Figure \ref{fig:nvram-cache-crash}, two stores are cached. One store becomes durable because its cache line is evicted, while the other remains in cache and is lost in a crash.

% \todo[inline]{Insert figure showing inconsistency due to crash}

%\begin{lstlisting}
%T1: r(A) w(A) c*   -- Cache ------------------------ (crash)
%T2: -------------- r(A) w(B) c -- Cache -- NVRAM --- (crash)
%\end{lstlisting}

\begin{figure}[!ht]
    \centering
    \includegraphics[width=0.70\textwidth]{figures/nvram/cache-crash.pdf}
    \caption{A sitation where only one of two cached stores reaches durability.}
    \label{fig:nvram-cache-crash}
\end{figure}

An approach to prevent such inconsistencies is to disable caching for selected
memory regions but that could introduce considerable overhead for frequently
used data. A more popular approach is to evict cache lines programmatically
whenever necessary \cite{condit2009better, dulloor2014system, oukid2017data}. On
x86 architectures this can be done with the \code{clflush} or \code{clflushopt}
instructions. However, the problem with a cache line flush is that wanting to
make a cache line durable does not always mean that is should be evicted.
Therefore, there are proposals for instructions, such as \code{clwb}, that send
a cache line to a memory controller without evicting it \cite{kolli2016high,
oukid2017data}.

% Unfortunately, at the time of this writing, there is no evidence
% of a processor to implement this instruction.

\paragraph{\acp{WPQ}}

Once a cache line is flushed, it is propagated to a memory controller where it
is buffered in a \ac{WPQ}. Again, the problem is that such a buffer is usually
volatile. This means that a power failure could lead to lost updates to
\ac{NVRAM}. Even though residual power in \ac{DRAM} has been shown to be
substantial, there is no reliable way to ensure a full \ac{WPQ} flush
\cite{halderman2008lest}. This circumstance has given rise to many discussions
in the past \cite{condit2009better, dulloor2014system, kolli2016high}. Some
authors proposed a designated instruction for flushing \acp{WPQ}. An example is
the meanwhile deprecated \code{pcommit} instruction (formerly known as
\code{pm\_wbarrier}) \cite{dulloor2014system, oukid2015instant,
volos2017whisper}. Others have developed more general mechanisms for preserving
consistency in \ac{NVRAM} that also address this issue \cite{condit2009better,
pelley2014memory}. The dotted rectangle in Figure \ref{fig:wpq} indicates the
domain in the memory subsystem that must be protected from power failures.

\begin{figure}[h!]
    \centering
    \includegraphics{figures/nvram/adr-example.pdf}
    \captionsetup{justification=centering}
    \caption{The memory domain to be protected against power failures \cite{rudoff2017persistent}.}
    \label{fig:wpq}
\end{figure}

The current state of affairs is that platforms must provide a feature called
\ac{ADR} \cite{volos2017whisper, rudoff2017persistent}. It works by exploiting
the fact that, even in case of a power failure, there is sufficient time and
power to flush \acp{WPQ} in all memory controllers. When the system's power
supply unit detects a power failure, it signals all memory controllers to flush
their \acp{WPQ}. As a result, the programmer does not need to worry about
\acp{WPQ} and no overhead is incurred.


\section{Summary}
\label{ch:nvram-summary}

\ac{NVRAM} is a promising technology, especially in storage-bound environments.
Until recently, both latencies and capacities had fallen short of expectations.
However, recent advances in the manufacture of \ac{NVRAM} have shown promising
results. Technologies such as \ac{PCM} provide larger capacities than \ac{DRAM}
at comparable latencies.

Apart from other more intricate use cases, recent research shows that \ac{MMDB}
can benefit greatly from \ac{NVRAM}. While access latencies are comparable to
conventional \ac{MMDB}, \ac{NVRAM} largely eliminates the need to ensure
recoverability on slower mass storage. This enables higher transaction
throughput and near-instantaneous restarts.

Still, there are also challenges to be addressed. First, current \ac{NVRAM}
technologies tend to have either high latency, low capacity, or low endurance.
Even though \ac{PCM} is a promising candidate, it is still slower than
\ac{DRAM}, especially when writing. Hence, manufacture still needs significant
improvement. Furthermore, there is, as of this writing, no generally accepted
programming model for \ac{NVRAM}. In general, there is no way for software to
determine whether a store has arrived in \ac{NVRAM} or not. Since \ac{CPU}s are
unaware of any transactional semantics between stores, programmers need to
manually ensure consistency for data in \ac{NVRAM}. Otherwise, a torn write from
a crash may lead to irreversible data corruption. The counter measures presented
in this chapter must be applied judiciously as they may incur significant
runtime penalties.
