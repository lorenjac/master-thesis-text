\chapter{Non-Volatile RAM}
\label{ch:nvram}

Current non-volatile storage technologies such as harddisks or \acp{SSD} are
block-oriented. Also accessing these devices can be up to three orders of
magnitude slower than accessing \ac{DRAM}. Recent research however, suggests
that fast and byte-addressable \ac{NVM} will be available in the near future.
Based on its projected characteristics, this new class of memory is referred to
as \ac{NVRAM}.

\todo{Short chapter overview}
\todo{re-consider choice of PCM over STT-RAM}

\paragraph{Terminology}

As of this writing there is no consensus as to how byte-addressable non-volatile
memory should be referred to. This thesis exclusively uses term \ac{NVRAM}.
There are two reasons for this decision. For one, alternative terms such as
\ac{BPRAM} or \ac{PM} suggest that non-volatile memory is also persistent which
however, may not always be correct\todo{proof?}. Another reason is that some
terms such as \ac{SCM}, \ac{NVM} and \ac{PM} may not reflect the property of
bytewise addressing which is central to these technologies. \ac{NVRAM} on the
other hand, explicitly denotes all of the required properties.

% When referring to hardware or processor architecture, this work implicitly
% refers to the Intel Core architecture and the most recent instruction set
% Intel 64 (also x86-64, x64 or AMD64).

\section{State of the Art}

% general properties (non-volatile, byte-addressable)

The main properties of NVRAM are non-volatility and access on a byte granule.

% technologies (PCM, STT-RAM, RRAM)
% latency
% density
% endurance

% design factors (latency vs capacity vs power)

% integration
% consensus that NVRAM will be attached as NVDIMM to system memory interface
% how will it be accessed
%   direct mapped
%       mapping would somehow have to be deterministic
%       or provide designated function for retrieving non-volatile address space
%   device driver
%       file system (map to virtual memory using mmap)
%           ordinary
%           with DAX (direct memory access) [Intel, oukid2017]
%               circumvents page cache for memory-like storage devices

\paragraph{Integration}

Before \ac{NVRAM}, there have been multiple attempts to creating fast
non-volatile system memory. Notable examples include battery-backed \ac{DRAM}
\cite{molina1992main, condit2009better, bailey2013exploring} and NAND-based
\acp{SSD} which were attached to the system memory interface \cite{wu1994envy,
shi2010write}. There have also been attempts to combine the low latencies of
\ac{DRAM} and the large capacities of NAND flash \cite{oe2016feasibility}. The
latter has led to the \ac{NVDIMM} specification by \ac{JEDEC}. It introduces
\texttt{NVDIMM-F} and \texttt{NVDIMM-N} which define how to integrate and
interact with combinations of \ac{DRAM} and NAND. In an effort to support recent
\ac{NVM} technologies such as \ac{PCM}, \ac{JEDEC} has announced
\texttt{NVDIMM-P}\cite{jedec2017nvdimm}. Therefore, it is assumed that
\ac{NVRAM} will be available as traditional DIMM modules which are connected to
the memory interface.

Operating systems need to find ways to efficiently manage \ac{NVRAM}. Similarly
to previous variants of \ac{NVDIMM}, it is proposed that \ac{NVRAM} is accessed
through a dedicated device driver \cite{intel2017nvdimm}. Another approach would
be to map \ac{NVRAM} directly into virtual memory \todo{elaborate!}.

At this stage, there are two ways to access the device: raw device access and
file systems. While raw device access appears unwieldy, file systems provide a
great deal of flexibility. Not only do file systems provide a more suitable
abstraction for user-level applications but they also enable memory-mapped IO
through virtual memory. Therefore, file systems are the currently preferred
method of accessing \ac{NVRAM} \cite{oukid2017data}. Still, there are a couple
of issues to address.

While traditional file systems will continue to work, they may not be
well-suited for \ac{NVRAM} as they were designed for external block-oriented
storage. Therefore, several \ac{NVRAM}-aware file systems such as \texttt{pmfs},
\texttt{bpfs}, and \texttt{scmfs} have been proposed \cite{dulloor2014system,
condit2009better, wu2011scmfs}. \todo{elaborate on differences}

A key feature of main memory is store/load semantics.

% file systems use operating system page cache
% was used to buffer modified pages to reduce disk IO
% no longer required + breaks store/load semantics
% nvram-aware file systems provide means to expose memory regions
% in linux/ext4: using DAX => circumvents page cache,
% nvram-aware file systems also proposed to expose actual memory regions for direct memory access

% use cases
%   operating systems
%       processes stay in memory and can be toggled
%   databases
%       fast disk caches for large databases
%           for data
%           for recovery (e.g. logging)
%       fast recovery for in-memory databases
%           no more disk IO for logging etc.
%           instant restart

% challenges
% some applications (at least in part) rely on information to be volatile (corrupt data from crash, security relevant data, garbage collection)
% memory mapping is volatile and often random (see ASLR)
% consistency in case of crash

\section{Preserving Consistency}

As pointed out earlier, \ac{NVRAM} is vulnerable to inconsistent data in case of
a crash or power failure. Unlike \acp{HDD} or \acp{SSD}, \acl{NVRAM} is part of
the memory address space and accessed through virtual memory rather than IO
ports. Therefore, all data operations on \ac{NVRAM} must pass through the memory
hierarchy of the \ac{CPU}. However, there are two essential mechanisms of modern
processors that can break the consistency of data stored in \ac{NVRAM}:
\emph{instruction reordering} and \emph{deferred write-back}.

% However, the memory hierarchy is designed for volatile memory which has weaker
% semantics than non-volatile counterpart. As modern \acp{CPU} have no way to
% distinguish between both types of memory, \ac{NVRAM} is treated as if it were
% \ac{DRAM}.

\paragraph{Instruction Reordering}

\todo[inline]{explain principle}
\todo[inline]{explain example for x86}

In order to optimimize instruction throughput of modern processors, instructions may be reordered. This is usually done in a bilateral approach. While compilers may optimize intructions orders statically, modern processors are able to reorder instructions at runtime. However, reordering is only possible if there are no data dependencies between

\todo[inline]{reasoning?}
\todo[inline]{requirements?}
\todo[inline]{example for failure?}

This leads to the conclusion that, preserving data consistency on \ac{NVRAM} requires the system or programmer to enforce strict ordering of memory operations.

\paragraph{Deferred Write-Back}

\todo[inline]{explain principle}
\todo[inline]{explain example for x86}

Second, many processor architectures do not immediately execute memory
operations but store them in dedicated store and load buffers, respectively. A
store operation is only executed when its store buffer is full or a fence
operation was issued. During this period the store is completely volatile.

Third, unless configured otherwise, processor caches keep cache lines until they
are either evicted to lower cache levels or written back to memory.

Fourth, once the affected cache line of a store is flushed, the store is
propagated to the memory controller where it is once more buffered and written
onto the device, eventually. During this stage the in-flight data is still
volatile and thus vulnerable to power loss.

This behaviour has two consequences: memory operations may be executed in a
wrong order and in-flight stores may not never the device in case of a crash
or power loss.

Modern processors queue memory operations in small buffers in order to achieve
efficient bulk transfers. On Intel \code{x86} architectures there are separate
buffers for load and store operations, respectively. Such buffers are referred
to as \ac{MOB}. Whenever a store operation is issued, it is added to an
available store buffer. Once satisfied, the buffer is flushed and the stores are
propagated into the first level data cache.

\subsection{Approaches with Existing Hardware}

With existing hardware
% fences

% caches

Instead of waiting for a cache line to be evicted, it can be flushed using
designated instructions. The \code{x86} instruction set provides multiple
instructions to do so: \code{CLFLUSH}, \code{CLFLUSHOPT}, and \code{CLWB}.
While \code{CLFLUSH}

% write-pending queues

While other levels of the memory hierarchy may be flushed through machine
instructions, there is no such instruction for memory controllers. Therefore,
\code{PCOMMIT} a special instruction to flush memory controller buffers was
proposed. Only recently, this instruction has been deprecated in favor of
\ac{ADR}. \ac{ADR} is a platform feature that guarantees that, in case of power
loss, all memory controllers flush their buffers using the remaining power of
the system. It works by exploiting the fact that even in case of a power failure
there is sufficient time and power to flush the write buffers of all memory
controllers. For this to work, the power supply unit has to detect a power
failure and issue a signal to the memory controllers.

\subsection{Alternative Approaches}

\todo[inline]{Epoch Barriers, condit, SOSP 2009}
\todo[inline]{Whole System Persistence, narayanan, 2012}
\todo[inline]{Strand Persistency, pelley, ISCA 2014}

\section{Phase-Change Memory}

\todo{explain why PCM instead of STT-RAM / RRAM}

density: DRAM < STT-RAM < PCM (higher is better)
endurance: PCM < STT-RAM < DRAM (higher is better)
latency: DRAM <= STT-RAM < PCM (lower is better)
dypower: STT-RAM < DRAM < PCM (lower is better)

implication: PCM
