As pointed out earlier, the consistency of data stored in \ac{NVRAM} is
vulnerable to crashes or power failures \cite{condit2009better,
dulloor2014system, oukid2017data}. Since \ac{NVRAM} is directly attached to the
processor memory interface, there is no need to use techniques such as \ac{DMA}
to transfer a modified page to external storage. This also means that a memory
operation solely relies on the \ac{CPU} which usually gives no confirmation when
that operation completes. In this context, there are two major issues that
threaten the consistency of data written to \ac{NVRAM}, namely out-of-order
execution and deferred write-back.

\subsubsection{Out-Of-Order Execution}

When executing a program, processors fetch instructions in a consecutive manner.
Some instructions may inflict minimal latency, while others such as load
operations may delay further execution for hundreds of cycles. In an attempt to
optimize instruction throughput, individual instructions may be reordered. While
compilers may statically define promising orders, processors are able to reorder
instructions at runtime. This enables processors to optimize resource
utilization and hide latencies of time-consuming instructions. However, only
reorderings that do not violate data dependencies between instructions are
possible. While processors do prevent such conflicts, there are dependencies
that cannot be observed. For example, in order to mark a chunk of data as
durable in \ac{NVRAM}, one might store a designated flag immediately after the
operation completed. With out-of-order execution it is possible that the flag is
written before the payload. This can lead to severe inconsistencies especially
when a crash prevents the chunk from being written.

A common method to counter this issue is to enforce memory order with memory
barriers (also fences) \cite{dulloor2014system, schwalb2016hyrise,
oukid2017data}. A memory barrier prevents the \ac{CPU} from proceeding until all
prior memory operations have completed. Although a barrier does not directly
order its preceding instructions, it can be used to impose an order on separate
sequences of instructions. An example for a memory barrier is \code{sfence} on
x86 architectures. While this approach solves the initial problem, it has a
notable drawback. Memory barriers defeat the purpose of out-of-order execution.
As a result, \ac{CPU} pipelines are likely to stall, hence reducing resource
utilization. Furthermore, store buffers are flushed leading to higher latencies
when accessing data of deferred store operations. Therefore, barriers can have a
significant impact on runtime performance, unless used judiciously. With
\emph{epoch barriers} a similar approach has been proposed to address both order
and durability issues \cite{condit2009better}.

\subsubsection{Deferred Write-Back}

In many modern processor architectures store operations may not immediately
lead to an update in main memory. This behavior can be caused by intermediary
buffers such as memory order buffers, caches, and memory controller buffers.
While their individual purpose may vary, they all defer memory write-back
operations. This is a known vulnerability for consistency in \ac{NVRAM} as the
mentioned buffers are volatile and deferred stores may be lost when power fails
\cite{condit2009better, oukid2017data}. In order to preserve consistency, it is
necessary to force write-back in all of these cases. Figure
\ref{fig:memory-interface} shows a typical memory hierarchy with several layers
by which stores can be deferred.

\begin{figure}[h!]
    \centering
    \includegraphics[scale=0.9]{figures/nvram/memory-subsystem.pdf}
    \caption{Architecture of memory subsystem \cite{bhandari2012implications}.}
    \label{fig:memory-interface}
\end{figure}

\paragraph{\acp{MOB}}

In conjunction with instruction scheduling and cache coherency protocols a
memory order buffer may be present. It holds all loads and stores, with the
exception of non-temporal operations. In order to prevent a deferred write-back
through \acp{MOB}, their store buffers must be flushed. On x86
architectures this can be achieved with a store fence operation such as
\code{sfence}. As mentioned above, memory barriers carry a considerable
overhead. However, if memory barriers are already used for enforcing program
order, then flushing store buffers is a desirable side effect and incurs no
overhead.

\paragraph{Caches}

Processor caches help avoid access latencies and reduce memory bus traffic for
frequently used data. A possible exception are non-temporal stores and data
chunks marked as uncacheable. Similar to \acp{MOB}, caches are
volatile so an abrupt power failure may lead to lost updates. The issue with
this is not that updates are lost but that it is unclear which updates are lost,
if any. The reason for this circumstance is the cache eviction policy trying to
compensate for typically narrow cache volume. Depending on policy, cache
content, and system load, a modified chunk may or may not be flushed to main
memory. An application scenario where such behavior is unacceptable is
transactions. In the example in Figure \ref{fig:nvram-cache-crash}, two stores are cached. One store becomes durable because its cache line is evicted, while the other remains in cache and is lost in a crash.

% \todo[inline]{Insert figure showing inconsistency due to crash}

%\begin{lstlisting}
%T1: r(A) w(A) c*   -- Cache ------------------------ (crash)
%T2: -------------- r(A) w(B) c -- Cache -- NVRAM --- (crash)
%\end{lstlisting}

\begin{figure}[!ht]
    \centering
    \includegraphics[width=0.70\textwidth]{figures/nvram/cache-crash.pdf}
    \caption{A sitation where only one of two cached stores reaches durability.}
    \label{fig:nvram-cache-crash}
\end{figure}

An approach to prevent such inconsistencies is to disable caching for selected
memory regions but that could introduce considerable overhead for frequently
used data. A more popular approach is to evict cache lines programmatically
whenever necessary \cite{condit2009better, dulloor2014system, oukid2017data}. On
x86 architectures this can be done with the \code{clflush} or \code{clflushopt}
instructions. However, the problem with a cache line flush is that wanting to
make a cache line durable does not always mean that is should be evicted.
Therefore, there are proposals for instructions, such as \code{clwb}, that send
a cache line to a memory controller without evicting it \cite{kolli2016high,
oukid2017data}.

% Unfortunately, at the time of this writing, there is no evidence
% of a processor to implement this instruction.

\paragraph{\acp{WPQ}}

Once a cache line is flushed, it is propagated to a memory controller where it
is buffered in a \ac{WPQ}. Again, the problem is that such a buffer is usually
volatile. This means that a power failure could lead to lost updates to
\ac{NVRAM}. Even though residual power in \ac{DRAM} has been shown to be
substantial, there is no reliable way to ensure a full \ac{WPQ} flush
\cite{halderman2008lest}. This circumstance has given rise to many discussions
in the past \cite{condit2009better, dulloor2014system, kolli2016high}. Some
authors proposed a designated instruction for flushing \acp{WPQ}. An example is
the meanwhile deprecated \code{pcommit} instruction (formerly known as
\code{pm\_wbarrier}) \cite{dulloor2014system, oukid2015instant,
volos2017whisper}. Others have developed more general mechanisms for preserving
consistency in \ac{NVRAM} that also address this issue \cite{condit2009better,
pelley2014memory}. The dotted rectangle in Figure \ref{fig:wpq} indicates the
domain in the memory subsystem that must be protected from power failures.

\begin{figure}[h!]
    \centering
    \includegraphics{figures/nvram/adr-example.pdf}
    \captionsetup{justification=centering}
    \caption{The memory domain to be protected against power failures \cite{rudoff2017persistent}.}
    \label{fig:wpq}
\end{figure}

The current state of affairs is that platforms must provide a feature called
\ac{ADR} \cite{volos2017whisper, rudoff2017persistent}. It works by exploiting
the fact that, even in case of a power failure, there is sufficient time and
power to flush \acp{WPQ} in all memory controllers. When the system's power
supply unit detects a power failure, it signals all memory controllers to flush
their \acp{WPQ}. As a result, the programmer does not need to worry about
\acp{WPQ} and no overhead is incurred.
