In the past, there have been multiple attempts to produce non-volatile
equivalents of main memory. One way is to emulate non-volatility by providing
backup power supplies or combining volatile DRAM and conventional non-volatile
storage in a single module. Another promising approach is to develop alternative
memory techniques that provide the features of NVRAM.

% This section presents notable developments in the field of NVRAM.

\todo[inline]{Present Phase-Change Memory (Concept/History/Parameters)}

% \paragraph{Early Approaches}
%
% Initially, the intention was to make systems more tolerant to failures, although
% opportunities for storage systems were also discussed \cite{molina1992main,
% wang2002conquest}. The main issue with DRAM is that, in order to retain its
% data, it needs to refresh its cells which requires a power supply. Therefore,
% early non-volatile main memories were equipped with batteries or UPS which
% allowed to hold information for a limited time in case of a power failure.
% Notable examples for systems using battery-backed DRAM are the file systems
% \emph{Harp} \cite{liskov1991replication} and \emph{Conquest}
% \cite{wang2002conquest} as well as the file cache \emph{Rio} \cite{chen1996rio}.
%
% In another approach, researchers used battery-backed SRAM as a write buffer
% which is flushed to an interconnected flash memory when full or in case of a
% power failure \cite{wu1994envy}. By limiting access to the SRAM, the
% non-volatile yet slower flash device is effectively shadowed. Still, since flash
% operates in block mode, the byte-addressable nature of SRAM cannot be exploited
% in this setup. This is an early example of hybrid memory solutions for fast mass
% storage. However, backup power supplies have also been subject to criticism.
% Arguments are that batteries are not inherently reliable and introduce
% additional maintenance overhead \cite{molina1992main}. Therefore experiments
% were conducted, where flash memory was directly attached to the memory interface
% \cite{shi2010write}. Similar ideas were later consolidated in the JEDEC NVDIMM
% specification, which defines several configurations for DIMMs consisting of
% flash memory and DRAM \cite{oe2016feasibility, huang2014design}.
%
% % This marks a shift in the development of NVRAM as reliability becomes second
% % to fast high-capacity storage.
%
% \paragraph{Modern Approaches}
%
% Early approaches for practical NVRAM were focused on making volatile DRAM retain its information across power outages. There are however promising alternatives such as PCM, MRAM or RRAM which are both byte-addressable and non-volatile by design. Although their underlying principles have been known for a while, further research was required to reach practical designs for manufacturing. Recent research suggests that these NVM technologies will see broad availability in the near future \todo{cite}.

% technologies
    % PCM (also PRAM)
    %     phase-change memory
    %     based on properties of chalcogenide glasses
    %     discovered in 1955
    %     first prototype in 1969
    % STT-MRAM (also ST[T]-[M]RAM)
    %     advanced type of MRAM
    %         magnetoresistive RAM
    %         similar to magnetic core memory (1955)
    %         GMR effect discovered in 1984
    %         developed since 1995 (Motorola)
    %     spin-transfer torque proposed in 1996
    % RRAM (also ReRam)
    %     resistive RAM
    %     resistive switching discovered in 1967
    %     disputed to be a memristor
    % memristors
    %     ???
    % FeRAM

% density: DRAM < STT-RAM < PCM (higher is better)
% endurance: PCM < STT-RAM < DRAM (higher is better)
% latency: DRAM <= STT-RAM < PCM (lower is better)
% dypower: STT-RAM < DRAM < PCM (lower is better)
