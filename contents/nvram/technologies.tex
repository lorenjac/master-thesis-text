The design and integration of fast \ac{NVRAM} is not a new research area. In the
past, there have been multiple attempts to produce non-volatile equivalents of
main memory. While earlier approaches were mainly designed to make systems more
tolerant to crashes \cite{molina1992main}, recent research suggests \ac{NVRAM}
to hold entire \acp{MMDB} and speed up recovery \cite{oukid2015instant,
schwalb2016hyrise, andrei2017sap}.

One way to achieve byte-addressable \ac{NVM} is to attach \ac{DRAM} or \ac{SRAM}
to backup power supplies as in \cite{liskov1991replication, wang2002conquest}.
In other cases, conventional non-volatile storage, such as flash memory, is
directly attached to the \ac{DRAM} module \cite{shi2010write, huang2014design,
oe2016feasibility}. However, these approaches rely on batteries, which must be
maintained, or block-oriented memory which is still much slower than DRAM. A
more promising approach is to develop alternative memory techniques that provide
the features of \ac{NVRAM}, natively.

Among a range of recent \ac{NVRAM} designs, the most promising are \ac{PCM} and
\ac{STT-RAM}\cite{zilberberg2013phase, mittal2016survey, jain2017computing}.
These technologies vary according to the following parameters:

\begin{itemize}
    \item density
    \item endurance
    \item latency
    \item power consumption
\end{itemize}

While \ac{PCM} is projected to have the highest density of all, including
\ac{DRAM}, it is also features a lower endurance. In terms of endurance,
\ac{STT-RAM} fares better than \ac{PCM} but is still surpassed by \ac{DRAM}.
Also, \ac{STT-RAM} has a very low density which prevents high capacity memory
modules. \ac{NVRAM} is known to incur higher access latency that \ac{DRAM}. This
is certainly true for \ac{PCM} which can have up to 10\% latency than \ac{DRAM}.
\ac{STT-RAM}, on the other hand, is projected to the on a par with \ac{DRAM}.
Concerning power consumption, \ac{STT-RAM} appears to have advantages over both
\ac{DRAM} and \ac{PCM}. In this case, \ac{PCM} registers an even higher
consumption than \ac{DRAM} \cite{mittal2016survey}.

In total, \ac{STT-RAM} poses a very promising technology but provides suboptimal
capacities and is also less enduring than \ac{DRAM}. \ac{PCM}, on the other
hand, provides a higher capacity but at the cost of higher latencies. Still,
recent research suggests \ac{PCM} to be the first fast high-capacity \ac{NVRAM}
to be ready to manufacture \cite{zilberberg2013phase, dulloor2014system,
mittal2016survey}. Nevertheless, this work is independent of the underlying
NVRAM technology.

%\subsection{Promising NVRAM Technologies}

% mittal2016survey

%\paragraph{PCM}

% lee2009architecting
% zilberberg2013phase

%\paragraph{STT-MRAM}
