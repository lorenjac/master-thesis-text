In the past, there have been multiple attempts to produce non-volatile
equivalents of main memory. One way is to emulate non-volatility by providing
backup power supplies or combining volatile \ac{DRAM} and conventional
non-volatile storage in a single module. Another promising approach is to
develop alternative memory techniques that provide the features of \ac{NVRAM}.

% This section presents notable developments in the field of \ac{NVRAM}.

\todo[inline]{Present Phase-Change Memory (Concept/History/Parameters)}

% \paragraph{Early Approaches}
%
% Initially, the intention was to make systems more tolerant to failures, although
% opportunities for storage systems were also discussed \cite{molina1992main,
% wang2002conquest}. The main issue with \ac{DRAM} is that, in order to retain its
% data, it needs to refresh its cells which requires a power supply. Therefore,
% early non-volatile main memories were equipped with batteries or \ac{UPS} which
% allowed to hold information for a limited time in case of a power failure.
% Notable examples for systems using battery-backed \ac{DRAM} are the file systems
% \emph{Harp} \cite{liskov1991replication} and \emph{Conquest}
% \cite{wang2002conquest} as well as the file cache \emph{Rio} \cite{chen1996rio}.
%
% In another approach, researchers used battery-backed \ac{SRAM} as a write buffer
% which is flushed to an interconnected flash memory when full or in case of a
% power failure \cite{wu1994envy}. By limiting access to the \ac{SRAM}, the
% non-volatile yet slower flash device is effectively shadowed. Still, since flash
% operates in block mode, the byte-addressable nature of \ac{SRAM} cannot be exploited
% in this setup. This is an early example of hybrid memory solutions for fast mass
% storage. However, backup power supplies have also been subject to criticism.
% Arguments are that batteries are not inherently reliable and introduce
% additional maintenance overhead \cite{molina1992main}. Therefore experiments
% were conducted, where flash memory was directly attached to the memory interface
% \cite{shi2010write}. Similar ideas were later consolidated in the JEDEC NV\ac{DIMM}
% specification, which defines several configurations for \ac{DIMM}s consisting of
% flash memory and \ac{DRAM} \cite{oe2016feasibility, huang2014design}.
%
% % This marks a shift in the development of \ac{NVRAM} as reliability becomes second
% % to fast high-capacity storage.
%
% \paragraph{Modern Approaches}
%
% Early approaches for practical \ac{NVRAM} were focused on making volatile \ac{DRAM} retain its information across power outages. There are however promising alternatives such as \ac{PCM}, \ac{MRAM} or \ac{RRAM} which are both byte-addressable and non-volatile by design. Although their underlying principles have been known for a while, further research was required to reach practical designs for manufacturing. Recent research suggests that these \ac{NVM} technologies will see broad availability in the near future \todo{cite}.

% technologies
    % \ac{PCM} (also \ac{PRAM})
    %     phase-change memory
    %     based on properties of chalcogenide glasses
    %     discovered in 1955
    %     first prototype in 1969
    % STT-\ac{MRAM} (also ST[T]-[M]RAM)
    %     advanced type of \ac{MRAM}
    %         magnetoresistive RAM
    %         similar to magnetic core memory (1955)
    %         GMR effect discovered in 1984
    %         developed since 1995 (Motorola)
    %     spin-transfer torque proposed in 1996
    % \ac{RRAM} (also ReRam)
    %     resistive RAM
    %     resistive switching discovered in 1967
    %     disputed to be a memristor
    % memristors
    %     ???
    % FeRAM

% density: \ac{DRAM} < STT-RAM < \ac{PCM} (higher is better)
% endurance: \ac{PCM} < STT-RAM < \ac{DRAM} (higher is better)
% latency: \ac{DRAM} <= STT-RAM < \ac{PCM} (lower is better)
% dypower: STT-RAM < \ac{DRAM} < \ac{PCM} (lower is better)
