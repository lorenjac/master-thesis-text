\chapter{Introduction}

% 1. Databases + KVS
% 2. NVRAM
% 3. Recent Research + Concurrency
% 4. Aufgabenstellung
% 5. Evaluation
% 6. Übersicht

% Databases / In-Memory Databases / Key-Value Stores

Modern database systems serve many purposes ranging from archiving to 
high-performance computing. Technologies such as search engines, trading 
and artificial intelligence rely on vast amounts of data that need to be 
processed with minimal latency. This also requires databases to be fault-
tolerant and recover quickly in case of failure.

With transactions being the predominant method to access data, a key factor 
in delivering these technologies is transaction throughput. Traditional 
databases store their data on mechanical disks. While this provides the 
necessary storage capacities, it also impairs the performance of transactions
accessing the data. In an attempt to eliminate disk I/O, main-memory databases
retain large parts or the entire database in faster but volatile main memory.
This enables significantly faster access but also makes recovery 

An important class of databases are key-value stores.

% Byte-Addressable Non-Volatile Memory

Recent research suggests novel non-volatile memory (NVM) technologies to 
complement or replace DRAM in the near future. Notable technologies include 
phase-change memory (also known as PCM or PRAM), magnetoresistive RAM (MRAM), 
and 3D XPoint. These technologies enable byte-addressable storage and are 
projected to be nearly as fast as traditional DRAM. It is also expected that 
NVM is more dense and cheaper to produce than DRAM. This makes way for new 
memory architectures and programming models.

% Transactions / Throughput / Concurrency

In order to increase throughput on multi-core architectures, most database 
systems support the concurrent execution of transactions. This way, incoming 
transactions do not have to wait for an active transaction to complete. 
However, concurrency comes at the cost of controlling which transactions may 
execute without conflicting with each other. Otherwise consistency cannot be 
guaranteed.

A traditional approach to ensure consistency is to lock all resources that are 
accessed by a transaction. When that transaction completes, all its acquired 
locks are released. A transaction that requests access to a locked resource 
has to wait until its lock is released. This method, however, may suffer from 
lock contention and deadlocks. Another approach that has seen a wide adoption 
is multiversion concurrency control. Instead of locking a resource and 
applying updates in-place, MVCC creates a new version on each update and keeps 
older versions, so that running transactions can still access them. This 
scheme has been implemented in both commercial and non-commercial transaction-
processing systems. These range from general-purpose database systems to high-
performance in-memory databases.


% TODO rework
On an architecture point of view, NVM enables new memory hierarchies by replacing disks or even DRAM. This would encourage in-memory databases based on NVM and would enable near-instantaneous restarts. It could also facilitate programming models with persistent objects.

% TODO rework
However, NVM also introduces new problems such as consistency in case of power failure and slower access. When power fails it is unclear which data have been persisted and which still are still in-flight. Such inconsistencies that did not exist for DRAM, are usually countered by flushing store buffers, caches, and write queues upon each operation that is required to be persistent. Forcing persistence however, leads to cache misses which are generally expensive. Another drawback is access time, where reading is estimated to be about as fast as with DRAM but writing is expected to significantly slower.

% TODO rework
This thesis explores different memory architectures based on NVM and evaluates its implications for concurrency control. In this regard, two fundamental control schemes, namely locking and MVCC are studied. The size of the resulting design space poses a major challenge. Not only are there many possible memory topologies to choose from but there are also different variants of MVCC. In order to deal with the complexity, several designs are evaluated on the basis of a prototype key value store. A key value store (KVS) is an appropriate tool for evaluation as it provides a very compact interface of a database. In fact, some databases rely on key value stores.

% TODO rework
The aim of this work is to find efficient concurrency control mechanisms for different memory configurations based on NVM. An important question in this regard is whether MVCC still is a valid choice over two-phase locking.

% TODO rework
The experiments are carried out on the basis of a prototype KVS. Comparing existing KVS instead would be less conclusive as structural differences are bound to yield distorted results. For meaningful results different workloads as in TATP or TPC-C are applied. As NVM is not yet commercially available, its expected latency characteristics must be simulated. This can be done via BIOS configuration, machine emulators such as GEM5, or CPU stall instructions.

% TODO rework
The implications of NVM for concurrency control has not seen much attention in recent research. Therefore the aim of this work is to provide the groundwork for database research interested in NVM technologies. Furthermore, this work provides a guide for database developers and users to properly leverage modern memory technologies according to their needs.
