\chapter{Introduction}
\label{ch:intro}

% 1. Databases + KVS
% 2. NVRAM
% 3. Recent Research + Concurrency
% 4. Aufgabenstellung
% 5. Evaluation
% 6. Übersicht

% Databases / In-Memory Databases / Key-Value Stores

Many modern technologies such as search engines or artificial intelligence rely
on databases to store large amounts of data and provide fast access. Further,
in order to perform essential tasks, such databases must also be reliable.

The predominant method to access data in a database is by using transactions. A
transaction is a sequence of operations on data that appears as single atomic
operation. Therefore, an important factor in delivering the mentioned
technologies is \textit{transaction throughput}.

There are several ways to increase transaction throughput. One approach is to
hold large portions or even the entire database in fast volatile memory as 
in \textit{main-memory databases}. In this case, recovery is crucial as
all data in volatile memory are lost in a crash. However, recovery usually
involves access to slower persistent storage media which again impairs
throughput.

Another approach to improve transaction throughput is to utilize multi-core
architectures by allowing transactions to execute concurrently. This way, an
incoming transaction does not have to wait until a running transaction
terminates. This however, introduces potential consistency issues due to race
conditions. Such conflicts must be prevented or resolved by a dedicated
\textit{concurrency control} such as locking or multiversioning.

An important class of databases that often feature both in-memory operation and
concurrent transactions are \textit{key-value stores}. Unlike relational
databases, a key-value store consists of a single associative collection.
Lacking query languages and data schemas, key-value stores avoid a lot of
overhead that may not be required in certain circumstances. A prominent use 
case is in-memory caching. Still, recovery can be taxing, especially in 
conjunction with concurrency.

% Byte-Addressable Non-Volatile Memory

Recent research suggests that byte-addressable non-volatile memory (NVM) with
parameters close to DRAM will be available in the near future. Notable
technologies currently in research are phase-change memory (PCM),
magnetoresistive RAM (MRAM), and 3D XPoint. It is suggested that, due to its
characteristics, NVM is a strong candidate for persistent main memories (also
NVRAM).

On the downside, persistence would be no longer managed by the operating system
but the underlying hardware. This can be a threat to transactional semantics in
case of crash or power failure. Enforcing persistence (e.g. by cache flushes)
on the other hand can have adverse effects on performance. Also write 
latencies are expected to be slightly higher compared to DRAM.

% Motivation

The use of NVRAM can have many implications to both operating systems and
applications. With respect to main-memory databases, recovery would be much
less of a bottleneck as disk I/O could be eliminated. Also, file systems or 
indices no longer excel from being block-oriented and should be revisited.

% For main-memory databases, in particular, recovery could be much 
% less of a bottleneck.

Another possible implication that has not seen much attention is concerned with
concurrency. Contrary to locking, multiversioning can be seen as implicit
in-memory logging that would be persistent if used with NVRAM. It is an open
question, whether (1) NVRAM has any consequence on the choice of concurrency
control and (2) this can be translated into an advantage over locking.

% Task

The task of the thesis is to explore the implications of NVRAM on concurrency
control in key-value stores. For this purpose a protoype of an in-memory
key-value store is developed and evaluated.

% Overview

\paragraph{Overview}

The thesis is structured as follows. First, chapter \ref{ch:domain} lays an 
adequate foundation on key-value stores, transaction processing, and NVRAM. 
The chapter consists of two parts. The first part introduces fundamental terms 
about key-value stores and transaction processing. The second part captures 
the current state of the art on NVRAM and the design of key-value stores.

Once the groundwork has been laid, chapter \ref{ch:concept} presents the 
concept of this work. This chapter gives an overview of the concepts and 
mechanisms involved in the design of the intended key-value store. The focus 
of this chapter lies in the partial design exploration of concurrency controls 
for NVRAM-aware key-value stores.

Based on the concept presented in the previous chapter, a prototype of an
NVRAM-aware key-value store is implemented. This process is documented in 
chapter \ref{ch:impl}.

An evaluation of the implementation based on microbenchmarks follows in 
chapter \ref{ch:eval}.

The \hyperref[ch:summary]{last chapter} provides a summary of this thesis and 
gives an outlook onto future works in the field.

