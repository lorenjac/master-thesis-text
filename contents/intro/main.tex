\chapter{Introduction}
\label{ch:intro}

% Databases / Transactions / In-Memory Databases / Key-Value Stores

% TODO improve first sentence: Databases are super important in many branches of modern society...

Many modern technologies such as search engines or artificial intelligence rely
on databases to store large amounts of data and provide fast access. Further, in
order to perform essential tasks, such databases must also be reliable.

The predominant method to access data in a database is by using transactions. A
transaction is a sequence of operations on data that appears as single atomic
operation. Therefore, an important factor in delivering the mentioned
technologies is \textit{transaction throughput}.

There are several ways to increase transaction throughput. One approach is to
hold large portions or even the entire database in fast volatile memory as in
\textit{main-memory databases}. In this case, recovery is crucial as all data in
volatile memory are lost in a crash. However, recovery usually involves access
to slower persistent storage media which impairs throughput.

Another approach to improve transaction throughput is to utilize multi-core
architectures by allowing transactions to execute concurrently. This way, an
incoming transaction does not have to wait until a running transaction
terminates. However, race conditions between concurrent transactions may result
in inconsistent data. Therefore a dedicated \textit{concurrency control} must be
employed. If a concurrency control provides \textit{serializability} then no
concurrent transactions may yield inconsistent data. Although desirable,
serializability is often unsupported or discouraged due to its performance
overhead.

An important class of databases that often feature both in-memory operation and
concurrent transactions are \textit{key-value stores}. Unlike relational
databases, a key-value store consists of a single associative collection.
Lacking query languages and data schemas, key-value stores avoid a lot of
overhead that may not be required in certain circumstances. A prominent use case
is in-memory caching.

% Byte-Addressable Non-Volatile Memory

Recent research suggests that byte-addressable non-volatile memory (NVM) with
parameters close to DRAM will be available in the near future. Notable
technologies currently in research are phase-change memory (PCM),
magnetoresistive RAM (MRAM), and 3D XPoint. It is suggested that, due to its
characteristics, NVM is a strong candidate for persistent main memories (also
NVRAM).

% On the downside, persistence would be no longer managed by the operating system
% but the underlying hardware. This can be a threat to transactional semantics in
% case of a crash or power failure. Enforcing persistence (e.g. by cache flushes)
% on the other hand can have adverse effects on performance.

% Motivation

The integration of NVRAM can have many implications to both operating systems
and applications. With projected densities higher than DRAM, NVRAM is suggested
to replace disk storage in some cases. This could significantly reduce recovery
overhead in main-memory databases. For one, databases would no longer have to
move their data from persistent storage to main-memory, thus enabling instant
restarts. More importantly, persisting recovery data would be faster by multiple
orders of magnitude.

Existing key-value stores for NVRAM have shown that transaction throughput can
be increased with NVRAM if it can hold the entire database. However, these
systems either do not support serializable transactions or rely on suboptimal
mechanisms to do so.

Nevertheless, serializability is important to guarantee data integrity and
should not be traded in favor of performance. This thesis aims to provide
affordable serializable transactions by leveraging the latency savings of NVRAM
over disks.

% Task

Hence, the task of this thesis is to design and implement a key-value store for
NVRAM that supports serializable transactions.

% Overview

\paragraph{Overview}

\todo{Überblick überarbeiten}

Once the groundwork has been laid, chapter \ref{ch:concept} presents the concept
of this work. This chapter gives an overview of the concepts and mechanisms
involved in the design of the intended key-value store. The focus of this
chapter lies in the partial design exploration of concurrency controls for
NVRAM-aware key-value stores.

Based on the concept presented in the previous chapter, a prototype of an
NVRAM-aware key-value store is implemented. This process is documented in
chapter \ref{ch:impl}.

An evaluation of the implementation based on microbenchmarks follows in chapter
\ref{ch:eval}.

The \hyperref[ch:summary]{last chapter} provides a summary of this thesis and
gives an outlook onto future works in the field.
