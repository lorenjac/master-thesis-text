\chapter{Introduction}
\label{ch:intro}

% Databases / Transactions / In-Memory Databases / Key-Value Stores

\todo[inline]{Improve first sentence of introduction}

% Databases are super important in many branches of modern society...

Many modern technologies such as search engines or artificial intelligence rely
on databases to store large amounts of data and provide fast access. Further, in
order to perform essential tasks, such databases must also be reliable.

The predominant method to access data in a database is by using \glspl{tx}. A
transaction is a sequence of operations on data that appears as a single atomic
operation. Therefore, an important factor in delivering the mentioned
technologies is \emph{transaction throughput}.

There are several ways to increase transaction throughput. One approach is to
hold large portions or even the entire database in fast volatile memory as in
\acp{MMDB}. In this case, recovery is crucial as all data in volatile memory are
lost in a crash. Therefore, recovery usually involves access to non-volatile,
yet slower storage media which impairs throughput.

Another approach to improve transaction throughput is to utilize multi-core
architectures by allowing transactions to execute concurrently. This way, an
incoming transaction does not have to wait until a running transaction
terminates. However, race conditions between concurrent transactions may result
in inconsistent data. Therefore, a dedicated \emph{concurrency control} must be
employed. If a concurrency control provides \emph{serializability} then no
concurrent transactions may yield inconsistent data. Although desirable,
serializability is often unsupported or discouraged due to its performance
overhead.

An important class of databases that often feature both in-memory operation and
concurrent transactions are \emph{\kvsp}. Unlike relational databases, a \kvs
consists of a single associative collection. Lacking query languages and data
schemas, \kvsp avoid a lot of overhead that may not be required in certain
circumstances. A prominent use case is in-memory caching.

% Byte-Addressable Non-Volatile Memory

Recent research suggests that byte-addressable \ac{NVM} with parameters close to
\ac{DRAM} will be available in the near future. Notable technologies currently
in research are \ac{PCM}, \ac{MRAM}, and 3D XPoint. It is suggested that, due to
its characteristics, these technologies are strong candidates for non-volatile
main memories. Commonly accepted terms for such memory are \ac{NVRAM} and
\ac{SCM}.

% Motivation

The integration of \ac{NVRAM} can have many implications to both operating
systems and applications. With projected densities higher than \ac{DRAM},
\ac{NVRAM} is suggested to replace disk storage in some cases. This could
significantly reduce recovery overhead in \acp{MMDB}. For one, databases would
no longer have to move their data from non-volatile storage to main memory, thus
enabling instant restarts. More importantly, persisting recovery data would be
faster by multiple orders of magnitude.

Existing \kvsp for \ac{NVRAM} have shown that transaction throughput can be
increased with \ac{NVRAM} if it can hold the entire database. However, these
systems either do not support serializable transactions or rely on suboptimal
mechanisms to do so.

Nevertheless, serializability is important to guarantee data integrity and
should not be traded in favor of performance. Upcoming \ac{NVRAM} technologies
pose an opportunity to satisfy both demands. This thesis aims to provide
affordable serializable transactions by leveraging the latency savings of
\ac{NVRAM} over disks.

% Task

The task is to design and implement an in-memory key-value store that exploits
the benefits of upcoming non-volatile \ac{RAM} to enable efficient serializable
transactions. For this purpose, an \ac{NVRAM}-aware multiversion concurrency
control protocol is implemented. In order to validate the approach and determine
the overhead of serializability, benchmarks are used to compare the key-value
store against non-serializable solutions.

% Overview

\paragraph{Overview}

First, a domain analysis in the fields of \ac{NVRAM} and key-value stores is
conducted. Chapter \ref{ch:nvram} introduces \ac{NVRAM} and discusses
applications and challenges such as consistency guarantees in the presence of
power failures. Next, chapter \ref{ch:kvs} introduces \kvsp and modern
concurrency control protocols.

Following the domain analysis, chapter \ref{ch:concept} provides a comprehensive
design space exploration on \kvsp with serializable transactions on systems with
byte-addressable \ac{NVM}. Based on these insights, the concept of the intended
\kvs is presented. Details concerning the implementation of the concept are
discussed in chapter \ref{ch:impl}. Afterwards, chapter \ref{ch:eval} gives an
evaluation based on benchmarks.

The \hyperref[ch:summary]{last chapter} provides a summary of this thesis and
gives an outlook onto future works in the field.

\paragraph{Terminology}

As of this writing there is no distinct consensus as to how byte-addressable
non-volatile memory should be referred to. This thesis exclusively uses term
\ac{NVRAM}. There are two reasons for this decision. For one, alternative terms
such as \ac{BPRAM} or \ac{PM} suggest that non-volatile memory is also
persistent which has ambiguous definitions and is not consistently used
\cite{volos2017whisper}. Another reason is that some terms such as \ac{SCM},
\ac{NVM}, and \ac{PM} may not reflect the property of bytewise addressing which
is central to these technologies. \ac{NVRAM} on the other hand, explicitly
denotes all of the desired properties.
