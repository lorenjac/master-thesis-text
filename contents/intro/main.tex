\chapter{Introduction}
\label{ch:intro}

% Databases / Transactions / In-Memory Databases / Key-Value Stores

% TODO improve first sentence: Databases are super important in many branches of modern society...

Many modern technologies such as search engines or artificial intelligence rely
on databases to store large amounts of data and provide fast access. Further, in
order to perform essential tasks, such databases must also be reliable.

The predominant method to access data in a database is by using transactions. A
transaction is a sequence of operations on data that appears as single atomic
operation. Therefore, an important factor in delivering the mentioned
technologies is \textit{transaction throughput}.

There are several ways to increase transaction throughput. One approach is to
hold large portions or even the entire database in fast volatile memory as in
\acp{MMDB}. In this case, recovery is crucial as all data in volatile memory are
lost in a crash. However, recovery usually involves access to slower persistent
storage media which impairs throughput.

Another approach to improve transaction throughput is to utilize multi-core
architectures by allowing transactions to execute concurrently. This way, an
incoming transaction does not have to wait until a running transaction
terminates. However, race conditions between concurrent transactions may result
in inconsistent data. Therefore a dedicated \textit{concurrency control} must be
employed. If a concurrency control provides \textit{serializability} then no
concurrent transactions may yield inconsistent data. Although desirable,
serializability is often unsupported or discouraged due to its performance
overhead.

An important class of databases that often feature both in-memory operation and
concurrent transactions are \textit{\kvsp}. Unlike relational databases, a \kvs
consists of a single associative collection. Lacking query languages and data
schemas, \kvsp avoid a lot of overhead that may not be required in certain
circumstances. A prominent use case is in-memory caching.

% Byte-Addressable Non-Volatile Memory

Recent research suggests that byte-addressable \ac{NVM} with parameters close to
\ac{DRAM} will be available in the near future. Notable technologies currently
in research are \ac{PCM}, \ac{MRAM}, and 3D XPoint. It is suggested that, due to
its characteristics, \ac{NVM} is a strong candidate for persistent main memories
(also \ac{NVRAM}).

% On the downside, persistence would be no longer managed by the operating system
% but the underlying hardware. This can be a threat to transactional semantics in
% case of a crash or power failure. Enforcing persistence (e.g. by cache flushes)
% on the other hand can have adverse effects on performance.

% Motivation

The integration of \ac{NVRAM} can have many implications to both operating
systems and applications. With projected densities higher than \ac{DRAM},
\ac{NVRAM} is suggested to replace disk storage in some cases. This could
significantly reduce recovery overhead in \acp{MMDB}. For one, databases would
no longer have to move their data from persistent storage to main memory, thus
enabling instant restarts. More importantly, persisting recovery data would be
faster by multiple orders of magnitude.

Existing \kvsp for \ac{NVRAM} have shown that transaction throughput can be
increased with \ac{NVRAM} if it can hold the entire database. However, these
systems either do not support serializable transactions or rely on suboptimal
mechanisms to do so.

Nevertheless, serializability is important to guarantee data integrity and
should not be traded in favor of performance. This thesis aims to provide
affordable serializable transactions by leveraging the latency savings of
\ac{NVRAM} over disks.

% Task

Hence, the task of this thesis is to design and implement a \kvs for \ac{NVRAM}
that supports serializable transactions.

% Overview

\paragraph{Overview}

First, a thorough domain analysis in the fields of \ac{NVRAM} and key-value
stores is conducted. Chapter \ref{ch:nvram} introduces \ac{NVRAM} and discusses
the issue of persistence guarantees in the presence of power failures. Next,
chapter \ref{ch:kvs} introduces \kvsp and modern concurrency control protocols.

% In chapter \ref{ch:kvs} the concept of \kvsp is explained and
% existing concurrency controls are presented.

Following the domain analysis, chapter \ref{ch:dse} provides a comprehensive
design space exploration on \kvsp with serializable transactions on systems with
byte-addressable \ac{NVM}. Based on these insights, the concept of the intended
\kvs is then presented in chapter \ref{ch:concept}.

A prototype of the developed concept has been implemented. Notable details
concerning this implementation are discussed in chapter \ref{ch:impl}.
Afterwards, chapter \ref{ch:eval} gives an evaluation based on microbenchmarks.

The \hyperref[ch:summary]{last chapter} provides a summary of this thesis and
gives an outlook onto future works in the field.
