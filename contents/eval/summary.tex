This chapter is dedicated to determine whether MMDB can leverage NVRAM to make
the isolation level of serializability affordable. For that purpose, the
serializable KVS from Chapter \ref{ch:impl}, Midas, and the non-serializable KVS
Echo are compared in terms of transaction throughput and scalability.

In three out of four scenarios, the original implementation of Midas performs
significantly worse than Echo. Especially in high-contention scenarios, it shows
much higher abort rates, lower throughput, and insignificant speedup. However,
in a low contention scenario (S3) Midas achieves higher absolute throughput.
This is a very promising result, as it shows that even with serializability MVCC
could continue to excel in low-contention scenarios. However, despite negligible
abort rates, which are on par with Echo, Midas shows poor scalability. This is
due to the non-concurrent hash table implementation which requires a global lock
for synchronization.

Since the index is not part of the synchronization concept, Midas$^{*}$ attempts
to approximate the performance of a lock-free concurrent hash table. Midas$^{*}$
has significantly higher throughput and scales well up until 16 cores, while
maintaining comparable abort rates. However, this result may be too optimistic
because concurrent data structures are often slower than their non-synchronized
counterparts. Midas with a concurrent index strucure may perform worse but still
better than Echo.

Echo has much lower abort rates in all scenarios, but only provides snapshot
isolation. Moreover, it fails to leverage the advantage and hardly scales beyond
8 cores. Possible reasons for that include excessive synchronization and
inefficient NVRAM consistency procedures.

The benchmark shows that serializability may become affordable with NVRAM after
all. Except in very high contention scenarios, Midas$^{*}$ can provide better
throughput and scale better than Echo. Moreover, if contention is low, even
Midas can perform better than its non-serializable counterpart Echo.
