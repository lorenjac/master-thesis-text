The benchmarks are conducted on a platform of four Intel Xeon E7-4830 processors
with a total of 32 cores running at 2.13GHz. Main memory consists of 256GB DDR3
SDRAM running at 1066Mhz. For durable storage, the system relies on mechanical
hard drives. The operating system is CentOS 7.4 based on Linux 3.10.

Since NVRAM is not available, all benchmarks are carried out on volatile DRAM.
In order to emulate persistency, each KVS stores its contents in a memory-mapped
file. To satisfy the consistency requirements shown in Chapter
\ref{ch:nvram-consistency}, it is important to prevent potential buffering of
data in kernel page caches. This is achieved by using a 127GB instance of
\emph{tmpfs} as RAM disk. However, even with this setup, memory pages may still
be swapped at will by the OS. In order to prevent this, swapping is disabled. As
a result, the hard disk is never used and no write operation within the mapped
region is deferred.

Promising NVRAM technologies such as PCM do incur higher latencies than DRAM. In
accordance with \cite{bailey2013exploring, zhou2016nvht}, this work refrains
from emulating access latencies for the same reasons given in Chapter
\ref{ch:eval-challenges}.
