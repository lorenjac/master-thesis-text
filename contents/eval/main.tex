\chapter{Evaluation}
\label{ch:eval}

The previous chapter elaborates on a prototypical implementation of the concept
introduced in Chapter \ref{ch:concept}. The goal is to enable fast serializable
database transactions through the use of novel byte-addressable non-volatile
memory. Based on the prototype, this chapter attempts to determine the
effectiveness of that concept.

It has been shown, that MMDB can benefit from NVRAM \cite{oukid2017data,
andrei2017sap}. However, recent database research still trades serializable
transactions for performance \cite{bailey2013exploring}. If NVRAM can be
leveraged to reduce or compensate serialization overhead, then serializability
could be adopted as the default isolation level in systems featuring NVRAM.
Therefore, the aim of this work is to determine whether the approach of enabling
fast serializable transactions through NVRAM is worth pursuing.

An important measure in this matter is transaction throughput, as it is known to
decrease with higher isolation levels. For that purpose, it is investigated
whether transaction throughput with serializability comes sufficiently close to
non-serializable systems and scales accordingly. While this statement may be too
vague for production it is sufficient for an initial work in this area.

The evaluation is based on a performance and scalability comparison between the
implemented prototype, labeled as Midas, and the NVRAM-aware KVS Echo from
Chapter \ref{ch:kvs}. It consists of two benchmarks: a baseline benchmark and a
transaction throughput benchmark. The throughput benchmark is used to compare
the transaction throughput of both stores and how well it scales with additional
processors. The baseline benchmark captures average operation latencies for each
store, which are helpful to accomodate for differences in the throughput
benchmark.

% At first, however, there a couple of challenges to discuss.

% \section{Challenges}

\todo[inline]{Optional: Discuss challenges regarding database benchmarks}

\section{Baseline}
\section{Transaction Throughput}

% \section{Summary}
