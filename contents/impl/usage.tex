In order to get an understanding of how Midas can be used, this section presents
a selected set of usage examples. The first example deals with how an instance
of Midas is created or recovered, respectively. The remaining examples showcase
the transaction API of Midas.

Before Midas can be instantiated, an underlying persistent object pool must be
created or recovered from the file system. Details on object pools are given in
the next section. Once the object pool is initialized, Midas can be instantiated
and transactions can be performed. The example is given in
Listing~\ref{lst:usage-init}.

\begin{lstlisting}[caption={Midas Bootstrapping.},captionpos=b, label=lst:usage-init]
#include "midas.hpp"

/* declarations for this example */

int main() {
    // Create persistent object pool
    midas::pop_type pop;

    // Initial size of the object pool (here: 1GB)
    const size_t pop_size = 1024ULL * 1024 * 1024;

    // Initialize persistent object pool from file.
    if (!midas::init(pop, "midas.db", pop_size))
        return EXIT_FAILURE;

    // Create Midas key-value store based on object pool
    midas::Store store(pop);

    // Run example transactions
    run_single_transaction(store);
    run_concurrent_transactions(store);

    return EXIT_SUCCESS;
}
\end{lstlisting}

Midas uses an ordinary transaction API. In order to execute a transaction, it
must first be started with a call to \code{Store::begin()} which yields a handle
to the new transaction. Using the transaction handle, the store can be modified
in the context of the associated transaction. The available modifiers are
\code{Store::put()}, \code{Store::get()}, and \code{Store::drop()}. As a
compromise between usability and flexibility, the store supports strings as keys
and values, exclusively. Unless a transaction is aborted prematurely, it must
be concluded with a call to \code{Store::commit()}. The resulting return code
denotes whether the transaction committed successfully or failed. The following
example shows a transaction checking the balance of an account and applying an
interest fee if the balance is negative (see Listing~\ref{lst:usage-single-tx}).

\begin{lstlisting}[caption={A single transaction on account data.},captionpos=b, label=lst:usage-single-tx]
void run_single_transaction(midas::Store& store) {
    midas::Transaction::ptr tx = store.begin();

    std::string deb = store.get("debit", tx);
    std::string sav = store.get("saving", tx);
    if (get_balance(sav, deb) < 0.0)
        store.put("debit", apply_interest(deb), tx);

    if (!store->commit(tx))
        // Error: transaction failed
}
\end{lstlisting}

Running transactions in concurrent threads requires no additional setup other
than creating the respective threads to begin with. Each thread only needs a
reference to the Midas store. While the store is a global resource, each
transaction buffers its updates in its private memory region. The last example
starts 10 threads which all execute the accounting transaction from
Listing~\ref{lst:usage-single-tx}. Because this transaction may modify a value
that is read by all concurrent transactions, many transactions are bound to fail
due to read-write conflicts. The example is shown in
Listing~\ref{lst:usage-concurrent-tx}.

\begin{lstlisting}[caption={Several concurrent accounting transactions on the same store.},captionpos=b, label=lst:usage-concurrent-tx]
void run_concurrent_transactions(midas::Store& store) {
    std::vector<std::thread> threads;
    for (unsigned i=0; i<10; ++i)
        threads.push_back(
            std::thread([&](){ run_single_transaction(store); }));

    for (auto& t : threads)
        t.join();
}
\end{lstlisting}
