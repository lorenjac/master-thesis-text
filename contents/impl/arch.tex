\todo[inline]{impl: describe architecture of the store (layers, components)}

\subsection{NVRAM Management}
\label{ch:impl-pmdk}

As explained in Chapter \ref{ch:nvram} there are several challenges to the integration of NVRAM into current systems.

\begin{itemize}
    \item Accessing NVRAM
    \item Programming against NVRAM
    \begin{itemize}
        \item Separation of volatile from non-volatile data
        \item Recovery of non-volatile objects after restart
        \item Preserving consistency across restarts
    \end{itemize}
\end{itemize}

In the past, several approaches have been discussed to address these challenges.
Recent research shows rising interest in PMDK (formerly known as NVML) which
appears to be the first practical holistic approach to managing NVRAM.
Therefore, this work uses PMDK to implement NVRAM management.

\subsubsection{NVRAM Access}

In PMDK, NVRAM is accessed via ordinary filesystems such as \code{ext4} or
\code{tmpfs}. That way, NVRAM can be mapped into a process' address space via
plain files. Using a novel kernel feature called \code{DAX} (for direct access)
programs can bypass the operating systems page cache. As a result, true load and
store semantics are feasible. Ideally, swapping is disabled for the respective
memory region.

\subsubsection{Programming Model}

Apart from its core, the essential part of PMDK used in this work is called
\code{pmemobj}. With it, a persistent object pool holds all data that are meant
to be durable. Memory inside the object pool is managed by a designated memory
allocator. For each object which is meant to be durable, it allocates the
required amount of memory and registers the resulting object in the object pool.
This way, all objects can be recovered on restart. From a programmer
perspective, objects are managed in a graph rooted in the object pool. Using
this graph, which is recovered after restart, the programmer can access all
previously durable objects.

\paragraph{Durable Data}

In order to separate volatile from non-volatile objects, there are templated
wrapper classes to manage integral values and dynamically allocated objects:

\begin{itemize}
    \item \code{pmdk::p<T>} for durable integral values
    \item \code{pmdk::persistent\_ptr<T>} for pointers to durable objects
\end{itemize}

\paragraph{Durable Objects \& Pointers}

A \code{persistent\_ptr} consists of a non-volatile unique object id and a
virtual memory address within the object pool. The virtual address is volatile
because it is not valid across restarts. Therefore, the object id is mapped to a
relative memory address within the object pool. Using the object id, the virtual
memory address of an object can be computed from its relative address to the
pool base on every restart. This is necessary, because in most operating systems
virtual address space layout is different for each session. When creating an
object, the function \code{pmdk::make\_persistent<T>()} is used. The lifetime of
a durable object spans across scopes and restarts. In order to deallocate an
object, the function \code{pmdk::delete\_persistent<T>()} must be used.

\paragraph{Preserving Consistency}

In order to ensure consistency across crashes in NVRAM, stores must be flushed
immediately. In PMDK, this can be done explicitly for selected memory regions or
implicitly by using a so-called transaction. A transaction receives the current
object pool and a functor. Each persistent property or pointer that is modified
by the functor, registers itself with the transaction. When the transaction
commits, all registered objects are made durable in NVRAM. A transaction commits
when its functor completes without errors. In the example below a durable object
of type \code{T} is created inside a pool which has a graph root of type
\code{Root} (see Listing \ref{lst:pmdk-tx}). Assume the global variable
\code{objCount} is supposed to reflect the correct number of durable objects.
For that purpose, both creating of objects and incrementing the counter are
consolidated in a transaction to form a single memory update.

\begin{figure}[h]
\begin{lstlisting}
#include "libpmemobj++/persistent_ptr.hpp"
#include "libpmemobj++/transaction.hpp"
#include "libpmemobj++/pool.hpp"
namespace pmdk = pmem::obj;

pmdk::p<size_t> objCount = 0;

template <class T, class Root>
pmdk::persistent_ptr<T> create(pmdk::pool<Root> pop)
{
    pmdk::persistent_ptr<T> obj = nullptr;

    // Consolidate updates in transaction
    pmdk::transaction::exec_tx(pop, [&, this](){
         obj = pmdk::make_persistent<T>();
         objCount = objCount + 1;
    });

    // At this point, all changes to NVM are either completely durable or absent
    return obj;
}
\end{lstlisting}
\caption{Consistent updates to NVM using PMDK transactions.}
\label{lst:pmdk-tx}
\end{figure}

Even though object creation is always transactional in PMDK, initialization is
not. Instead, the memory of an object is only nulled. As a result, each class
intended for durability must be \emph{trivially-default-constructible}, which
means that a compiler-generated default constructor must be sufficient to
initialize an instance of the class. If a default constructor is not enough,
then the programmer has to provide and use custom initialization methods.

Internally, PMDK uses largely the same mechanism for consistency as described in
Chapter \ref{ch:nvram-consistency}. That means, modified data are flushed from
memory order buffers and caches. In some situations, non-temporal writes are
used which make use of write combine buffers. When it comes to flushing
write-pending queues inside the memory controller, PMDK relies on ADR.

% =============================================================================
% =============================================================================
% =============================================================================

\subsection{Durable Data Structures}
\label{ch:impl-data}

\subsubsection{Linked List}

\subsubsection{Hash Table}

% * policy-based design for
%     * hashing volatile and non-volatile keys
%         * user does not know about a different represent. of obj. in pmem
%         * volatile keys must have equal hashes with non-volatile counterparts
%         * for safety many hash functions are not equal across restarts
%             * => cannot find pairs after restart
%             * => a) use fixed hash function (impl. from JDK)
%             * => b) rehash entire table (expensive!)
%     * memory management (initial size, load factor, growth factor)
% * conflict handling
%     * chaining (pairs of colliding keys are inserted in a list (aka bucket))
%     * use linked list for buckets
%     * => table is array of buckets

% =============================================================================
% =============================================================================
% =============================================================================

\subsection{Serializable MVCC}
\label{ch:impl-mvcc}

\subsubsection{Transactions}

\subsubsection{Versioning}
\paragraph{Versions}
\paragraph{Histories}
\paragraph{Visibility}

\subsubsection{Conflicts}
