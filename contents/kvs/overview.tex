KVS form an integral part in modern database technology \cite{fiebig2016one}.
This section gives an overview of their properties, classes, and applications.
Compared to other types of databases, KVS are very simple databases that are
sometimes better described by what they are not or do not provide:

\begin{itemize}
    \item Non-relational data model
    \item No data schemas
    \item No query languages
\end{itemize}

In general, a KVS consists of a single associative container, where each key is
mapped to exactly one value. A key is an arbitrary string with possible
restrictions on its length. In terms of relational databases, KVS comprise a
single table of two columns. As a result, much of the structural complexity
adherent to relational DBMS is omitted, thus making way for profound
optimization and better response times. Common data structures for associativity
in KVS are hash tables and search trees, in particular B-trees.

Unlike traditional databases, KVS do not impose data schemas. Consequently,
arbitrary chunks of data can be stored as values which is especially useful
in scenarios with no fixed data format or when enforcing one is not a priority.

\begin{lstlisting}
| user | Max                  |
| auth | dxlf240r0g7jr4u8n2oe |
| ...  | ...                  |
\end{lstlisting}

Furthermore, KVS do not provide query languages such as SQL to
store and retrieve data. Instead, KVS are accessed programmatically through a
concise set of operations which is why KVS are also referred to as
\emph{embedded} databases. Although their API is not standardized, it can be
essentially broken down to the following operations.

\begin{itemize}
    \item Open/Close
    \item Insertion
    \item Retrieval
    \item Removal
\end{itemize}

\paragraph{Applications}

Traditional DBMS are often based on complex architectures featuring query
frontends and sophisticated storage mechanisms. While this works well in many
cases, it severely limits the performance in situations where a simpler storage
paradigm (e.g. key-value pairs) is sufficient. As a consequence, high access
latencies and convoluted, error-prone concurrency schemes inhibit the
scalability of storage systems. KVS on the other hand are designed to compensate
for these shortcomings. A driving force in this regard, are large internet
platforms, e-commerce for instance, and cloud computing services.

A longstanding example of a KVS is BerkeleyDB which acts as a database in a
variety of software solutions. Apart from open-source software such as OpenLDAP
or Apache Web server, BerkeleyDB is also used in a number of proprietary
software such as messaging servers, switches, and routers
\cite{kaestner2007aspect, olson1999berkeley}.

A more recent use case are distributed in-memory caches often found in big-data
environments. Web caches have received great attention as service providers
struggle to scale with rising traffic where many requests target only a small
amount of data \cite{xu2014characterizing}. With caching, a dedicated eviction
policy ensures that \emph{relevant} items reside in memory. As a result, caching
can improve response times significantly. For this purpose, KVS provide an
appropriate abstraction. Important representatives of this class are Redis and
memcached \cite{redis2017home, memcached2017home}. Not only have these KVS been
deployed at companies such as Facebook or Twitter, but they have also formed the
basis for considerable amounts of research in this area
\cite{xu2014characterizing}. Examples include FPGA acceleration
\cite{lavasani2014fpga}, memory partitioning for better cache hit rates
\cite{carra2014memory}, and NVRAM integration \cite{wu2016nvmcached,
malinowski2017using, venkataraman2011consistent}. Still, large companies tend to
maintain in-house solutions to suit their needs \cite{chang2008bigtable,
decandia2007dynamo, lakshman2010cassandra, wang2015hydradb}.

Beyond databases and caches, KVS have also been proposed as a basis for file
systems. In the past, there have been several attempts to integrate database
concepts into file systems, some of which are logging \cite{rosenblum1992design,
tweedie1998journaling} and transactions \cite{seltzer1990transaction,
wright2007extending, spillane2009enabling}. Some studies even suggest that
traditional hierarchical file systems may often be suboptimal
\cite{stein2005stupid, seltzer2009hierarchical}. While databases in general are
still considered too heavy-weight for use in file systems
\cite{seltzer2009hierarchical}, KVS may be a viable alternative. Examples
include the network file system DBFS which is based on BerkeleyDB
\cite{murphy2002design} and FlatFS, a simple file system for NVRAM
\cite{volos2014aerie}. In addition, KVS are also used to complement file
systems, for example, to store metadata as in PVFS \cite{carns2009small}. Still,
the predominant use case of KVS is found in light-weight databases and caches on
top of existing file systems.

\paragraph{Transactions}

An essential feature of most databases is transactions. Transactions enable a
sequence of database operations to appear as a single atomic operation. If a
single operation involved in a transaction fails, the entire transaction fails and its side effects are rolled back.

Transactions are a powerful mechanism that enables aggregated operations without
worrying about inconsistencies even in case of failure. Given the prevalence of
transactions, most KVS support them. Due to their importance for this work,
transactions are covered in more detail in the next section.

\paragraph{In-Memory Operation}

The performance of a database is often denoted in terms of transaction
throughput. One way to increase throughput is to mitigate data access latencies.
Apart from faster storage, this can be done by placing the entire database in
main memory which enables speedups by multiple orders of magnitude. This
approach, which dates back to the mid 1980s has been adopted in many
high-performance databases such as the more recent HANA database
\cite{molina1992main, faerber2012hana}. Likewise, most KVS are explicitly
designed for in-memory operation. Notable exceptions are the popular BerkeleyDB
or Apache's Cassandra where in-memory operation is only an option
\cite{bdb2017doc, lakshman2010cassandra}.

\paragraph{Concurrency}

Another approach to increase transaction throughput is to utilize multi-core
processors by executing transactions concurrently. In order to achieve maximum
performance, it is common for main-memory databases to also support concurrency
\cite{grund2010hyrise, faerber2012hana, diaconu2013hekaton}. Further, it has
been shown that KVS can gain substantial performance benefits through
concurrency \cite{fan2013memc3, li2015architecting, xu2014building}. In fact,
most KVS natively support concurrency with the exception of Redis
\cite{redis2017home}. Unfortunately, concurrency also introduces new issues such
as inconsistencies through race conditions on shared data. Mitigating this issue
can degrade performance which is why many designs trade full consistency against
faster relaxations \cite{decandia2007dynamo}. This issue is dealt with in the
next section about transactions.

\paragraph{Distributed Databases}

As mentioned earlier, KVS play a crucial role in big-data environments. Since
availability is often a requirement in this area, KVS are often implemented as
distributed services \cite{decandia2007dynamo, lakshman2010cassandra,
wang2015hydradb}. Distributed databases and their mechanisms such as distributed
transactions are beyond the scope of this work.
