\chapter{Key-Value Stores}
\label{ch:kvs}

% Context

Recent research suggests the use of novel non-volatile memory as persistent
main memory. Apart from being byte-addressable, the upcoming modules are
projected to be almost as fast as DRAM and provide a higher capacity. Among
others, this could have considerable consequences for databases such as
key-value stores. Imagine recovery without retrieving checkpoints from disk.
Nevertheless, this work aims to assess the influence of NVRAM on the design of
concurrency control.

% Overview (coarse)

This chapter provides a brief overview on key-value stores with regard to
upcoming NVRAM. But before giving insight into the current state of the art,
a few fundamental terms must be introduced. For this purpose the chapter has
been separated into two parts.

% Overview (fine)

The first part introduces the concept of key-value stores with an emphasis on
managing concurrent transactions. Important aspects in this regard are
concurrency control and serializability. In the second part, a comprehensive
analysis on novel non-volatile memory and modern key-value stores is conducted.

% Basics

%\section{Basics}

% introduce fundamental terms and concepts about key-value stores and 
% transaction processing

% focus is on management of concurrent transactions
% => ensure data consistency and recoverability

\subsection{Key-Value Stores}

% Definition (DB -> KVS: no queries, no schema, single associative collection)

% Structure / API (???)

\subsection{Transactions}

% Definition (versch. Arten: flat, nested, distributed)

% Eigenschaften -> ACID (Isolation/Serialization später!)

% Modellierung (bound to ADTs, bound to memory = STM)

\subsection{Concurrency Control}

% Problem -> Race Conditions -> Conflicts (RW, WR, WW)

% Task -> ensure consistency, recover when in doubt

% Serializability

% Methods (Locking, MVCC: Snapshot Isolation)



% State of the Art

%\section{State of the Art}

% \subsection{Byte-Addressable Non-Volatile Memory}

% 1) Idea / Motivation
% 2) Technologies (example for PCM / 3D XPoint)
% 3) Implications (bailey2013)
%       Transactional Semantics
%           OS no longer manages persistence
%           hardware is unaware
%       Virtual Memory
%       Filesystems
%       Unwanted Persistence
%           remaining confidential data (security)
%           remaining crash data (safety)
%       Databases
%           cheaper logging/checkpointing
%           cheaper recovery (rollback, restart)
% 4) Ensuring Persistence

% \subsection{Key-Value Stores}

% 1) Key-Value Stores
%       redis, memcached, aerospike, ... (used at Amazon, Google, ...)
%       echo, nvht, berkeley db, flatfs
% 2) Applications
%       databases, caches, file systems
% 3) Data Structures
%       b-trees
%       hash tables
%       radix tree (proposed but not seen in action)
%       (non-)blocking???
% 4) Concurrency Control
%       locking
%           two-phase locking (2PL)
%       MVCC
%           (serializable) Snapshot Isolation
%           custom MVCC (neumann2015)


\section{Overview}
\section{Concurrency Control}
\subsection{Locking}
\subsection{Multiversioning}
\subsection{Serializable MVCC}
