\chapter{Key-Value Stores}
\label{ch:kvs}

A prominent use case for NVRAM are main-memory databases. Even though main
memory and processor caches have become more affordable, MMDB still suffer from
recovery on slower disk drives \cite{oukid2015instant, schwalb2016hyrise}. NVRAM
on the other hand, provides an opportunity to eliminate recovery altogether. An
important class of databases often implemented as MMDB is KVS. Due to their
simplicity and low overhead, KVS have been adopted both in big-data computing
and database research \cite{decandia2007dynamo, lakshman2010cassandra,
wang2015hydradb}. In recent works, KVS were used to explore database design for
NVRAM \cite{bailey2013exploring, zhou2016nvht, wu2016nvmcached}.

This chapter provides a domain analysis of KVS. First, a brief overview of KVS
is given. The aim of this work is to exploit NVRAM for a KVS with fast
conflict-free concurrent transactions. Therefore, a substantial part of the
remaining chapter is dedicated to transactions and concurrency control. The
chapter is concluded with an examination of existing KVS for NVRAM.

\section{Overview}
\ac{KVS} form an integral part in modern database technology
\cite{fiebig2016one}. This section gives an overview of their properties,
classes, and applications. Compared to other types of databases, \acp{KVS} are
very simple databases that are sometimes better described by what they are not
or do not provide:

\begin{itemize}
    \item non-relational data model
    \item no data schemas
    \item no query languages
\end{itemize}

In general, a \ac{KVS} consists of a single associative container, where each
key is mapped to exactly one value. A key is an arbitrary string with possible
restrictions on its length. In terms of relational databases, \acp{KVS} comprise
a single table of two columns. As a result, much of the structural complexity
adherent to relational \ac{DBMS} is omitted, thus making way for profound
optimization and better response times. Common data structures for associativity
in \ac{KVS} are hash tables and search trees, in particular B-trees.

Unlike traditional databases, \acp{KVS} do not impose data schemas. Consequently,
arbitrary chunks of data can be stored as values which is especially useful in
scenarios with no fixed data format or when enforcing one is not a priority.
Furthermore, \acp{KVS} do not provide query languages such as SQL to store and
retrieve data. Instead, \acp{KVS} are accessed programmatically through a concise
set of operations which is why \acp{KVS} are also referred to as \emph{embedded}
databases. Although their \ac{API} is not standardized, it can be essentially
broken down to the following operations:

\begin{itemize}
    \item insertion
    \item retrieval
    \item removal
\end{itemize}

\paragraph{Applications}

Traditional \acp{DBMS} are often based on complex architectures featuring query
front ends and sophisticated storage mechanisms. While this works well in many
cases, it severely limits the performance in situations where a simpler storage
paradigm (e.g. key-value pairs) is sufficient. As a consequence, high access
latencies and convoluted, error-prone concurrency schemes inhibit the
scalability of storage systems. \acp{KVS} on the other hand are designed to
compensate for these shortcomings. A driving force in this regard, are large
internet platforms, e-commerce for instance, and cloud computing services.

A longstanding example of a \ac{KVS} is BerkeleyDB which acts as a database in a
variety of software solutions. Apart from open-source software such as OpenLDAP
or Apache Web server, BerkeleyDB is also used in a number of proprietary
software such as messaging servers, switches, and routers
\cite{kaestner2007aspect, olson1999berkeley}.

A more recent use case are distributed in-memory caches often found in big-data
environments. Web caches have received great attention as service providers
struggle to scale with rising traffic where many requests target only a small
amount of data \cite{xu2014characterizing}. With caching, a dedicated eviction
policy ensures that \emph{relevant} items reside in memory. As a result, caching
can improve response times significantly. For this purpose, \acp{KVS} provide an
appropriate abstraction. Important representatives of this class are Redis and
memcached \cite{redis2017home, memcached2017home}. Not only have these \acp{KVS}
been deployed at companies such as Facebook or Twitter, but they have also
formed the basis for considerable amounts of research in this area
\cite{xu2014characterizing}. Examples include \ac{FPGA} acceleration
\cite{lavasani2014fpga}, memory partitioning for better cache hit rates
\cite{carra2014memory}, and \ac{NVRAM} integration \cite{wu2016nvmcached,
malinowski2017using, venkataraman2011consistent}. Still, large companies tend to
maintain in-house solutions to suit their needs \cite{chang2008bigtable,
decandia2007dynamo, lakshman2010cassandra, wang2015hydradb}.

Beyond databases and caches, \acp{KVS} have also been proposed as a basis for
file systems. In the past, there have been several attempts to integrate
database concepts into file systems, some of which are logging
\cite{rosenblum1992design, tweedie1998journaling} and transactions
\cite{seltzer1990transaction, wright2007extending, spillane2009enabling}. Some
studies even suggest that traditional hierarchical file systems may often be
suboptimal \cite{stein2005stupid, seltzer2009hierarchical}. While databases in
general are still considered too heavy-weight for use in file systems
\cite{seltzer2009hierarchical}, \acp{KVS} may be a viable alternative. Examples
include the network file system DBFS which is based on BerkeleyDB
\cite{murphy2002design} and FlatFS, a simple file system for \ac{NVRAM}
\cite{volos2014aerie}. In addition, \acp{KVS} are also used to complement file
systems, for example, to store metadata as in PVFS \cite{carns2009small}. Still,
the predominant use case of \ac{KVS} is found in light-weight databases and
caches on top of existing file systems.

\paragraph{Transactions}

An essential feature of most databases are transactions. Transactions enable a
sequence of database operations to appear as a single atomic operation. If a
single operation involved in a transaction fails, the entire transaction fails
and its side effects are rolled back. Transactions are a powerful mechanism that enables aggregated operations without
worrying about inconsistencies even in case of failure. Given the prevalence of
transactions, most \acp{KVS} support them. A notable exception is memcached
\cite{wu2016nvmcached}. Due to their importance for this work, transactions are
covered in more detail in the next section.

\paragraph{In-Memory Operation}

The performance of a database is often denoted in terms of transaction
throughput. One way to increase throughput is to mitigate data access latencies.
Apart from faster storage, this can be done by placing the entire database in
main memory which enables speedups by multiple orders of magnitude. This
approach, which dates back to the mid 1980s, has been adopted in many
high-performance databases such as the more recent HANA database
\cite{molina1992main, faerber2012hana}. Likewise, most \acp{KVS} are explicitly
designed for in-memory operation. Notable exceptions are the popular BerkeleyDB
or Apache's Cassandra where in-memory operation is only an option
\cite{bdb2017doc, lakshman2010cassandra}.

\paragraph{Concurrency}

Another approach to increase transaction throughput is to utilize multi-core
processors by executing transactions concurrently. In order to achieve maximum
performance, it is common for main-memory databases to also support concurrency
\cite{grund2010hyrise, faerber2012hana, diaconu2013hekaton}. Further, it has
been shown that \ac{KVS} can gain substantial performance benefits through
concurrency \cite{fan2013memc3, li2015architecting, xu2014building}. In fact,
most \acp{KVS} natively support concurrency with the exception of Redis
\cite{redis2017home}. Unfortunately, concurrency also introduces new issues such
as inconsistencies through race conditions on shared data. Mitigating this issue
can degrade performance which is why many designs trade full consistency against
faster relaxations \cite{decandia2007dynamo}. This issue is dealt with in the
next section about transactions.

\paragraph{Distributed Databases}

As mentioned earlier, \acp{KVS} play a crucial role in big-data environments.
Since availability is often a requirement in this area, \acp{KVS} are often
implemented as distributed services \cite{decandia2007dynamo,
lakshman2010cassandra, wang2015hydradb}. Distributed databases and their
mechanisms such as distributed transactions are beyond the scope of this work.


\section{Transactions}
Transactions are a powerful concept that has been adopted in various branches of
computer science. Examples include databases, transactional memory, and
operating systems. With transactions, multiple operations, such as reading or
updating a record, can be grouped into a single unit that succeeds if and only
if neither of its operations fails. Especially in high-performance computing
environments, the utilization of computing resources through concurrent
transactions plays an essential role.

This section introduces the concept of transactions and its properties with
regard to concurrency, in particular.

\paragraph{Definition}

A transaction is a sequence of operations that is treated as a single atomic
operation, i.e. it either succeeds if all its suboperations succeed or it fails.
In general, an incomplete or failed transaction must not have any observable
side effects. A transaction \emph{commits} when all its subordinate operations
have completed. Once this process is complete, the transaction is
\emph{committed} and all its side effects, if any, become visible.

In general, the concept of a transaction does not impose any restrictions on the
kind of operation enclosed inside a transaction. That is, apart from primitive
operations such as read or update, transactions may also consist of inner
transactions as well. This concept is known as \emph{nested} transactions
\cite{gray1981transaction}. In contrast, \emph{flat} transactions only permit
primitive operations.

Despite their general nature, nested transactions are not a subject of
discussion in this thesis, for a couple of reasons. First, nesting has been
found useful primarily for distributed transaction systems and transactional
programming models neither of which are within the scope of this work
\cite{moss1981nested, moss2006open}. In addition to implementation issues, there
are several semantic models for nesting which further complicates its discussion
\cite{harder1993concurrency, weikum1992concepts}. In the end, nesting has not
found wide adoption with the prominent exception of transactional memory
\cite{moss2006nested, moravan2006supporting, jacobi2012transactional} and a few
databases \cite{olson1999berkeley}. Hence, unless stated otherwise, the term
transaction always refers to flat transactions.

Especially when discussing concurrent transactions, a meaningful notation is
required to describe their interactions. For this purpose, the concept of a
\emph{schedule} is used. A schedule is a list of operations enclosed in one or
more transactions. Within a schedule, all operations appear in the same order
they are executed. Although there seems to be no standard notation, the
operations of a schedule typically comprise abstract operations for reading and
writing a record, as well as the basic transactional primitives of commit and
abort. A transaction is implicitly started by its first operation, so there is
no need for an explicit primitive. In order to make schedules more readable,
operations are denoted by shorthands as shown in Table
\ref{tab:schedule_notation}.

\begin{figure}[h!]
    \centering
    \begin{tabular}{|c|l|}
        \hline
        \textbf{Notation} & \textbf{Operation}\\
        \hline
        r(A) & Read a record A  \\ \hline
        w(A) & Write a record A \\ \hline
        c    & Commit           \\ \hline
        a    & Abort            \\
        \hline
    \end{tabular}
    \caption{}
    \label{tab:schedule_notation}
\end{figure}

There are several ways to print a schedule. A common method is to render a
linear list containing shorthands of indexed operations where the indices denote
the associated transaction, respectively (see Figure
\ref{fig:schedule_interleaved}).

\begin{figure}[h!]
    \centering
    \[
        r_1(A)\; r_2(A)\; w_2(A)\; c_2\; w_1(A)\; c_1
    \]
    \caption{Interleaved notation of a write-write conflict.}
    \label{fig:schedule_interleaved}
\end{figure}

For the better readability, this work chooses to list operations for each transaction invidually, while retaining their global chronological order as shown in Figure \ref{fig:schedule_projected}.

\begin{figure}[h!]
    \centering
    \begin{tabular}{r c c c}
        $t_1:$ & $r(A)$ &                     & $w(A)\; c$ \\
        $t_2:$ &        & $r(A)$\; $w(A)\; c$ &            \\
    \end{tabular}
    \caption{Projected version of Figure \ref{fig:schedule_interleaved}.}
    \label{fig:schedule_projected}
\end{figure}

\paragraph{Applications}

Transactions are useful when a series of operations must either execute in
conjunction or not at all. A simple example is the transfer between bank
accounts. The action of withdrawing a value from one account and depositing it
on another comprises two separate actions that must both be successful in order
to take effect.

The predominant domain of transactions is databases where they are used to
achieve consistent and reliable data access. However, there were also attempts
to establish transactional semantics as an operating system feature
\cite{porter2009operating, spinellis2009user, black1991understanding}. This way,
subsequent system calls could be executed as a unit and be undone if one of them
fails. Another application that sees increasing attention is transactional
memory which aims to provide a synchronization alternative to locking
\cite{knight1986architecture, herlihy1993transactional, shavit1997software}.
With regard to this thesis, recent use cases include transactions in a MMDB
\cite{leis2014exploiting} and durable updates to NVRAM
\cite{volos2011mnemosyne}. Despite being an intriguing concept, transactional
memory is beyond the scope of this work. Instead this work sets its focus on
plain software-based transactions in the field of databases and KVS, in
particular.

\paragraph{Transactional Semantics}

The previous definition of transactions was of rather intuitive nature. However,
in order to be useful, the semantics of a transaction need to be described more
precisely. The predominant characterization of transactional semantics is ACID
\cite{gray1981transaction, haerder1983principles}. It comprises a set of necessary properties:

\begin{itemize}
    \item Atomicity
    \item Consistency
    \item Isolation
    \item Durability
\end{itemize}

\emph{Atomicity} captures the all-or-nothing notion of a transaction, i.e.
either all operations in its context succeed or none. As a consequence, any
already completed operation of a transaction must be undone should the latter
fail. Reverting the affected data to their previous state is often referred to
as \emph{rollback}.

The property of \emph{consistency} asserts that if the underlying data are in a
consistent state, then any transaction must preserve consistency. For example,
an ACID-compliant database cannot be transitioned into an illegal state by means
of a transaction. If  a transaction is bound to break the consistency of the
database, then it has to be aborted and rolled back.

In case multiple transactions are executed concurrently, each transaction could
observe intermediate side effects of other concurrent transactions. In order to
prevent this scenario and ensure the consistency property, \emph{isolation}
precludes transactions from seeing any concurrent activity. The property of
isolation is later dealt with in more detail.

The last of the four ACID properties is \emph{durability}. It ensures that all
side effects incurred by a committed transaction must be durable across any
subsequent system failure. Durability can be very hard to enforce, especially in
the face of catastropic failures with failing backup media. Therefore its notion
is often relaxed to a reasonable extent.

The ACID criteria have become the prevalent reference for characterizing
transactional systems. However, not all systems enforce the complete set of
properties. Notable examples are transactional memory for conventional RAM and
some cache-like databases which do not support durability as they are volatile
by design. A prominent example for relaxation that is also fundamental to this thesis is the isolation of concurrent transactions.

\paragraph{Serial Transactions}

In a serial transaction-processing system all transactions are executed in a
sequential order. That is, only one transaction, if any, is being executed at a
time and overlapping is not possible. If a transaction $t_1$ attempts to start
while another transaction $t_0$ is still active then $t_1$ has to wait until
$t_0$ terminates.

This reveals two important drawbacks. First, transaction throughput does not
scale as the number of overlapping transaction requests increases. Second, with
only one computing resource active at a time, resource utilization is low on
multi-core architectures. The same is true on single-core systems as execution
latencies cannot be hidden by switching to other transactions. To mitigate these
issues, transactions can be allowed to run concurrently.

\paragraph{Concurrent Transactions}

Concurrent transaction execution can largely remove the shortcomings outlined above. Now, an incoming transaction does not need to wait for an in-flight transaction to complete. In addition to increasing transaction throughput, it also enables better resource utilization. This works as long as data are read but not written. Allowing concurrent updates however, bears potential conflicts that threaten the consistency of data and must therefore be addressed. Possible conflicts are:

\begin{itemize}
    \item write-write
    \item write-read
    \item read-write
\end{itemize}

When a transaction $t_1$ attempts to update a record $A$ that was previously
written but not committed by another transaction $t_2$ then $t_1$'s update could
overwrite $t_2$'s update to $A$ before it has become visible. Unaware of the
condition $t_2$ will successfully commit even though its update is lost (see
Figure \ref{fig:ww-conflict}). This situation is called a \emph{write-write}
conflict.

\begin{figure}[h!]
    \centering
    \begin{tabular}{r c c c}
        $t_1:$ &        & $w(A)\; c$ &     \\
        $t_2:$ & $w(A)$ &            & $c$ \\
    \end{tabular}
    \caption{A write-write conflict resulting in a lost update for $t_2$.}
    \label{fig:ww-conflict}
\end{figure}

A \emph{write-read} conflict occurs when a transaction reads data that has not
been committed yet. Imagine a transaction $t_1$ that reads a record $A$ that was
previously updated but not committed by another transaction $t_2$. If $t_2$
updates the same item again or fails, then $t_1$ has processed a value that was
never committed. This situation is also called \emph{dirty read}. For an example see Figure \ref{fig:wr-conflict}.

\begin{figure}[h!]
    \centering
    \begin{tabular}{r c c c}
        $t_1:$ &        & $r(A)\; w(B)$ & $c$ \\
        $t_2:$ & $w(A)$ & $a$           &     \\
    \end{tabular}
    \caption{A write-read conflict resulting in an erroneous update for $t_1$.}
    \label{fig:wr-conflict}
\end{figure}

The last conflict is called \emph{read-write} conflict and denotes a situation
when a transaction updates a record that was previously read by another
transaction that is still running. Consider two transactions $t_1$, $t_2$ where
$t_1$ reads a record $A$ which is later updated by $t_2$ before either
transaction commits. If $t_1$ reads $A$ again, then the result may be
inconsistent with the earlier read as shown in Figure \ref{fig:rw-conflict}. The
situation is also referred to as \emph{inconsistent read} or
\emph{non-repeatable read}.

\begin{figure}[h!]
    \centering
    \begin{tabular}{r c c c c}
        $t_1:$ & $r(A)$ &        &     & $r(A)$ \\
        $t_2:$ &        & $w(A)$ & $c$ &        \\
    \end{tabular}
    \caption{A read-write conflict resulting in an inconsistent read for $t_1$.}
    \label{fig:rw-conflict}
\end{figure}

Note that, since isolation must be guaranteed for the entire lifetime of a
transaction, conflicts are not precluded by protecting individual read or update
operations. Instead, a dedicated synchronization mechanism for transactions is
needed. An important formalism in this regard is serializability.

\paragraph{Serializability}

A core concept to preserve consistency in the presence of concurrent
transactions is \emph{serializability}. It is based on the observation that in
an ACID-compliant serial transaction processing system, every sequence of
transactions always yields consistent data. Likewise, a schedule of concurrent
transactions should yield consistent data if it behaves in a way that is
equivalent to a serial sequence of the same transactions.

More precisely, a concurrent schedule is called \emph{serializable} if and only
if there exists a serial schedule of the same transactions that produces the
same output. A transaction processing system provides serializability if and
only if it guarantees that all its concurrent schedules are serializable.

\todo[inline]{Insert example for serializable vs. non-serializable schedules}

In order to enforce serializability, a decidable classification for serial
transaction schedules is required. An early definition of serializability
appeared in ANSI SQL \cite{berenson1995critique, melton1993understanding}. The
idea is to identify and detect \emph{anomalies} of non-serializable schedules at
runtime. If an anomaly is detected then any affected transaction must fail.
Based on whether these anomalies were permitted, several \emph{isolation levels}
were defined. In terms of ANSI SQL, a transaction is serializable if none of the
following anomalies was present:

\begin{itemize}
    \item Dirty Read
    \item Non-Repeatable Read
    \item Phantom
\end{itemize}

A \emph{phantom} is similar to a non-repeatable read but differs in that the
item in question is not modified but added or removed. Imagine a transaction
$t_1$ making a conditional selection of items. If another transaction $t_2$ adds
an item and $t_1$ repeats its selection then the result may contain the item
which is inconsistent with the first result.

Note that this formalization is built around observable artifacts of
non-serializable schedules, rather than their cause such as read-write
conflicts. While it is pragmatic to address only observable anomalies it is also
unreliable as more complicated consistencies may remain undetected. In fact, it
was later found that the above characterization is insufficient as further
anomalies were discovered \cite{berenson1995critique, fekete2004read}. In the
wake of these findings, additional restrictions were imposed on the notion of
serializability.

Nevertheless, all of the discovered anomalies can be attributed to the access
conflicts shown above. For example, the most recently discovered anomalies,
\emph{write skew} and \emph{non-serializable read-only} are essentially results
of read-write conflicts. In this sense, a transaction is serializable if and
only if all possible conflicts are precluded. This characterization has several
advantages. Most importantly, it is more plausible to discuss non-serializable
schedules in terms of causes instead of effects. In addition, as opposed to
anomalies the number of conflicts is smaller and also fixed. Therefore, this
thesis primarily defines serializability in terms of conflicts.


\section{Concurrency Control Protocols}
Concurrency is a major building block for scalable transaction processing. It
enables higher transaction throughput and resource utilization compared to
serial processing. On the downside, concurrent schedules are subject to
potential conflicts that may result in data corruption. However, it is not
sufficient to provide mutual exclusion for individual operations within a
transaction. In other words, the scope in which isolation is required spans
beyond critical sections. Therefore, a dedicated concurrency control is required
to ensure isolation. Unfortunately, concurrency controls do not come without
overhead which is why, in most cases, a compromise between isolation and
performance must be found. This section deals with concurrency control
strategies and outlines the state of the art with a focus on optimistic
approaches, in particular multiversion concurrency control.

\subsection{Strategies}

There are two fundamental approaches to the design of concurrency controls:
\emph{pessimistic} and \emph{optimistic} protocols \cite{kung1981optimistic,
larson2011high, sadoghi2014reducing}. The distinction is based on whether
conflicts are assumed to be frequent or infrequent. Still, both strategies share
their intent to prevent conflicts from manifesting into inconsistencies.

\paragraph{Pessimistic Concurrency Control}

A pessimistic concurrency control assumes that conflicts are frequent and
strives to prevent conflicts before they can even emerge. To this end,
pessimistic control mechanisms employ some form of exclusive ownership. That
means, a transaction must acquire the ownership of all data items it wishes to
access. If the resource acquisition succeeds, then the transaction can freely
operate on the temporarily owned data. Only when that transaction terminates,
will this ownership be released. If a transaction fails to claim the exclusive
ownership on its data then it has to wait until the required ownership is
granted or abort. Pessimistic concurrency control is usually implemented with
locks. Locking provides a solid mechanism to ensure serializability and most
database systems implement it \cite{kung1981optimistic, berenson1995critique,
larson2011high}.

Despite their prevalence, pessimistic concurrency controls have notable
drawbacks. Locking-based concurrency controls are prone to \emph{deadlocks}
\cite{bernstein1981concurrency, kung1981optimistic}. In order to prevent
deadlocks, they must be detected and resolved which introduces runtime overhead.
Another problem is \emph{lock contention} which occurs when a large portion of
concurrent threads in a system compete for a single shared resource
\cite{berenson1995critique, sadoghi2014reducing}. Since only one transaction can
manage to acquire ownership, all remaining transactions are left waiting for it
to complete and contend again. As a result, only one transaction is executed at
a time, thus defeating the purpose of concurrency.

\paragraph{Optimistic Concurrency Control}

Optimistic concurrency controls form the opposite of pessimistic control
schemes. Instead of preventing conflicts altogether, optimistic control schemes
do not enforce consistency until a transaction commits. Only when a transaction
commits, the concurrency control starts to check for violations, a step called
\emph{validation}. If no conflict is detected, then the transaction may commit,
otherwise, it must abort. Validation is crucial for consistency and its
implementation substantially determines the achievable isolation level
\cite{larson2011high}.

Optimistic concurrency control protocols rely on \ac{COW} and timestamp
ordering to synchronize data races of competing transactions
\cite{bernstein1981concurrency, kung1981optimistic}. While readers can access
data without further means of synchronization, writers apply their modifications
only on copies of the original data. Apart from short-duration locks for
critical sections, this approach works without locking data for the duration of
an accessing transaction. As a result, readers do not block other readers or
writers. Depending on the type of concurrency control, however, concurrent
writers may need to operate under mutual exclusion. This issue is addressed
later on. An important class of optimistic concurrency controls is multiversion
concurrency control or multiversioning \cite{reed1978naming,
bernstein1983multiversion}.

There are, however, disadvantages to this type of concurrency control. While
validation is necessary to ensure that a transaction is not involved in data
conflicts, it can also introduce a significant overhead. First, even when there
are no conflicts, validation is conducted nevertheless. Second, validation
usually requires certain metadata about the operations within a transaction. As
a result, validation complexity scales in size of metadata required to
determine conflict freedom. Another drawback is that aborting a conflicting
transaction means that its entire progress is discarded. In this case, valuable
computing resources are wasted.

Despite some drawbacks, optimistic concurrency control has found wide adoption
especially in domains where reads are dominant and conflicts are known to be
infrequent or negligible \cite{carey1986performance, larson2011high,
wu2017empirical}. The scenario of read-dominated workloads has been shown to
apply more often than not \cite{andrei2017sap, wang2017efficiently}. Given its
promising properties, optimistic concurrency control is discussed in more depth
in the subsequent sections.

\subsection{Multiversion Concurrency Control}

\ac{MVCC} or multiversioning, is an optimistic concurrency control method.
Initially, \ac{MVCC} was designed as a solution for concurrency control in
distributed systems \cite{reed1978naming}. However, it was also studied in
non-distributed settings and was soon considered a promising alternative to
locking \cite{kung1981optimistic, bernstein1983multiversion, carey1983multiple,
hadzilacos1986algorithmic, carey1986performance}. Subsequently, \ac{MVCC} has
been adopted in both commercial and non-commercial transaction processing
systems, ranging from general-purpose database systems to high-performance
in-memory databases \cite{larson2011high, lee2013high, diaconu2013hekaton,
schwalb2015efficient}. More recent examples include prototypes of \ac{MMDB} and
\ac{KVS} for \ac{NVRAM} \cite{bailey2013exploring, zhou2016nvht,
oukid2014sofort, schwalb2016hyrise}.

\subsubsection{Principle}

A \emph{version} is a snapshot of a particular data item within a database. In
terms of \ac{KVS}, that would be a value. In traditional concurrency schemes,
there is exactly one version of each item. These are also referred to as
\emph{single-version concurrency controls}. If a transaction issues an update to
a version, then it is performed in-place. In order to ensure isolation, a
transaction has to be protected against concurrent reading or writing.

In a \emph{multiversion concurrency control}, there can be multiple versions of
data items. This fundamentally changes the nature of both read and write
operations. Instead of updating versions in-place, which would have to be
isolated, write operations create copies of existing versions and only modify
those copies. As a result, read operations are implicitly decoupled from
concurrent updates. That means, in particular, that a read operation may access
an item even when newer versions have been committed. In order to keep track of
when versions were created or modified, versions are usually equipped with
timestamps, respectively.

Through its \ac{COW} approach, multiversioning can effectively isolate read
operations from concurrent operations without the need for locking. An important
implication of this circumstance is that read operations never wait for write
operations and vice versa. This is a significant advantage over single-version
schemes especially in applications where reading is much more frequent than
writing. In fact, it has been shown that many workloads are dominated by queries
\cite{krueger2011fast, andrei2017sap}. This is also reflected in the \ac{MMDB}
benchmark TATP which assumes 80 \% of all transactions to be read-only
\cite{larson2011high}. Another useful side effect of multiversioning is that it
forms an implicit logging infrastructure which can be used for recovery
purposes \cite{condit2009better, venkataraman2011consistent}. \ac{MVCC} is
similar to \ac{RCU} \cite{mckenney1998read} but differs in that it is more
general and can manage more than two versions at a time.

While read operations benefit from \ac{COW}, updates can be expensive.
Updating a version incorporates additional overhead for allocating a new version
and copying the original version before modifying it. In the case of \ac{KVS},
however, copies may not be necessary if entire values are updated.

\subsubsection{Visibility}

A central aspect of multiversioning is the concept of \emph{visibility}. Since
there may be multiple versions of a single data item, an operation must first
figure out to which version it should apply. For this purpose, a visibility
property is defined. It determines which versions can be accessed by the
operations of a transaction. In other words, a version is \emph{visible} to a
transaction and its operations if and only if it satisfies the visibility
property. In general, visibility is defined as a predicate over timestamps of
transactions and versions. The concrete definition of the predicate is subject
to the respective \ac{MVCC} protocol.

When a transaction attempts to access an item, it would typically traverse
the versions of that item and determine for each version whether it is visible
to the transaction. Only if a version is visible to the issuing transaction, it
can be selected for reading or writing. The implementation of visibility is of
paramount importance for \ac{MVCC} protocols and the desired isolation level
\cite{larson2011high}.

\subsubsection{Challenges}

The most promising feature of multiversioning is the non-blocking nature of read
operations due to the absence of in-place updates. However, there are also
important problems that need to be addressed in order to leverage the merits of
multiversion concurrency control.

First, the maintenance of more than one version per item implies a significant
overhead in storage. Note that a version may not only contain payload but
additional data such as pointers to adjacent versions. A version may also be
required to hold certain metadata such as timestamps, further increasing the
overall memory footprint. This is relevant especially in areas where memory is
comparatively scarce as is in main-memory databases. However, not all versions
need to be retained. Instead, only the versions that are visible to at least one
transaction are needed. All other versions are considered \emph{stale} and can
be disposed of. This task is usually achieved by a designated garbage collection
mechanism. Although garbage collection may improve the overall memory footprint,
it is also known to have adverse effects on performance.

Second, whenever an item is accessed, the system first has to find a visible
version of the item. Accessing an item may therefore incur a significant runtime
overhead. The overhead mainly depends on the size of the history which, in
theory, is only bounded by the amount of available memory. Employing a garbage
collection mechanism can help by reducing the size of histories. Another
optimization would be to have a transaction keep track of all the versions it
references. This way, visibility would only have to be computed once for each
item.

\subsection{Snapshot Isolation}

The most widely used \ac{MVCC} protocol to date is \ac{SI} \cite{larson2011high,
neumann2015fast}. Originally, \ac{SI} was developed as a response to the
insufficient definition of serializability in the SQL standard
\cite{berenson1995critique}. Since then, it has been deployed in numerous
databases and \acp{KVS} \cite{cahill2009serializable, wu2017empirical}. In the
context of isolation levels, \ac{SI} defines a new isolation level that goes by
the same name. Although \ac{SI} is weaker than serializability, it provides a
good trade-off between performance and consistency. This section describes the
concept of \ac{SI} and its properties.

The core principle of \ac{SI} is that a transaction $t$ only sees a private
snapshot of the database as of when $t$ started. In this sense, the notion of a
snapshot comprises the set of the latest versions that have been committed
before $t$ was invoked. The key to this behavior is the definition of the
visibility property.

\subsubsection{Visibility}

Each version $v$ stores two timestamps $begin_v$ and $end_v$, denoting when $v$
was created and when it was invalidated by an update or deletion, respectively.
The interval $[begin_v,  end_v]$ is called the \emph{lifetime} of $v$. Also,
when a transaction $t$ starts, it is given a timestamp $begin_t$ to capture when
$t$ started. In order to determine which version is visible to the operation of
a transaction, the concurrency control needs to test for each version whether
$t$ started during the lifetime of $v$. The latest valid version satisfying this
property is selected to be visible by the requesting transaction. More
precisely, the version seen by a transaction $t$ is

\[
\operatorname*{max}_{i \in \mathbb{N}}\, \{\, v_i\, |\, begin_{v_i} < begin_t < end_{v_i}\}.
\]

\subsubsection{Conflict Handling}
\label{ch:kvs-cc-conflicts}

Concurrent updates by a transaction $t_2$ that happen after a transaction $t_1$
started, are not included in the snapshot of $t_1$ and are therefore invisible
to $t_1$. If however, $t_1$ decides to also update the same data item, then a
write-write conflict emerges. In this case, the \emph{first-committer-wins}
principle is applied and $t_1$ must abort as $t_2$ also modified the same item
and committed earlier \cite{berenson1995critique}. A popular variant of this
property is the equivalent \emph{first-updater-wins} principle
\cite{fekete2004read, larson2011high}. According to this property, a writer
fails immediately if he is not the first to attempt an update on a given
version, thus making the earlier transaction the first committer. Note, that
based on this strategy, \ac{SI} can be implemented without validation on commit.
It can also reduce the size of individual rollbacks as write-write conflicts are
detected immediately.

\vfill

\subsubsection{Shortcomings}

Under \ac{SI} reads can always be satisfied, provided the requested item exists.
Note that even in the face of concurrent updates, a transaction under \ac{SI}
always sees the same items. This precludes inconsistent reads, read skew, and
phantoms. In addition, \ac{SI} does not allow dirty reads, since snapshots only
contain committed data. However, \ac{SI} does not prevent all possible anomalies
and is therefore not serializable \cite{berenson1995critique, fekete2004read}.
In particular, these conflicts are write skew and non-serializable read-only
transactions.

\paragraph{Write Skew}

The earliest known anomaly of \ac{SI} is write skew. The reason for its
occurrence is that under \ac{SI} a transaction does not see modifications to
versions that have been read during the transaction.

Imagine two transactions $t_1, t_2$ reading two data items $x, y$ constrained by
a predicate $C$. Next, $t_1$ updates $x$ and finds that $C(x^{*}, y)$ still
holds true. At this point $t_2$ is unaware that $x$ has been modified and
updates $y$. Since $t_2$ does not see the modifications of $t_1$, it also
evaluates $C(x, y^{*})$ to be true. Finally, both transactions may commit even
if $C$ is now violated because none observed the others changes (see Figure
\ref{fig:write_skew}). Note that no write-write conflict occurs as both updated
items are distinct. In fact, write skew is said to occur if read sets overlap
while write sets are distinct.

\begin{figure}[!h]
    \centering
    \begin{tabular}{r c c c c c}
        $t_1:$ & $r(x,y)$ &          & $w(x)$ &        & $c$ \\
        $t_2:$ &          & $r(x,y)$ &        & $w(y)$ & $c$ \\
    \end{tabular}
    \caption{Write skew due to transactions $t_1, t_2$ not seeing each others changes.}
    \label{fig:write_skew}
\end{figure}

In the field, write skew has been countered by inducing artificial write
conflicts between transactions that are expected to exhibit write skew
\cite{fekete2005making}.

\paragraph{Non-Serializable Read-Only Anomaly}

Another anomaly was discovered almost 10 years after the introduction of
\ac{SI}. It proved, contrary to common understanding, that even read-only
transactions may not always be serializable \cite{fekete2004read}. The proof
consists of a schedule of three transactions with one being read-only. The
schedule is constructed in a way that only two but never all three
transactions can execute without a conflict.

Suppose a pair of data items $x = 0$ and $y = 0$ and transactions $t_1, t_2$,
and a read-only transaction $t_{RO}$. Further, let $t_1$ compute $y = y - 10$
and also subtract one if $x + y < 0$, while $t_2$ sets $x = x + 20$. The
schedule given in Figure \ref{fig:bad_read_only} shows that while $t_1$ is the
first transaction to start execution, both $t_2$ and $t_{RO}$ start and commit
sequentially before $t_1$ issues its update on $y$. This means that $t_{RO}$
will see the update of $x$ by $t_2$ while $t_1$ does not. According to the
output of $t_{RO}$ ($x = 20$, $y = 0$), a serializable schedule would require
$t_1$ to have been executed after both $t_2$ and $t_{RO}$. This however, is not
possible since $t_1$ would have seen the update of $t_2$ and no penalty would
have been applied as $20 - 10 \geq 0$. Likewise, in order for $t_1$ to yield
$y = -11$, it would have had to be executed before $t_2$ (and $t_{RO}$) which,
however, is not consistent with the output of $t_{RO}$. In fact, the output of
$t_{RO}$ corresponds the exact opposite serial ordering as do those of
$t_1$ and $t_2$.

\begin{figure}[h!]
    \centering
    \begin{tabular}{r c c c c}
    $t_1:$    & $r(x,y)$ &                   &              & $w(y)\, c$ \\
    $t_2:$    &          & $r(x)\, w(x)\, c$ &              &            \\
    $t_{RO}:$ &          &                   & $r(x,y)\, c$ &
    \end{tabular}
    \caption{Transaction $t_{RO}$ is read-only but not serializable.}
    \label{fig:bad_read_only}
\end{figure}

% \todo[inline]{make intermediate values more visible..}

Both symptoms can be attributed to the fact, that \ac{SI} fails to observe
read-write conflicts. When a transaction requests an item, it reads the latest
version that has been committed before the transaction started. This way, a
transaction always reads the same version even if a newer version has been
committed concurrently. This relieves the system from locking a version when
accessing it. The downside is that every transaction is effectively isolated
from any concurrent modifications. As a consequence, transactions may
successfully commit even if one or more versions they have read has been updated
in the meantime. All anomalies known to be emitted by \ac{SI} can be reduced to
read-write conflicts.

\subsection{Serializable MVCC}

\acl{SI} provides affordable isolation but fails to preclude all
non-serializable schedules such as write skew. Nevertheless, \ac{SI} is the most
widely adopted \ac{MVCC} implementation \cite{larson2011high,
bailey2013exploring, neumann2015fast}. In some cases, serializing alternatives
are available but disabled by default. This policy is often motivated by
significant performance benefits compared to strictly serializing concurrency
control \cite{cahill2009serializable}. Others argue that \ac{SI} anomalies may
be negligible as even renowned ACID-compliant benchmarks such as TPC-C do not
expose them \cite{fekete2005making}. However, the need for data integrity should
not depend on benchmarks failing to prove it. Therefore, there are efforts to
overcome the weaknesses of \ac{SI} and provide strong consistency without
falling back to pessimistic approaches. This section presents an overview of
approaches to make serializing \ac{MVCC} affordable.

The main reason for non-serializable schedules in \ac{SI} is that it cannot
detect read-write conflicts. While \ac{SI} keeps track of each transactions'
updates for installment on commit, reads are not tracked. Therefore, a naive
approach to achieve serializable schedules with \ac{SI} is to track the read
operations of each transaction. In doing so, read-write conflicts can be
detected during validation by looking at the timestamps of all versions read. If
at least one of these versions has been invalidated after the transaction
started, then a read-write conflict has emerged. Unless the conflicting updater
is still running, the reading transaction must abort. This method is both simple
and effective but introduces significant overhead. Note that traditional \ac{SI}
does not perform any validation if first-updater-wins is used for precluding
write-write conflicts. The additional overhead especially affects read-mostly
transactions which is undesirable in read-dominated environments. Since the
latter is where \ac{SI} has been especially successful, tracking reads followed
by validation is often stated as prohibitively expensive
\cite{cahill2009serializable}. Lacking viable alternatives, research interests
primarily focus on mitigating the footprint of read tracking and validation.

\subsubsection{Exploiting Query Languages}

Several authors have proposed to detect conflicts from query language statements
\cite{fekete2005making, faleiro2015rethinking, neumann2015fast}. For instance, a
common strategy to prevent write skew in \ac{SI} is to inject detectable
conflicts whenever the required access patterns are detected
\cite{fekete2005making}. Others have found efficient ways to determine whether
an item is included in a range query, thus improving validation time
\cite{neumann2015fast}. Although intriguing, these approaches cannot be applied
to \acp{KVS} since they operate through ad hoc queries instead of query
languages.

\subsubsection{Reducing Contention}

A major challenge for high-performance implementations of \ac{MVCC} and \ac{SI},
in particular, is that important aspects such as timestamping or validation are
often centralized which can cause considerable contention. Therefore, a
substantial amount of research is dedicated to providing protocols in the spirit
of \ac{SI} but lower contention. Note that, in contrast to locking, contention
for \ac{MVCC} denotes much smaller intervals of a transaction's lifetime.

A major bottleneck in \ac{MVCC} implementations is validation
\cite{tu2013speedy, bailey2013exploring, ding2015centiman,
faleiro2015rethinking, wang2017efficiently, zhou2017posterior}. The reason is
that, validation typically requires mutual exclusion since concurrent updates to
items from the read set could falsify the validation result. As a result,
validation does not scale, thus inhibiting transaction throughput. Therefore,
many authors propose protocols featuring concurrent validation
\cite{bailey2013exploring, ding2015centiman, faleiro2015rethinking,
wang2017efficiently}. Parallel validation has already been proposed in the early
days of \ac{MVCC} \cite{kung1981optimistic}, but received renewed attention
lately.

Another point of contention is the assignment of timestamps. A typical \ac{SI}
implementation requires multiple timestamps for both versions and transactions.
However, most implementations rely on a global assignment policy which
introduces a significant contention on multi-core systems \cite{tu2013speedy,
zhou2017posterior}. First, concurrency is reduced as mutual exclusion is
required to guarantee strictly monotone timestamps. Second, the \ac{CPU} must be
informed that changes to the cached timestamp counter must be globally visible.
This is usually done with fences which can further reduce performance. In
response, some authors have proposed protocols featuring decentralized timestamp
assignment \cite{tu2013speedy, zhou2017posterior}. Further sources of contention
addressed in these works are transaction id assignment and shared memory access,
in general.


\section{Key-Value Stores for NVRAM}

\paragraph{Summary}

\begin{itemize}
    \item KVS are vital database technology; esp. distributed but beyond scope
    \item performance can be increased through in-memory operation + concurrency
    \item a widely-adopted concurrency control is MVCC; but most impl. non-ser.
    \item KVS for NVRAM have shown some potential over traditional KVS
    \item recent KVS for NVRAM all non-ser. => leverage boost to provide ser.
\end{itemize}
