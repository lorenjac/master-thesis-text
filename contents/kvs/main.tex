\chapter{Key-Value Stores}
\label{ch:kvs}

A prominent use case for NVRAM are main-memory databases. Even though main
memory and processor caches have become more affordable, MMDB still suffer from
recovery on slower disk drives \cite{oukid2015instant, schwalb2016hyrise}. NVRAM
on the other hand, provides an opportunity to eliminate recovery altogether. An
important class of databases often implemented as MMDB is KVS. Due to their
simplicity and low overhead, KVS have been adopted both in big-data computing
and database research \cite{decandia2007dynamo, lakshman2010cassandra,
wang2015hydradb}. In recent works, KVS were used to explore database design for
NVRAM \cite{bailey2013exploring, zhou2016nvht, wu2016nvmcached}.

This chapter provides a domain analysis of KVS. First, a brief overview of KVS
is given. The aim of this work is to exploit NVRAM for a KVS with fast
conflict-free concurrent transactions. Therefore, a substantial part of the
remaining chapter is dedicated to transactions and concurrency control. The
chapter is concluded with an examination of existing KVS for NVRAM.

\section{Overview}
\ac{KVS} form an integral part in modern database technology
\cite{fiebig2016one}. This section gives an overview of their properties,
classes, and applications. Compared to other types of databases, \acp{KVS} are
very simple databases that are sometimes better described by what they are not
or do not provide:

\begin{itemize}
    \item Non-relational data model
    \item No data schemas
    \item No query languages
\end{itemize}

In general, a \ac{KVS} consists of a single associative container, where each
key is mapped to exactly one value. A key is an arbitrary string with possible
restrictions on its length. In terms of relational databases, \acp{KVS} comprise
a single table of two columns. As a result, much of the structural complexity
adherent to relational \ac{DBMS} is omitted, thus making way for profound
optimization and better response times. Common data structures for associativity
in \ac{KVS} are hash tables and search trees, in particular B-trees.

Unlike traditional databases, \acp{KVS} do not impose data schemas. Consequently,
arbitrary chunks of data can be stored as values which is especially useful in
scenarios with no fixed data format or when enforcing one is not a priority.
Furthermore, \acp{KVS} do not provide query languages such as SQL to store and
retrieve data. Instead, \acp{KVS} are accessed programmatically through a concise
set of operations which is why \acp{KVS} are also referred to as \emph{embedded}
databases. Although their \ac{API} is not standardized, it can be essentially
broken down to the following operations:

\begin{itemize}
    \item Open/Close
    \item Insertion
    \item Retrieval
    \item Removal
\end{itemize}

\paragraph{Applications}

Traditional \acp{DBMS} are often based on complex architectures featuring query
front ends and sophisticated storage mechanisms. While this works well in many
cases, it severely limits the performance in situations where a simpler storage
paradigm (e.g. key-value pairs) is sufficient. As a consequence, high access
latencies and convoluted, error-prone concurrency schemes inhibit the
scalability of storage systems. \acp{KVS} on the other hand are designed to
compensate for these shortcomings. A driving force in this regard, are large
internet platforms, e-commerce for instance, and cloud computing services.

A longstanding example of a \ac{KVS} is BerkeleyDB which acts as a database in a
variety of software solutions. Apart from open-source software such as OpenLDAP
or Apache Web server, BerkeleyDB is also used in a number of proprietary
software such as messaging servers, switches, and routers
\cite{kaestner2007aspect, olson1999berkeley}.

A more recent use case are distributed in-memory caches often found in big-data
environments. Web caches have received great attention as service providers
struggle to scale with rising traffic where many requests target only a small
amount of data \cite{xu2014characterizing}. With caching, a dedicated eviction
policy ensures that \emph{relevant} items reside in memory. As a result, caching
can improve response times significantly. For this purpose, \acp{KVS} provide an
appropriate abstraction. Important representatives of this class are Redis and
memcached \cite{redis2017home, memcached2017home}. Not only have these \acp{KVS}
been deployed at companies such as Facebook or Twitter, but they have also
formed the basis for considerable amounts of research in this area
\cite{xu2014characterizing}. Examples include \ac{FPGA} acceleration
\cite{lavasani2014fpga}, memory partitioning for better cache hit rates
\cite{carra2014memory}, and \ac{NVRAM} integration \cite{wu2016nvmcached,
malinowski2017using, venkataraman2011consistent}. Still, large companies tend to
maintain in-house solutions to suit their needs \cite{chang2008bigtable,
decandia2007dynamo, lakshman2010cassandra, wang2015hydradb}.

Beyond databases and caches, \acp{KVS} have also been proposed as a basis for
file systems. In the past, there have been several attempts to integrate
database concepts into file systems, some of which are logging
\cite{rosenblum1992design, tweedie1998journaling} and transactions
\cite{seltzer1990transaction, wright2007extending, spillane2009enabling}. Some
studies even suggest that traditional hierarchical file systems may often be
suboptimal \cite{stein2005stupid, seltzer2009hierarchical}. While databases in
general are still considered too heavy-weight for use in file systems
\cite{seltzer2009hierarchical}, \acp{KVS} may be a viable alternative. Examples
include the network file system DBFS which is based on BerkeleyDB
\cite{murphy2002design} and FlatFS, a simple file system for \ac{NVRAM}
\cite{volos2014aerie}. In addition, \acp{KVS} are also used to complement file
systems, for example, to store metadata as in PVFS \cite{carns2009small}. Still,
the predominant use case of \ac{KVS} is found in light-weight databases and
caches on top of existing file systems.

\paragraph{Transactions}

An essential feature of most databases are transactions. Transactions enable a
sequence of database operations to appear as a single atomic operation. If a
single operation involved in a transaction fails, the entire transaction fails
and its side effects are rolled back.

Transactions are a powerful mechanism that enables aggregated operations without
worrying about inconsistencies even in case of failure. Given the prevalence of
transactions, most \acp{KVS} support them. Due to their importance for this work,
transactions are covered in more detail in the next section.

\todo[inline]{Which KVS do not support transactions? -> memcached}

\paragraph{In-Memory Operation}

The performance of a database is often denoted in terms of transaction
throughput. One way to increase throughput is to mitigate data access latencies.
Apart from faster storage, this can be done by placing the entire database in
main memory which enables speedups by multiple orders of magnitude. This
approach, which dates back to the mid 1980s, has been adopted in many
high-performance databases such as the more recent HANA database
\cite{molina1992main, faerber2012hana}. Likewise, most \acp{KVS} are explicitly
designed for in-memory operation. Notable exceptions are the popular BerkeleyDB
or Apache's Cassandra where in-memory operation is only an option
\cite{bdb2017doc, lakshman2010cassandra}.

\paragraph{Concurrency}

Another approach to increase transaction throughput is to utilize multi-core
processors by executing transactions concurrently. In order to achieve maximum
performance, it is common for main-memory databases to also support concurrency
\cite{grund2010hyrise, faerber2012hana, diaconu2013hekaton}. Further, it has
been shown that \ac{KVS} can gain substantial performance benefits through
concurrency \cite{fan2013memc3, li2015architecting, xu2014building}. In fact,
most \acp{KVS} natively support concurrency with the exception of Redis
\cite{redis2017home}. Unfortunately, concurrency also introduces new issues such
as inconsistencies through race conditions on shared data. Mitigating this issue
can degrade performance which is why many designs trade full consistency against
faster relaxations \cite{decandia2007dynamo}. This issue is dealt with in the
next section about transactions.

\paragraph{Distributed Databases}

As mentioned earlier, \acp{KVS} play a crucial role in big-data environments.
Since availability is often a requirement in this area, \acp{KVS} are often
implemented as distributed services \cite{decandia2007dynamo,
lakshman2010cassandra, wang2015hydradb}. Distributed databases and their
mechanisms such as distributed transactions are beyond the scope of this work.


\section{Transactions}
Transactions are a powerful concept that has been adopted in various branches of
computer science. Examples include databases, transactional memory, and
operating systems. With transactions, multiple operations, such as reading or
updating a record, can be grouped into a single unit that succeeds if and only
if neither of its operations fails. Especially in high-performance computing
environments, the utilization of computing resources through concurrent
transactions plays an essential role.

This section introduces the concept of transactions and its properties with
regard to concurrency, in particular.

\paragraph{Definition}

A transaction is a sequence of operations that is treated as single atomic
operation, i.e. it either succeeds if all its suboperations succeed or it fails.
In general, an incomplete or failed transaction must not have any observable
side effects. A transaction \emph{commits} when all its subordinate operations
have completed. Once this process is complete, the transaction is
\emph{committed} and all its side effects, if any, become visible.

In general, the concept of a transaction does not impose any restrictions on the
kind of operation enclosed inside a transaction. That is, apart from primitive
operations such as read or update, transactions may also consist of inner
transactions as well. This concept is known as \emph{nested} transactions
\cite{gray1981transaction}. In contrast, \emph{flat} transactions only permit
primitive operations.

Despite their general nature, nested transactions are not a subject of
discussion in this thesis, for a couple of reasons. First, nesting has been
found useful primarily for distributed transaction systems and transactional
programming models neither of which are within the scope of this work
\cite{moss2006open}. In addition to implementation issues, there are several
semantic models for nesting which further complicates its discussion
\cite{harder1993concurrency, weikum1992concepts}. In the end, nesting has not
found wide adoption with the prominent exception of transactional memory
\cite{moss2006nested, moravan2006supporting, saha2006mcrt,
jacobi2012transactional} and a few databases \cite{olson1999berkeley}. Hence,
unless stated otherwise, the term transaction always refers to flat
transactions.

Transactions are useful when a series of operations must either execute in
conjunction or not at all. A simple example is the transfer between bank
accounts. The action of withdrawing a value from one account and depositing it
on another comprises two separate actions that must both be successful in order
to take effect.

% Still, as shown later, even single operations can benefit from
% transactional semantics.

\todo[inline]{Define \textbf{schedule} somewhere}

\paragraph{Transactional Semantics}

The previous definition of transactions was of rather intuitive nature. However,
in order to be useful the semantics of a transaction need to be described more
precisely. The predominant characterization of transactional semantics is ACID
\cite{gray1981transaction, haerder1983principles}. It comprises a set of necessary properties:

\begin{itemize}
    \item atomicity
    \item consistency
    \item isolation
    \item durability
\end{itemize}

Atomicity captures the all-or-nothing notion of a transaction, i.e. either all
operations in its context succeed or none. As a consequence, any already
completed operation of a transaction must be undone should the latter fail.
Reverting the affected data to their previous state is often referred to as
\emph{rollback}.

The property of consistency asserts that if the underlying data are in a
consistent state, then any transaction must preserve consistency. For example,
an ACID-compliant database cannot be transitioned into an illegal state by means
of a transaction. If  a transaction is bound to break the consistency of the
database, then it has to be aborted and rolled back.

In case multiple transactions are executed concurrently, each transaction could
observe intermediate side effects of other concurrent transactions. In order to
prevent this scenario and ensure the consistency property, isolation precludes
transactions from seeing any concurrent activity. The property of isolation is
later dealt with in more detail.

The last of the four ACID properties is durability. It ensures that all side
effects incurred by a committed transaction must be durable across any
subsequent system failure. Durability can be very hard to enforce, especially in
the face of catastropic failures with failing backup media. Therefore its notion
is often relaxed to a reasonable extent.

The ACID criteria have become the prevalent reference for characterizing
transactional systems. However, not all systems enforce the complete set of
properties. Notable examples are transactional memory for conventional RAM and
some cache-like databases which do not support durability as they are volatile
by design. A prominent example for relaxation that is also fundamental to this thesis is the isolation of concurrent transactions.

\paragraph{Serial Transactions}

In a serial transaction-processing system all transactions are executed in a
sequential order. That is, only one transaction, if any, is being executed at a
time and overlapping is not possible. If a transaction $t_1$ attempts to start
while another transaction $t_0$ is still active then $t_1$ has to wait until
$t_0$ terminates.

This reveals two important drawbacks. First, transaction throughput does not
scale as the number of overlapping transaction requests increases. Second, with
only one computing resource active at a time, resource utilization is low on
multi-core architectures. The same is true on single-core systems as execution
latencies cannot be hidden by switching to other transactions. To mitigate these
issues, transactions can be allowed to run concurrently.

\paragraph{Concurrent Transactions}

Concurrent transaction execution can largely remove the shortcomings outlined above. Now, an incoming transaction does not need to wait for an in-flight transaction to complete. In addition to increasing transaction throughput it also enables better resource utilization. This works as long as data are read but not written. Allowing concurrent updates however, bears potential conflicts that threaten the consistency of data and must therefore be addressed. Possible conflicts are:

\begin{itemize}
    \item write-write
    \item write-read
    \item read-write
\end{itemize}

When a transaction $t_1$ attempts to update a record $A$ that was previously
written but not committed by another transaction $t_2$ then $t_1$'s update could
overwrite $t_2$'s update to $A$ before it has become visible. Unaware of the
condition $t_2$ will successfully commit even though its update is lost. This
situation is called a \emph{write-write} conflict.

\begin{lstlisting}
t1: -------w(A)-commit---------
t2: --w(A)---- ... ----commit-- (update is lost)
\end{lstlisting}

In a \emph{write-read} conflict occurs when a transaction reads data that has
not been committed yet. Imagine a transaction $t_1$ that reads a record $A$ that
was previously updated but not committed by another transaction $t_2$. If $t_2$
updates the same item again or fails, then $t_1$ has read a value that was never
committed. This situation is also called \emph{dirty read}.

\begin{lstlisting}
t1: -------r(A)-w(B)-commit-- (B is inconsistent to committed A)
t2: --w(A)--- !!! ----------- (update to A was rolled back)
\end{lstlisting}

The last conflict is called \emph{read-write} conflict and denotes a situation
when a transaction updates a record that was previously read by another
transaction that is still running. Consider two transactions $t_1$, $t_2$ where
$t_1$ reads a record $A$ which is later updated by $t_2$ before either
transaction commits. If $t_1$ reads $A$ again, then the result may be
inconsistent with the earlier read. The situation is also referred to as
\emph{inconsistent read} or \emph{non-repeatable read}.

\begin{lstlisting}
t1: --r(A)--- ... ---r(A)--------- (has read inc. values of A)
t2: ----------w(A)--------commit--
\end{lstlisting}

It is important to note that the conflicts explained above are not precluded by protecting individual read or update operations.

\todo[inline]{Explain consistency in transactional sense vs. race conditions}

With regard to data integrity it is imperative to ensure isolation by preventing these conflicts.

\paragraph{Serializability}

A core concept to preserve consistency in the presence of concurrent
transactions is \emph{serializability}. It is based on the observation that in
an ACID-compliant serial transaction processing system, every sequence of
transactions always yields consistent data. Likewise, a schedule of concurrent
transactions should yield consistent data if and only if it behaves in a way
that is equivalent to a serial sequence of the same transactions. More
precisely, if and only if a concurrent schedule produces the same output as a
serial schedule would have then there are no inconsistencies. This leads to the
formal definition of serializability.

A concurrent schedule is called \emph{serializable} if and only if there exists
a serial schedule of the same transactions that produces the same output. A
transaction processing system provides serializability if and only if it
guarantees that all concurrent schedules are serializable.

\todo[inline]{Examples}

In order to enforce serializability, a decidable classification for serial
transaction schedules is required. A naive approach would be to search for
equivalent serial schedules whenever needed. However, the complexity of
searching for such a schedule grows exponentially, thus making this approach
infeasible. Therefore, more pragmatic approaches were designed.

An early definition of serializability was given in ANSI SQL
\cite{berenson1995critique}. The idea is to identify and detect \emph{anomalies}
of non-serializable schedules at runtime. If an anomaly is detected then any
affected transaction must fail. Based on whether these anomalies were permitted,
several \emph{isolation levels} were defined. In terms of ANSI SQL, a
transaction was serializable if none of the following anomalies was present:

\begin{itemize}
    \item Dirty Read
    \item Non-Repeatable Read
    \item Phantom
\end{itemize}

A \emph{phantom} is similar to a non-repeatable read but differs in that the
item in question is not modified but added or removed. Imagine a transaction
$t_1$ making a conditional selection of items. If another transaction $t_2$ adds
an item and $t_1$ repeats its selection then the result may contain the item
which is inconsistent with the first result.

Note that this formalization is built around the observable artifacts of
non-serializable schedules, rather than their cause such as read-write
conflicts. While it is pragmatic to address only observable anomalies it also
unreliable as more complicated consistencies may remain undetected. In fact, it
was later found that the above characterization is insufficient as further
anomalies were discovered \cite{berenson1995critique, fekete2004read}. In the
wake of these findings, additional restrictions were imposed on the notion of
serializability.

Nevertheless, all of the discovered anomalies can be attributed to the access
conflicts shown above. For example, the most recently discovered anomalies,
\emph{write skew} and \emph{non-serializable read-only} are essentially results
of read-write conflicts. In this sense, a transaction is serializable if and
only if all possible conflicts are precluded. This characterization has several
advantages. Most importantly, it is more plausible to discuss non-serializable
schedules in terms of causes instead of effects. In addition, as opposed to
anomalies the number of conflicts is smaller and also fixed. Therefore, this
thesis primariliy defines serializability in terms of conflicts.

\todo[inline]{Stress importance of isolation/serializability}

\paragraph{Transaction Models}

\todo[inline]{distinguish software tx vs. transactional memory}


\section{Concurrency Control Protocols}
Concurrency is a major building block for scalable transaction processing. It
enables higher transaction throughput and resource utilization compared to
serial processing. On the downside, concurrent schedules are subject to
potential conflicts that may result in data corruption. However, it is not
sufficient to provide mutual exclusion for individual operations within a
transaction. In other words, the scope in which isolation is required spans
beyond critical sections. Therefore, a dedicated concurrency control is required
to ensure isolation. Unfortunately, concurrency controls do not come without overhead which is why, in most cases, a compromise between isolation and performance must be found.

This section deals with concurrency control strategies and outlines the state of
the art with a focus on optimistic approaches, in particular multiversion
concurrency control.

\subsection{Strategies}

There are two fundamental approaches to the design of concurrency controls:
\emph{pessimistic} and \emph{optimistic} protocols \cite{kung1981optimistic,
larson2011high, sadoghi2014reducing}. The distinction is based on whether
conflicts are assumed to be frequent or infrequent. Still, both strategies share
their intent to prevent conflicts from manifesting into inconsistencies.

\paragraph{Pessimistic Concurrency Control}

A pessimistic concurrency control assumes that conflicts are frequent and
strives to prevent conflicts before they can even emerge. To this end,
pessimistic control mechanisms employ some form of exclusive ownership. That
means, a transaction must acquire the ownership of all data items it wishes to
access. If the resource acquisition succeeds, then the transaction can freely
operate on the temporarily owned data. Only when that transaction terminates,
will this ownership be released. If a transaction fails to claim the exclusive
ownership on its data then it has to wait until the required ownership is
granted or abort.

Pessimistic concurrency control is usually implemented with locks. Locking
provides a solid mechanism to ensure serializability and most database systems
implement it \cite{kung1981optimistic, berenson1995critique, larson2011high}.

Despite their prevalence, pessimistic concurrency controls have notable
drawbacks. Locking-based concurrency controls are prone to \emph{deadlocks}
\cite{bernstein1981concurrency, kung1981optimistic}. In order to prevent
deadlocks, they must be detected and resolved which introduces runtime overhead.
Unfortunately, there is no general-purpose locking protocol that precludes
deadlocks \cite{kung1981optimistic}. Another problem is \emph{lock contention}
which occurs when a large portion of concurrent threads in a system compete for
a single shared resource \cite{berenson1995critique, sadoghi2014reducing}. Since
only one transaction can manage to acquire ownership, all remaining transactions
are left waiting for it to complete and contend again. As a result, only one
transaction is executed at a time, thus defeating the purpose of concurrency.

\paragraph{Optimistic Concurrency Control}

Optimistic concurrency controls form the opposite of pessimistic control
schemes. Instead of preventing conflicts altogether, optimistic control schemes
do not enforce consistency until a transaction commits. Only when a transaction
commits, the concurrency control starts to check for violations, a step called
\emph{validation}. If no conflict is detected, then the transaction may
commit, otherwise, it must abort.

Optimistic concurrency control protocols rely on copy-on-write (CoW) and
timestamp ordering to synchronize data races of competing transactions
\cite{bernstein1981concurrency, kung1981optimistic}. While readers can access
data without further means of synchronization, writers apply their modifications
only on copies of the original data. Apart from short-duration locks for
critical sections, this approach does not require locking and is therefore not
subject to deadlocks or lock contention. A signature property is that readers
never block which means that both readers and writers never wait for other
readers. Due to its nature, optimistic concurrency control is also called
multiversion concurrency control or multiversioning
\cite{bernstein1983multiversion}.

There are, however, disadvantages to this type of concurrency control. While
validation is necessary to ensure that a transaction is not involved in data
conflicts, it can also introduce a significant overhead. First, even when there
are no conflicts, validation is conducted nevertheless. Second, validation
usually requires certain meta data about the operations within a transaction. As
a result, validation complexity scales in size of meta data required to
determine conflict freedom. Another drawback is that aborting a conflicting
transaction means that its entire progress is discarded. In this case, valuable
computing resources have been wasted. This is especially true for update
operations as they involve potentially expensive copy operations.

Despite some drawbacks, optimistic concurrency control has found wide adoption
especially in domains where reads are dominant and conflicts are known to be
infrequent or negligible \cite{carey1986performance, larson2011high,
wu2017empirical}. The scenario of read-dominated workloads has been shown to
apply more often than not \cite{krueger2011fast, andrei2017sap}. Given its
promising properties, optimistic concurrency control is discussed in more depth
in the subsequent sections.

\subsection{Multiversion Concurrency Control}

Multiversioning is a popular concurrency control mechanism. While originating in
distributed systems research \cite{reed1978naming}, it was soon considered a
promising alternative to traditional locking-based approaches
\cite{kung1981optimistic, bernstein1983multiversion, carey1983multiple,
hadzilacos1986algorithmic, carey1986performance}.

Today, MVCC forms the
foundation of many commercial products \cite{larson2011high}. It is also featured in main-memory databases
such as SAP HANA and recent research in non-volatile main memory databases
\cite{lee2013high, schwalb2015efficient, schwalb2016hyrise, oukid2014sofort}.
However, most implementations of MVCC, such as snapshot isolation, do not
guarantee full serializability and are thus prone to inconsistencies
\cite{neumann2015fast, berenson1995critique}. Although recent research has shown
ways to achieve serializability with MVCC, implementations are hesitant to
follow \cite{larson2011high}.

MVCC has not only been applied in traditional DBMS but also in main-memory
databases. With emerging non-volatile memory that is both affordable and little
slower than conventional DRAM, new problems for MVCC such as durability,
recovery or access time arise \cite{bailey2011operating, larson2011high,
oukid2014sofort, schwalb2016hyrise}. In order to fully leverage the emerging
memory technology, it is important to assess opportunities and issues of MVCC
especially when serializability is required.

Multiversioning, which is also referred to as multiversion concurrency control
(MVCC), is a popular concurrency control method. It has been implemented in both
commercial and non-commercial transaction-processing systems, ranging from
general-purpose database systems to high-performance in-memory databases.
Initially designed as a solution to concurrency control for distributed and
nested transactions \cite{reed1978naming}, MVCC has been widely adopted as an
alternative to conventional locking-based concurrency schemes. This move was
motivated in overcoming typical problems associated with locking, such as
deadlocks and lock contention.

A \emph{version} is a snapshot of a particular data item within a database. In
traditional concurrency schemes, there is exactly one version of each item.
These are also referred to as \emph{single-version} concurrency schemes. If a
transaction issues an update to this version, then it is performed in-place. In
order to ensure isolation though, a transaction has to be protected against
concurrent reading or writing. This is usually achieved by having each
transaction acquire locks on its data which are only released once the
transaction terminates. This approach is very effective in that it can provide
the highest isolation level of serializability as in DB2 or MySQL
\cite{berenson1995critique}. However, it also has some notable drawbacks such as
deadlocks and lock contention.

In a \emph{multiversion} concurrency scheme, multiple versions of an item can be
maintained. This fundamentally changes the nature of both read and write
operations. Instead of updating an item in-place which would have to be
isolated, write operations create new versions by modifying copies of existing
versions. Also, since multiple versions are kept, read operations do not need to
access the latest version and may instead select an earlier version. This means
that not only may a transaction read an item which is currently being updated
but it is also able to continue accessing the same item even if there exists a
newer version.

This scheme can effectively isolate read operations from concurrent operations
without the need for locking. An important implication of this property is that
read operations never wait for write operations and vice versa. This is a
significant advantage over single-version schemes especially in applications
where reading is much more frequent than writing (e.g. OLAP). In fact, it has
been shown that even in OLTP systems \emph{typical} workloads are dominated by
queries \cite{krueger2011fast, andrei2017sap}. This is also reflected in the
TATP benchmark for OLTP systems which assumes 80 \% of all transactions to be
read-only \cite{larson2011high, neumann2015fast, oukid2015instant}.

This is different with write operations. Updating a version incorporates
additional overhead for allocating a new version and copying the original
version before modifying it. Also concurrent writes to an item may need to be
addressed to prevent write-write conflicts.

On the other hand, the presence of multiple versions also implies that whenever
a transaction issues a operation on a data item, it first has to determine which
version the operation applies to. This is done by traversing the version
\emph{history} of the item and testing for each version its \emph{visibility}. A
version is said to be \emph{visible} if and only if it satisfies a well-defined
predicate. This predicate can be formulated over operations or transactions.
Only if a version is visible it can be selected for reading or writing,
respectively. The implementation of the selection method depends both on the
type of concurrency control, that is optimistic or pessimistic, and the desired
isolation level.

The most promising feature of multiversioning is the non-blocking nature of read
operations due to the absence of in-place updates. However, there also important
problems that need to be addressed in order to leverage the merits of
multiversion concurrency control.

First, the maintenance of multiple versions per item implies a significant
overhead in storage. Note that a version may not only contain payload but
additional data such as handles pointing to adjacent versions. A version may
also be required to hold certain metadata such as timestamps, further increasing
the overall memory footprint. This may be relevant especially in areas where
memory is comparatively scarce as is in main-memory databases. However, not all
versions need to be retained. Instead, only the versions that are visible to at
least one transaction are needed. All other versions can be considered
\emph{stale} and be disposed. This task is usually achieved by a designated
garbage collection mechanism. Although garbage collection may improve the
overall memory footprint it is also known to have adverse effects on
performance. Another drawback is that in the presence of many long-running
transactions, versions are less likely to go stale and cannot be released.

Second, whenever an item is accessed, the system first has to find a visible
version of the item. Accessing an item may therefore incur a significant runtime
overhead. The overhead mainly depends on the size of the history and is thus
bounded by the longest available history. Employing a garbage collection
mechanism can reduce the size of version histories thus improving the upper
bound of the visibility test. Another optimization would be to have a
transaction keep track of all the versions it references. This way, visibility
would only have to be computed once for each item.

\subsection{Snapshot Isolation}

Snapshot Isolation \cite{berenson1995critique} is a popular MVCC algorithm that
has been deployed in numerous databases such as the ORACLE RDBMS, Microsoft SQL
Server, PostgreSQL and BerkeleyDB \cite{cahill2009serializable}. It is based on
timestamping and does not require locking. This section introduces the concept
of Snapshot Isolation and its properties.

The core princple of snapshot isolation is that a transaction $T$ only sees a
private snapshot of the database as of when $T$ started. In this sense, the
notion of a snapshot comprises the set of the latest versions of each data item
that have been committed before $T$ was invoked.

Concurrent updates by a transaction $t_2$ that happen after a transaction $t_1$
started, are not included in the snapshot of $t_1$ and are thus invisible to
$t_1$. If however, $t_1$ decides to also update the same data item, then a
write-write conflict emerges. In this case, the \textit{first-committer-wins}
principle is applied and $t_1$ must abort as $t_2$ also modified the same item
and committed earlier. A popular variant of this property is the equivalent
\textit{first-updater-wins} principle \cite{fekete2004read, larson2011high}.
According to this property, a writer fails immediately if it is not the first to
update a given version, thereby making the updating transaction the first
committer.

Each version keeps a \textit{begin timestamp} and an \textit{end timestamp},
that store the point in time when it was created and when it was invalidated by
deletion or an update. Also, when starting, every transaction is given a
timestamp denoting its start time. In order to determine which version is
visible to the operation of a transaction, the concurrency control tests for
each version whether the transaction begin falls between the begin and the end
timestamp of the version. The latest valid version satisfying this property is
selected to be visible by the requesting transaction.

Under snapshot isolation reads can always be satisfied, provided the requested
item exists. Note that even in the face of concurrent updates, a transaction
always reads the same data items. This precludes inconsistent reads, read skew
and phantoms. In addition, snapshot isolation does not allow dirty reads, since
snapshots only contain committed data. However, snapshot isolation does not
prevent all possible anomalies and is therefore not serializable
\cite{berenson1995critique}. These anomalies, namely write skew and another
similar anomaly will be discussed below.

\paragraph{Write Skew}

A classic example in this regard is write skew. The reason for its occurrence is
that under snapshot isolation a transaction does not see modifications to
versions that have been read during the transaction.

Imagine two transactions $t_1, t_2$ reading two data items $x, y$ constrained by
a predicate $C$. Next, $t_1$ updates $x$ and finds that $C(x^{*}, y)$ still
holds true. At this point $t_2$ is unaware that $x$ has been modified and
updates $y$. Since $t_2$ does not see the modifications of $t_1$, it also
evaluates $C(x, y^{*})$ to be true. Finally, both transactions may commit even
if $C$ is now violated because none observed the others changes (see figure
\ref{fig:write_skew}). Note that no write-write conflict occurs as both updated
items are distinct. In fact, write skew is said to occur if read sets overlap
while write sets are distinct.

\begin{figure}
    \centering
    \[
        r_1[x,y]\; r_2[x,y]\; w_1[x]\; w_2[y]\; c_1\; c_2\;
    \]
    \caption{Write skew due to transactions $t_1, t_2$ not seeing each others changes.}
    \label{fig:write_skew}
\end{figure}

In the field, write skew has been countered by inducing artificial write
conflicts between transactions that are expected to exhibit write skew
\cite{fekete2005making}.

\paragraph{Non-Serializable Read-Only}

Another anomaly was discovered almost 10 years after the introduction of
snapshot isolation. It proved, contrary to common understanding, that even
read-only transactions may not always be serializable \cite{fekete2004read}. The
proof consists of a schedule of three transactions with one being read-only. The
schedule is constructed in a way that at most two but never all three
transactions can execute serializably.

Suppose a pair of data items $x = 0$ and $y = 0$ and transactions $t_1, t_2,
t_{RO}$. Further, let $t_1$ compute $y = y - 10$ and also subtract one if $x + y <
0$, while $t_2$ sets $x = x + 20$. The schedule given in figure
\ref{fig:bad_read_only} shows that while $t_1$ is the first transaction to start
execution, both $t_2$ and $t_{RO}$ start and commit sequentially before $t_1$
issues its update on $y$. This means that $t_{RO}$ will see the update of $x$ by
$t_2$ while $t_1$ does not. According to the output of $t_{RO}$ ($x = 20$, $y =
0$), a serializable schedule would require $t_1$ to have been executed after
both $t_2$ and $t_{RO}$. This however, is not possible since $t_1$ would have
seen the update of $t_2$ and no penalty would have been applied as $20 - 10 \geq
0$. Likewise, in order for $t_1$ to yield $y = -11$, it would have had to be
executed before $t_2$ (and $t_{RO}$) which however is not consistent with the
output of $t_{RO}$. In fact, the output of $t_{RO}$ corresponds the the exact
opposite serial ordering as do those of $t_1$ and $t_2$.

\begin{figure}[h!]
    \centering
    \[
        r_1[x,y]\; r_2[x]\; w_2[x]\; c_2\; r_3[x,y]\; c_3\; w_2[y]\; c_2\;
    \]
    \caption{Transaction $t_3$ is read-only but not serializable.}
    \label{fig:bad_read_only}
\end{figure}

Both symptoms can be related to the fact, that snapshot isolation fails to
observe read-write conflicts. When a transaction requests an item, it reads the
latest version that has been committed before the transaction started. This way,
a transaction always reads the same version even if a newer version has been
committed concurrently. The downside is that every transaction is effectively
isolated from any concurrent modifications. As a consequence, transactions may
successfully commit even if one or more versions they have read has been updated
in the meantime. All anomalies known to be emitted by snapshot isolation can be
reduced to read-write conflicts.

\subsection{Serializable MVCC}

The previous chapter introduced the renowned MVCC algorithm of snapshot
isolation. Even though snapshot isolation permits non-serializable schedules
that can lead to inconsistent data, it is still and to this date the most widely
adopted MVCC algorithm \cite{cahill2009serializable, larson2011high,
sikka2012efficient, neumann2015fast}. Notable anomalies resulting from
non-serializable schedules are write skew and non-serializable read-only
transactions. In fact, most systems do not provide serializable isolation
degrees by default, even if supported. This is often motivated by significantly
improved performance. Others argue that anomalies may be negligible as even the
renowned ACID-compliant TPC-C benchmark does not exhibit them
\cite{fekete2005making}.

In order to still achieve serializable multiversion concurrency controls some
advanced methods have been proposed \cite{fekete2005making,
cahill2009serializable, neumann2015fast}. The examined approaches can be broken
down as follows:

\begin{itemize}
    \item Keep Snapshot Isolation but modify transactional database programs
    \item Extend Snapshot Isolation with read-write conflict detection
    \item Replace Snapshot Isolation with a custom implementation
\end{itemize}


\section{Key-Value Stores for NVRAM}

\paragraph{Summary}

\begin{itemize}
    \item KVS are vital database technology; esp. distributed but beyond scope
    \item performance can be increased through in-memory operation + concurrency
    \item a widely-adopted concurrency control is MVCC; but most impl. non-ser.
    \item KVS for NVRAM have shown some potential over traditional KVS
    \item recent KVS for NVRAM all non-ser. => leverage boost to provide ser.
\end{itemize}
