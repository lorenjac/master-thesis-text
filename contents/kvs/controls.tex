Concurrency is a major building block for scalable transaction processing. It
enables higher transaction throughput and resource utilization compared to
serial processing. On the downside, concurrent schedules are subject to
potential conflicts that may result in data corruption. However, it is not
sufficient to provide mutual exclusion for individual operations within a
transaction. In other words, the scope in which isolation is required spans
beyond critical sections. Therefore, a dedicated concurrency control is required
to ensure isolation. Unfortunately, concurrency controls do not come without
overhead which is why, in most cases, a compromise between isolation and
performance must be found.

This section deals with concurrency control strategies and outlines the state of
the art with a focus on optimistic approaches, in particular multiversion
concurrency control.

\subsection{Strategies}

There are two fundamental approaches to the design of concurrency controls:
\emph{pessimistic} and \emph{optimistic} protocols \cite{kung1981optimistic,
larson2011high, sadoghi2014reducing}. The distinction is based on whether
conflicts are assumed to be frequent or infrequent. Still, both strategies share
their intent to prevent conflicts from manifesting into inconsistencies.

\paragraph{Pessimistic Concurrency Control}

A pessimistic concurrency control assumes that conflicts are frequent and
strives to prevent conflicts before they can even emerge. To this end,
pessimistic control mechanisms employ some form of exclusive ownership. That
means, a transaction must acquire the ownership of all data items it wishes to
access. If the resource acquisition succeeds, then the transaction can freely
operate on the temporarily owned data. Only when that transaction terminates,
will this ownership be released. If a transaction fails to claim the exclusive
ownership on its data then it has to wait until the required ownership is
granted or abort.

Pessimistic concurrency control is usually implemented with locks. Locking
provides a solid mechanism to ensure serializability and most database systems
implement it \cite{kung1981optimistic, berenson1995critique, larson2011high}.

Despite their prevalence, pessimistic concurrency controls have notable
drawbacks. Locking-based concurrency controls are prone to \emph{deadlocks}
\cite{bernstein1981concurrency, kung1981optimistic}. In order to prevent
deadlocks, they must be detected and resolved which introduces runtime overhead.
Another problem is \emph{lock contention} which occurs when a large portion of
concurrent threads in a system compete for a single shared resource
\cite{berenson1995critique, sadoghi2014reducing}. Since only one transaction can
manage to acquire ownership, all remaining transactions are left waiting for it
to complete and contend again. As a result, only one transaction is executed at
a time, thus defeating the purpose of concurrency.

\paragraph{Optimistic Concurrency Control}

Optimistic concurrency controls form the opposite of pessimistic control
schemes. Instead of preventing conflicts altogether, optimistic control schemes
do not enforce consistency until a transaction commits. Only when a transaction
commits, the concurrency control starts to check for violations, a step called
\emph{validation}. If no conflict is detected, then the transaction may commit,
otherwise, it must abort. Validation is crucial for consistency and its
implementation substantially determines the achievable isolation level
\cite{larson2011high}.

Optimistic concurrency control protocols rely on \ac{COW} and timestamp
ordering to synchronize data races of competing transactions
\cite{bernstein1981concurrency, kung1981optimistic}. While readers can access
data without further means of synchronization, writers apply their modifications
only on copies of the original data. Apart from short-duration locks for
critical sections, this approach works without locking data for the duration of
an accessing transaction. As a result, readers do not block other readers or
writers. Depending on the type of concurrency control, however, concurrent
writers may need to operate under mutual exclusion. This issue is addressed
later on. An important class of optimistic concurrency controls is multiversion
concurrency control or multiversioning \cite{reed1978naming,
bernstein1983multiversion}.

There are, however, disadvantages to this type of concurrency control. While
validation is necessary to ensure that a transaction is not involved in data
conflicts, it can also introduce a significant overhead. First, even when there
are no conflicts, validation is conducted nevertheless. Second, validation
usually requires certain metadata about the operations within a transaction. As
a result, validation complexity scales in size of metadata required to
determine conflict freedom. Another drawback is that aborting a conflicting
transaction means that its entire progress is discarded. In this case, valuable
computing resources are wasted.

Despite some drawbacks, optimistic concurrency control has found wide adoption
especially in domains where reads are dominant and conflicts are known to be
infrequent or negligible \cite{carey1986performance, larson2011high,
wu2017empirical}. The scenario of read-dominated workloads has been shown to
apply more often than not \cite{andrei2017sap, wang2017efficiently}. Given its
promising properties, optimistic concurrency control is discussed in more depth
in the subsequent sections.

\subsection{Multiversion Concurrency Control}

\ac{MVCC} or multiversioning, is an optimistic concurrency control method.
Initially, \ac{MVCC} was designed as a solution for concurrency control in
distributed systems \cite{reed1978naming}. However, it was also studied in
non-distributed settings and was soon considered a promising alternative to
locking \cite{kung1981optimistic, bernstein1983multiversion, carey1983multiple,
hadzilacos1986algorithmic, carey1986performance}. Subsequently, \ac{MVCC} has
been adopted in both commercial and non-commercial transaction processing
systems, ranging from general-purpose database systems to high-performance
in-memory databases \cite{larson2011high, lee2013high, diaconu2013hekaton,
schwalb2015efficient}. More recent examples include prototypes of \ac{MMDB} and
\ac{KVS} for \ac{NVRAM} \cite{bailey2013exploring, zhou2016nvht,
oukid2014sofort, schwalb2016hyrise}.

\subsubsection{Principle}

A \emph{version} is a snapshot of a particular data item within a database. In
terms of \ac{KVS}, that would be a value. In traditional concurrency schemes,
there is exactly one version of each item. These are also referred to as
\emph{single-version concurrency controls}. If a transaction issues an update to
a version, then it is performed in-place. In order to ensure isolation, a
transaction has to be protected against concurrent reading or writing.

In a \emph{multiversion concurrency control}, there can be multiple versions of
data items. This fundamentally changes the nature of both read and write
operations. Instead of updating versions in-place, which would have to be
isolated, write operations create copies of existing versions and only modify
those copies. As a result, read operations are implicitly decoupled from
concurrent updates. That means, in particular, that a read operation may access
an item even when newer versions have been committed. In order to keep track of
when versions were created or modified, versions are usually equipped with
timestamps, respectively.

Through its \ac{COW} approach, multiversioning can effectively isolate read
operations from concurrent operations without the need for locking. An important
implication of this circumstance is that read operations never wait for write
operations and vice versa. This is a significant advantage over single-version
schemes especially in applications where reading is much more frequent than
writing. In fact, it has been shown that many workloads are dominated by queries
\cite{krueger2011fast, andrei2017sap}. This is also reflected in the \ac{MMDB}
benchmark TATP which assumes 80 \% of all transactions to be read-only
\cite{larson2011high}. Another useful side effect of multiversioning is that it
forms an implicit logging infrastructure which can be used for recovery
purposes \cite{condit2009better, venkataraman2011consistent}. \ac{MVCC} is
similar to \ac{RCU} \cite{mckenney1998read} but differs in that it is more
general and can manage more than two versions at a time.

While read operations benefit from \ac{COW}, updates can be expensive.
Updating a version incorporates additional overhead for allocating a new version
and copying the original version before modifying it. In the case of \ac{KVS},
however, copies may not be necessary if entire values are updated.

\subsubsection{Visibility}

A central aspect of multiversioning is the concept of \emph{visibility}. Since
there may be multiple versions of a single data item, an operation must first
figure out to which version it should apply. For this purpose, a visibility
property is defined. It determines which versions can be accessed by the
operations of a transaction. In other words, a version is \emph{visible} to a
transaction and its operations if and only if it satisfies the visibility
property. In general, visibility is defined as a predicate over timestamps of
transactions and versions. The concrete definition of the predicate is subject
to the respective \ac{MVCC} protocol.

When a transaction attempts to access an item, it would typically traverse
the versions of that item and determine for each version whether it is visible
to the transaction. Only if a version is visible to the issuing transaction, it
can be selected for reading or writing. The implementation of visibility is of
paramount importance for \ac{MVCC} protocols and the desired isolation level
\cite{larson2011high}.

\subsubsection{Challenges}

The most promising feature of multiversioning is the non-blocking nature of read
operations due to the absence of in-place updates. However, there are also
important problems that need to be addressed in order to leverage the merits of
multiversion concurrency control.

First, the maintenance of more than one version per item implies a significant
overhead in storage. Note that a version may not only contain payload but
additional data such as pointers to adjacent versions. A version may also be
required to hold certain metadata such as timestamps, further increasing the
overall memory footprint. This is relevant especially in areas where memory is
comparatively scarce as is in main-memory databases. However, not all versions
need to be retained. Instead, only the versions that are visible to at least one
transaction are needed. All other versions are considered \emph{stale} and can
be disposed of. This task is usually achieved by a designated garbage collection
mechanism. Although garbage collection may improve the overall memory footprint,
it is also known to have adverse effects on performance.

Second, whenever an item is accessed, the system first has to find a visible
version of the item. Accessing an item may therefore incur a significant runtime
overhead. The overhead mainly depends on the size of the history which, in
theory, is only bounded by the amount of available memory. Employing a garbage
collection mechanism can help by reducing the size of histories. Another
optimization would be to have a transaction keep track of all the versions it
references. This way, visibility would only have to be computed once for each
item.

\subsection{Snapshot Isolation}

The most widely used \ac{MVCC} protocol to date is \ac{SI} \cite{larson2011high,
neumann2015fast}. Originally, \ac{SI} was developed as a response to the
insufficient definition of serializability in the SQL standard
\cite{berenson1995critique}. Since then, it has been deployed in numerous
databases and \acp{KVS} \cite{cahill2009serializable, wu2017empirical}. In the
context of isolation levels, \ac{SI} defines a new isolation level that goes by
the same name. Although \ac{SI} is weaker than serializability, it provides a
good trade-off between performance and consistency. This section describes the
concept of \ac{SI} and its properties.

The core principle of \ac{SI} is that a transaction $t$ only sees a private
snapshot of the database as of when $t$ started. In this sense, the notion of a
snapshot comprises the set of the latest versions that have been committed
before $t$ was invoked. The key to this behavior is the definition of the
visibility property.

\subsubsection{Visibility}

Each version $v$ stores two timestamps $begin_v$ and $end_v$, denoting when $v$
was created and when it was invalidated by an update or deletion, respectively.
The interval $[begin_v,  end_v]$ is called the \emph{lifetime} of $v$. Also,
when a transaction $t$ starts, it is given a timestamp $begin_t$ to capture when
$t$ started. In order to determine which version is visible to the operation of
a transaction, the concurrency control needs to test for each version whether
$t$ started during the lifetime of $v$. The latest valid version satisfying this
property is selected to be visible by the requesting transaction. More
precisely, the version seen by a transaction $t$ is

\[
\operatorname*{max}_{i \in \mathbb{N}}\, \{\, v_i\, |\, begin_{v_i} < begin_t < end_{v_i}\}.
\]

\subsubsection{Conflict Handling}
\label{ch:kvs-cc-conflicts}

Concurrent updates by a transaction $t_2$ that happen after a transaction $t_1$
started, are not included in the snapshot of $t_1$ and are therefore invisible
to $t_1$. If however, $t_1$ decides to also update the same data item, then a
write-write conflict emerges. In this case, the \emph{first-committer-wins}
principle is applied and $t_1$ must abort as $t_2$ also modified the same item
and committed earlier \cite{berenson1995critique}. A popular variant of this
property is the equivalent \emph{first-updater-wins} principle
\cite{fekete2004read, larson2011high}. According to this property, a writer
fails immediately if he is not the first to attempt an update on a given
version, thus making the earlier transaction the first committer. Note, that
based on this strategy, \ac{SI} can be implemented without validation on commit.
It can also reduce the size of individual rollbacks as write-write conflicts are
detected immediately.

\vfill

\subsubsection{Shortcomings}

Under \ac{SI} reads can always be satisfied, provided the requested item exists.
Note that even in the face of concurrent updates, a transaction under \ac{SI}
always sees the same items. This precludes inconsistent reads, read skew, and
phantoms. In addition, \ac{SI} does not allow dirty reads, since snapshots only
contain committed data. However, \ac{SI} does not prevent all possible anomalies
and is therefore not serializable \cite{berenson1995critique, fekete2004read}.
In particular, these conflicts are write skew and non-serializable read-only
transactions.

\paragraph{Write Skew}

The earliest known anomaly of \ac{SI} is write skew. The reason for its
occurrence is that under \ac{SI} a transaction does not see modifications to
versions that have been read during the transaction.

Imagine two transactions $t_1, t_2$ reading two data items $x, y$ constrained by
a predicate $C$. Next, $t_1$ updates $x$ and finds that $C(x^{*}, y)$ still
holds true. At this point $t_2$ is unaware that $x$ has been modified and
updates $y$. Since $t_2$ does not see the modifications of $t_1$, it also
evaluates $C(x, y^{*})$ to be true. Finally, both transactions may commit even
if $C$ is now violated because none observed the others changes (see Figure
\ref{fig:write_skew}). Note that no write-write conflict occurs as both updated
items are distinct. In fact, write skew is said to occur if read sets overlap
while write sets are distinct.

\begin{figure}[!h]
    \centering
    \begin{tabular}{r c c c c c}
        $t_1:$ & $r(x,y)$ &          & $w(x)$ &        & $c$ \\
        $t_2:$ &          & $r(x,y)$ &        & $w(y)$ & $c$ \\
    \end{tabular}
    \caption{Write skew due to transactions $t_1, t_2$ not seeing each others changes.}
    \label{fig:write_skew}
\end{figure}

In the field, write skew has been countered by inducing artificial write
conflicts between transactions that are expected to exhibit write skew
\cite{fekete2005making}.

\paragraph{Non-Serializable Read-Only Anomaly}

Another anomaly was discovered almost 10 years after the introduction of
\ac{SI}. It proved, contrary to common understanding, that even read-only
transactions may not always be serializable \cite{fekete2004read}. The proof
consists of a schedule of three transactions with one being read-only. The
schedule is constructed in a way that only two but never all three
transactions can execute without a conflict.

Suppose a pair of data items $x = 0$ and $y = 0$ and transactions $t_1, t_2$,
and a read-only transaction $t_{RO}$. Further, let $t_1$ compute $y = y - 10$
and also subtract one if $x + y < 0$, while $t_2$ sets $x = x + 20$. The
schedule given in Figure \ref{fig:bad_read_only} shows that while $t_1$ is the
first transaction to start execution, both $t_2$ and $t_{RO}$ start and commit
sequentially before $t_1$ issues its update on $y$. This means that $t_{RO}$
will see the update of $x$ by $t_2$ while $t_1$ does not. According to the
output of $t_{RO}$ ($x = 20$, $y = 0$), a serializable schedule would require
$t_1$ to have been executed after both $t_2$ and $t_{RO}$. This however, is not
possible since $t_1$ would have seen the update of $t_2$ and no penalty would
have been applied as $20 - 10 \geq 0$. Likewise, in order for $t_1$ to yield
$y = -11$, it would have had to be executed before $t_2$ (and $t_{RO}$) which,
however, is not consistent with the output of $t_{RO}$. In fact, the output of
$t_{RO}$ corresponds the exact opposite serial ordering as do those of
$t_1$ and $t_2$.

\begin{figure}[h!]
    \centering
    \begin{tabular}{r c c c c}
    $t_1:$    & $r(x,y)$ &                   &              & $w(y)\, c$ \\
    $t_2:$    &          & $r(x)\, w(x)\, c$ &              &            \\
    $t_{RO}:$ &          &                   & $r(x,y)\, c$ &
    \end{tabular}
    \caption{Transaction $t_{RO}$ is read-only but not serializable.}
    \label{fig:bad_read_only}
\end{figure}

% \todo[inline]{make intermediate values more visible..}

Both symptoms can be attributed to the fact, that \ac{SI} fails to observe
read-write conflicts. When a transaction requests an item, it reads the latest
version that has been committed before the transaction started. This way, a
transaction always reads the same version even if a newer version has been
committed concurrently. This relieves the system from locking a version when
accessing it. The downside is that every transaction is effectively isolated
from any concurrent modifications. As a consequence, transactions may
successfully commit even if one or more versions they have read has been updated
in the meantime. All anomalies known to be emitted by \ac{SI} can be reduced to
read-write conflicts.

\subsection{Serializable MVCC}

The previous section describes the \ac{MVCC} protocol \ac{SI}. \ac{SI} provides
affordable isolation but fails to preclude all non-serializable schedules such
as write skew. Nevertheless, \ac{SI} is the most widely adopted \ac{MVCC}
implementation \cite{larson2011high, bailey2013exploring, neumann2015fast}. In
some cases, serializing alternatives are available but disabled by default. This
policy is often motivated by significant performance benefits compared to
strictly serializing concurrency control \cite{cahill2009serializable}. Others
argue that \ac{SI} anomalies may be negligible as even renowned ACID-compliant
benchmarks such as TPC-C do not expose them \cite{fekete2005making}. However,
the need for data integrity should not depend on benchmarks failing to prove
it. Therefore, there are efforts to overcome the weaknesses of \ac{SI} and
provide strong consistency without falling back to pessimistic approaches. This
section presents an overview of approaches to make serializing \ac{MVCC}
affordable.

The main reason for non-serializable schedules in \ac{SI} is that it cannot
detect read-write conflicts. While \ac{SI} keeps track of each transactions'
updates for installment on commit, reads are not tracked. Therefore, a naive
approach to achieve serializable schedules with \ac{SI} is to track the read
operations of each transaction. In doing so, read-write conflicts can be
detected during validation by looking at the timestamps of all versions read. If
at least one of these versions has been invalidated after the transaction
started, then a read-write conflict has emerged. Unless the conflicting updater
is still running, the reading transaction must abort. This method is both simple
and effective but introduces significant overhead. Note that traditional \ac{SI}
does not perform any validation if first-updater-wins is used for precluding
write-write conflicts. The additional overhead especially affects read-mostly
transactions which is undesirable in read-dominated environments. Since the
latter is where \ac{SI} has been especially successful, tracking reads followed
by validation is often stated as prohibitively expensive
\cite{cahill2009serializable}. Lacking viable alternatives, research interests
primarily focus on mitigating the footprint of read tracking and validation.

\subsubsection{Exploiting Query Languages}

Several authors have proposed to detect conflicts from query language statements
\cite{fekete2005making, faleiro2015rethinking, neumann2015fast}. For instance, a
common strategy to prevent write skew in \ac{SI} is to inject detectable
conflicts whenever the required access patterns are detected
\cite{fekete2005making}. Others have found efficient ways to determine whether
an item is included in a range query, thus improving validation time
\cite{neumann2015fast}. Although intriguing, these approaches cannot be applied
to \acp{KVS} since they operate through ad hoc queries instead of query
languages.

\subsubsection{Reducing Contention}

A major challenge for high-performance implementations of \ac{MVCC} and \ac{SI},
in particular, is that important aspects such as timestamping or validation are
often centralized which can cause considerable contention. Therefore, a
substantial amount of research is dedicated to providing protocols in the spirit
of \ac{SI} but lower contention. Note that, in contrast to locking, contention
for \ac{MVCC} denotes much smaller intervals of a transaction's lifetime.

A major bottleneck in \ac{MVCC} implementations is validation
\cite{tu2013speedy, bailey2013exploring, ding2015centiman,
faleiro2015rethinking, wang2017efficiently, zhou2017posterior}. The reason is
that, validation typically requires mutual exclusion since concurrent updates to
items from the read set could falsify the validation result. As a result,
validation does not scale, thus inhibiting transaction throughput. Therefore,
many authors propose protocols featuring concurrent validation
\cite{bailey2013exploring, ding2015centiman, faleiro2015rethinking,
wang2017efficiently}. Parallel validation has already been proposed in the early
days of \ac{MVCC} \cite{kung1981optimistic}, but received renewed attention
lately.

Another point of contention is the assignment of timestamps. A typical \ac{SI}
implementation requires multiple timestamps for both versions and transactions.
However, most implementations rely on a global assignment policy which
introduces a significant contention on multi-core systems \cite{tu2013speedy,
zhou2017posterior}. First, concurrency is reduced as mutual exclusion is
required to guarantee strictly monotone timestamps. Second, the \ac{CPU} must be
informed that changes to the cached timestamp counter must be globally visible.
This is usually done with fences which can further reduce performance. In
response, some authors have proposed protocols featuring decentralized timestamp
assignment \cite{tu2013speedy, zhou2017posterior}. Further sources of contention
addressed in these works are transaction id assignment and shared memory access,
in general.
