% The previous sections have introduced the fundamentals on transactions,
% serializability, and modern concurrency control protocols. The aim is to design
% and implement a robust concurrency control for an in-memory \ac{KVS} based on
% \ac{NVRAM}. Hence, the chapter concludes with an overview on concurrency in
% \ac{KVS} for \ac{NVRAM}.

The previous sections describe the fundamentals on transactions,
serializability, and modern concurrency control protocols.  Since the aim of
this thesis is to design and implement a concurrent \ac{KVS} for \ac{NVRAM},
this chapter concludes with an overview of existing \acp{KVS} for \ac{NVRAM}.

\ac{NVRAM} is not yet commercially available, but there is a number of studies
involving \ac{KVS} and \ac{NVRAM}. In essence, there are three research
branches:

\begin{itemize}
    \item evaluation of programming facilities for \ac{NVRAM}
    \item evaluation of implications of \ac{NVRAM} for existing \acp{KVS}
    \item design of \ac{NVRAM}-aware \ac{KVS}
\end{itemize}

While, ultimately, all branches aim to understand the implications of
\ac{NVRAM}, individual scenarios and approaches differ substantially. As for the
aim of this work, focus is given to \ac{KVS} specifically designed for
\ac{NVRAM}.

% The remaining areas are omitted as none of the respective studies in
% \cite{venkataraman2011consistent, pelley2013storage, volos2014aerie,
% lersch2017analysis, malinowski2017using} give insight on the implications of
% \ac{NVRAM} for transactions and concurrency control.
% \todo[inline]{Try to remove this citation blunder}

% As pointed out in Chapter {ch:nvram}, working with \ac{NVRAM} requires
% additional programming mechanisms not needed for \ac{DRAM} or conventional
% durable storage. Issues comprise recoverable memory mappings, consistency
% ensurance, and memory management accounting for the byte-addressable nature of
% \ac{NVRAM}. In this regard, a number of facilities to address these issues have
% been proposed. Due to their recent prevalence in high-performance environments,
% some these techniques have been evaluated against \ac{KVS}. In
% {venkatamaran2011consistent} \ac{NVRAM}-ware b-trees based on versioning were
% designed and integrated into the Redis \ac{KVS} for evaluation. However, the
% described implementation of versioning is was only meant to replace logging and
% all simulations were conducted on a single thread.

At the moment, there are only few designs for \ac{NVRAM}-aware \ac{KVS}: Echo
\cite{bailey2013exploring}, NVHT \cite{zhou2016nvht}, and MetraDB
\cite{marmol2016nonvolatile}. Below, the architecture of these \acp{KVS} is
outlined with an emphasis on transactions and concurrency control.

\subsubsection{Echo}

Echo is one of the earliest \acp{KVS} designed to leverage the benefits of
\ac{NVRAM} \cite{bailey2013exploring}. It aims to achieve scalable
high-performance transactions through optimistic concurrency control and
light-weight persistence management. For this purpose, Echo uses a two-level
store architecture featuring both volatile and non-volatile \ac{RAM}. Only
committed data is written to \ac{NVRAM}, while uncommitted data is kept in
volatile \ac{RAM}. Moreover, there are two groups of threads: workers and
masters. The former execute the operations of transactions and buffer their
updates, whereas masters are in charge of committing them. By keeping updates
from \ac{NVRAM} until commit, persistence guarantees need to be enforced less
often and degradation effects such as wearing are reduced. Also, updates are
buffered locally in each thread which avoids contention on shared data. In order
to ensure isolation between concurrent transactions, Echo employs classic
\ac{SI}. As a result, some non-serializable schedules of transactions are
permitted.

Contrary to many other works, Echo resolves write-write conflicts
using the original first-committer-wins strategy. This eliminates the need to
acquire exclusive ownership when updating an item but leads to late conflict
detection with larger rollbacks. The core data structure beneath Echo is a hash
table which maps keys to version histories. Concerning \ac{NVRAM} consistency,
Echo settles for existing instruction sets with cache line flushes and store
fences. The authors further anticipate a hardware capability, such as the now
obligatory \ac{ADR}, which ensures that cache flushes always become durable.
During the evaluation of Echo, its design is shown to provide strong durability
and consistency while providing performance characteristics of volatile
main-memory \ac{KVS}. However, the evaluation is carried out on volatile
\ac{RAM} and does account for latencies of \ac{NVRAM} and cache flushes. Also,
the separation of worker and master threads is effectively removed, as workers
\emph{become} masters and commit their own transactions.

Being one of the most meticulously designed and documented \acp{KVS} for
\ac{NVRAM}, Echo is a guiding example for this thesis. It achieves high
performance through its two-level architecture and optimistic concurrency
control. Drawbacks include non-serializing \ac{SI} and the first-committer-wins
strategy. In the end, the authors' evaluation is not entirely conclusive as some
design aspects were considerably altered.

\subsubsection{NVHT}

The goal of NVHT is to leverage \ac{NVRAM} to achieve fast updates without
sacrificing durability \cite{zhou2016nvht}. The architecture of NVHT differs
significantly from that of Echo. Most importantly, NVHT only supports
transactional updates instead of full-grown transactions. That means, that each
operation is implicitly committed, which is sometimes called auto-commit.
Nevertheless its architecture may give valuable insights into the design of a
\ac{KVS} for \ac{NVRAM}. Similar to Echo, NVHT relies on a hybrid memory
architecture consisting of both volatile \ac{RAM} and \ac{NVRAM}. However, Echo
is a two-level store where each update is buffered in volatile \ac{RAM} until it
is committed to \ac{NVRAM}. In NVHT, on the other hand, all updates are directly
written to \ac{NVRAM}. Similar to Echo, the key data structure of NVHT is an
\ac{NVRAM}-aware hash table. However, NVHT does not use multiversioning to
control concurrent operations. Instead, it applies half-coarse synchronization
by locking individual partitions of the hash table upon access.

Despite NVHT having shown good performance against prominent \acp{KVS}, there
are some problems with its design. First, accessing an item in the hash map
locks an entire partition. The authors point out that this design decision can
increase lock contention, thus reducing concurrency. More importantly, NVHT does
not address the issues of ordering and deferred write-back on \ac{NVRAM}.
Instead, the authors merely devise a kind of commit record whose presence or
absence denotes whether the preceding item should be taken into account.
However, without enforcing an ordering on store operations, the commit record
could be durable before the actual item. Also, in order to ensure durability,
cache lines need to be flushed. Omitting these precautions could lead to
inconsistent data.

\subsubsection{MetraDB}

In \cite{marmol2016nonvolatile} \acp{KVS} are proposed as a middleware for
\ac{NVRAM}. As an example, the authors present MetraDB, a solution for
distributed storage based on \ac{NVRAM}. Like NVHT, MetraDB is a single-level
store which means that changes are written to \ac{NVRAM} immediately. Since
MetraDB is required to support overlapping namespaces, the \ac{KVS} consists of
multiple hash tables referred to as containers. While this seems to complicate
memory allocation schemes for \ac{NVRAM}, the authors assert that the size of
hash tables is fixed. There are two kinds of transaction is MetraDB:
transactions on containers and transactions on metadata. Since containers are
designed for single-threaded access, transactions on containers need not worry
about isolation. In contrast, transactions on metadata can be multi-threaded.
These transactions are responsible for adding and removing containers. The
authors point out that these operations are infrequent and, therefore, need not
be very efficient. For this reason, transactions on metadata are protected by a
global lock on the collection of containers. Recovery is based on redo logging,
since undo logging would require additional writes to \ac{NVRAM} during a
transaction. With redo logging, log entries only need to be flushed on commit
which can be optimized, for example with non-temporal stores. In an evaluation,
MetraDB has shown superior performance when compared to several lookup data
structures contained in NVML (currently known as PMDK). Given that NVML provides
general-purpose \ac{NVRAM} facilities, as opposed to MetraDB's use case
optimizations, the comparison is not always fair. Still, the evaluation exposes
scalability issues related to the operating system rather than the \ac{KVS}
itself. Upon finding scalability to falter when increasing the number of
\ac{CPU} cores, the authors link the issue with kernel locking on memory-mapped
files.

When compared to Echo or NVHT, MetraDB is a very application-specific \ac{KVS}.
Especially its use of containers, which can only be accessed by one thread at a
time, conveys little guidance for the design of a \ac{KVS} with concurrent
transactions. Apart from a few isolated cases, single-threading and very coarse
locking may not be the most promising approaches to achieve high transaction
throughput.
