The previous sections have introduced the fundamentals on transactions,
serializability, and modern concurrency control protocols. The aim is to design
and implement a robust concurrency control for an in-memory KVS based on NVRAM.
Hence, the chapter concludes with an overview on concurrency in KVS for NVRAM.

NVRAM is not yet commercially available, but there is a number of studies
involving KVS and NVRAM. In essence, there are three research branches:

\begin{itemize}
    \item Evaluation of programming facilities for NVRAM
    \item Evaluation of implications of NVRAM for existing KVS
    \item Design of NVRAM-aware KVS
\end{itemize}

While, ultimately, all branches aim to understand the implications of NVRAM,
individual scenarios and approaches differ substantially. As for the aim of this
work, focus is given to KVS specifically designed for NVRAM. The remaining areas
are omitted as none of the respective studies in
\cite{venkataraman2011consistent, pelley2013storage, volos2014aerie,
lersch2017analysis, malinowski2017using} give insight on the implications of
NVRAM for transactions and concurrency control.

At the moment, there are few designs for NVRAM-aware KVS: Echo
\cite{bailey2013exploring}, NVHT \cite{zhou2016nvht}, and MetraDB
\cite{marmol2016nonvolatile}. Below, the architecture of these KVS is outlined
with an emphasis on transactions and concurrency control.

\subsubsection{Echo}

Echo is one of the earliest KVS designed to leverage the benefits of NVRAM
\cite{bailey2013exploring}. It aims to achieve scalable high-performance
transactions through optimistic concurrency control and light-weight persistence
management. For this purpose, Echo uses a two-level store architecture featuring
both volatile and non-volatile RAM. Only committed data are written to NVRAM,
while uncommitted data are kept in volatile RAM. Moreover, there are two groups
of threads: workers and masters. The former execute the operations of
transactions and buffer their updates, whereas masters are in charge of
committing them. By keeping updates from NVRAM until commit, persistence
guarantees need to be enforced less often and degradation effects such as
wearing are reduced. Also, updates are buffered locally in each thread which
avoids contention on shared data. In order to ensure isolation between
concurrent transactions, Echo employs classic snapshot isolation. As a result,
some non-serializable schedules of transactions are permitted. Contrary to many
other works, Echo resolves write-write conflicts using the original
first-committer-wins strategy. This eliminates the need to acquire exclusive
ownership when updating an item but leads to late conflict detection with larger
rollbacks. The core data structure beneath Echo is a hash table which maps keys
to version histories. Concerning NVRAM consistency, Echo settles for existing
instruction sets with cache line flushes and store fences. The authors further
anticipated a hardware capability, such as the now obligatory ADR, which ensures
that cache flushes always become durable. During the evaluation of Echo, its
design was shown to provide strong durability and consistency while providing
performance characteristics of volatile main-memory KVS. However, the evaluation
was carried out on volatile RAM and did not account for latencies of NVRAM and
cache flushes. Also the separation of worker and master threads was simplified,
as workers would temporarily become a master to commit its transaction.

Being one of the most meticulously designed and documented KVS for NVRAM, Echo
is a guiding example for this thesis. It achieves high performance through its
two-level architecture and optimistic concurrency control. Drawbacks include
non-serializing SI and the first-committer-wins strategy. Also the authors'
evaluation is not entirely conclusive as some aspects were considerably altered.

\subsubsection{NVHT}

The goal of NVHT is to leverage NVRAM to achieve fast updates without
sacrificing durability \cite{zhou2016nvht}. The architecture of NVHT differs
significantly from that of Echo. Most importantly, NVHT only supports
transactional updates instead of full-grown transactions. That means, that each
operation is implicitly committed, which is sometimes called auto-commit.
Nevertheless its architecture may give valuable insights into the design of a
KVS for NVRAM. Similar to Echo, NVHT relies on a hybrid memory architecture
consisting of both volatile RAM and NVRAM. However, Echo is a two-level store
where each update is buffered in volatile RAM until it is committed to NVRAM. In
NVHT, on the other hand, all updates are directly written to NVRAM. Similar to
Echo, the key data structure of NVHT is a NVRAM-aware hash table. However, NVHT
does not use multiversioning to control concurrent operations. Instead, it
applies half-coarse synchronization by locking individual partitions of the hash
table upon access.

Despite NVHT having shown good performance against prominent KVS, there are some
problems with its design. First, accessing an item in the hash map locks an
entire partition. The authors correctly point out that this design decision can
increase lock contention, thus reducing concurrency. More importantly, NVHT does
not address the issues of ordering and deferred write-back on NVRAM. Instead,
the authors merely devise a kind of commit record whose presence or absence
denotes whether the preceding item should be taken into account. However,
without enforcing an ordering on store operations, the commit record could be
durable before the actual item. Also, in order to ensure durability, cache lines
need to be flushed. Omitting these precautions could lead to inconsistent data.
These issues are not present in Echo.

\subsubsection{MetraDB}

In \cite{marmol2016nonvolatile} KVS are proposed as a middleware for NVRAM. As
an example the authors, present MetraDB, a solution for distributed storage
based on NVRAM. Like NVHT, MetraDB is a single-level store which means that
changes are written to NVRAM immediately. Since MetraDB is required to support
overlapping namespaces, the KVS consists of multiple hash tables referred to as
containers. While this seems to complicate memory allocation schemes for NVRAM,
the authors assert that the size of hash tables is fixed. There are two kinds of
transaction is MetraDB: transactions on containers and on meta data. Since
containers are designed for single-threaded access, transactions on containers
need not worry about isolation. In contrast, transactions on meta data can be
multi-threaded. These transactions are responsible for adding and removing
containers. The authors point out that these operations are infrequent and,
therefore, need not be very efficient. For this reason, transactions on meta
data are protected by a global lock on the collection of containers. Recovery is
based on redo logging, since undo logging would require additional writes to
NVRAM during a transactions. With redo logging, log entries only need to be
flushed on commit which can be optimized, for example with non-temporal stores.
In an evaluation, MetraDB has shown superior performance when compared to
several lookup data structures contained in Intel's NVML. Given that NVML
provides general-purpose NVRAM facilities, as opposed to MetraDB's use-case
optimizations, the comparison is not always fair. Still, the evaluation exposes
scalability issues related to the operating system rather than the KVS itself.
Upon finding scalability to falter when increasing the number of CPU cores, the
authors link the issue with kernel locking on memory-mapped files.

When compared to Echo or NVHT, MetraDB is a very application-specific KVS.
Especially, its use of containers, which can only be accessed by one thread at a
time, conveys little guidance for the design of a KVS with concurrent
transactions. Apart from a few isolated cases, single-threading and very coarse
locking may not be the most promising approaches to achieving high transaction
throughput \cite{molina1992main}.

\subsubsection{Summary}

\todo[inline]{TODO}

% As pointed out in Chapter {ch:nvram}, working with NVRAM requires additional programming mechanisms not needed for DRAM or conventional durable storage. Issues comprise recoverable memory mappings, consistency ensurance, and memory management accounting for the byte-addressable nature of NVRAM. In this regard, a number of facilities to address these issues have been proposed. Due to their recent prevalence in high-performance environments, some these techniques have been evaluated against KVS. In {venkatamaran2011consistent} NVRAM-ware b-trees based on versioning were designed and integrated into the Redis KVS for evaluation. However, the described implementation of versioning is was only meant to replace logging and all simulations were conducted on a single thread.
